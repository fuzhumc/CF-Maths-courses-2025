\chapter{群论}

\section{元群}

\begin{deff}
    设$E$为集合.称映射$*: \begin{array}[t]{ccc} E \times E & \to & E \\ (x,y) & \mapsto & x*y \end{array}$为$E$上的一个\textbf{内部合成法则}或\textbf{(内部)运算}.

    称$(E, *)$为一个\textbf{元群}.
\end{deff}

\begin{deff}
    设$E$为集合,$*$为$E$上的内部合成法则.称$*$是\textbf{结合的},若$\forall x, y, z \in E, x * (y * z) = (x * y) * z$.
\end{deff}

\begin{deff}
    设$E$为集合,$*$为$E$上的内部合成法则.称$*$是\textbf{交换的},若$\forall x, y \in E, x * y = y * x$.
\end{deff}

\begin{rqq}
    一般用$*$表示不交换的运算,用$+$表示交换的运算.
\end{rqq}

\begin{deff}
    设$E$为集合,$\oplus, \otimes$为$E$上的两个内部合成法则,称$\otimes$对$\oplus$是\textbf{交换的},若$\forall x, y, z \in E, x \otimes (y \oplus z) = (x \otimes y) \oplus (x \otimes z), (x \oplus y) \otimes z = (x \otimes z) \oplus (y \otimes z)$.
\end{deff}

\begin{deff}
    设$E$为集合,$*$为$E$上的内部合成法则,$a \in E$.称$a$对$*$是\textbf{正则的},若$\forall x, y \in E, (a * x = a * y \Rightarrow x = y) \land (x * a = y * a \Rightarrow x = y)$.
\end{deff}

\begin{deff}
    设$E$为集合,$*$为$E$上的内部合成法则,$e \in E$.称$e$对$*$是\textbf{中性的},若$\forall x, y \in E, x * e = e * x = x$.
\end{deff}

\begin{prop}
    中性元若存在则必唯一.
\end{prop}

\begin{proof}
    设$e, e'$均为$E$的中性元,则$e' = e' * e = e$.
\end{proof}

\begin{deff}
    设$E$为集合,$*$为$E$上的内部合成法则,$e$为其中性元,$a \in E$.称$e$对$*$是\textbf{可逆的},若$\exists b \in E, a * b = b * a = e$.

    此时称$b$为$a$的\textbf{逆元},记为$b = a^{-1}$.
\end{deff}

\begin{prop}
    $(E, *)$结合时,逆元若存在则必唯一.
\end{prop}

\begin{proof}
    设$x \in E$,$y, z$均为$x$的逆元,则$z = z * e = z * (x * y) = (z * x) * y = e * y = y$.
\end{proof}

\begin{prop}
    设$(E, *)$结合,$a, b \in E$均可逆,则$a * b$也可逆,且$(a * b)^{-1} = b^{-1} * a^{-1}$.
\end{prop}

\begin{prop}
    设$(E, *)$结合且有中性元,则$E$中的可逆元均正则.
\end{prop}

\begin{deff}
    设$E$为集合,$*$为$E$上的内部合成法则,$F \subseteq E$.称$F$在$*$下是\textbf{稳定的},若$\forall x, y \in F, x * y \in F$.
\end{deff}

\section{群}

\begin{deff}
    设$G$为集合,$*$为$G$上的内部合成法则.称$(G, *)$为一个\textbf{群},若
    \begin{enumerate}
        \item $*$结合.\label{def:group1}
        \item $G$中有中性元.\label{def:group2}
        \item $G$中的每个元素均可逆.\label{def:group3}
    \end{enumerate}
\end{deff}

\begin{deff}
    设$(G, *)$为群,称$(G, *)$为一个\textbf{交换群}或\textbf{Abel群},若$*$交换.
\end{deff}

\begin{rqq}
    称满足\ref{def:group1}的元群为\textbf{半群}.
    称满足\ref{def:group1}和\ref{def:group2}的元群为\textbf{幺半群}.
\end{rqq}

\begin{exer}
    设$(G, *)$为群,$A$为非空集合,定义$\calF(A, G)$上的一个内部合成法则如下:$$\otimes: \begin{array}[t]{ccc} \calF(A, G) \times \calF(A, G) & \to & \calF(A, G) \\ (f, g) & \mapsto & f \otimes g: \begin{array}[t]{ccc} A & \to & G \\ a & \mapsto & f(a) * g(a) \end{array} \end{array}$$
    
    证明:$(\calF(A, G), \otimes)$为群.
\end{exer}

\subsection*{群的一些例子}

待整理.

\section{子群}

\begin{deff}
    设$(G, *)$为群,$H \subseteq G$.称$(H, *)$为$(G, *)$的\textbf{子群},若$(H, *)$为群.记为$H \leqslant G$.
\end{deff}

\begin{rqq}
    $G$与$\{ e \}$均为$G$的子群,称为\textbf{平凡子群}.
\end{rqq}

\begin{deff}
    设$H \leqslant G$,称$H$为\textbf{真子群},若$H \neq G$.记为$H < G$.
\end{deff}

\begin{prop}
    设$(G, *)$为群,$H \leqslant G$,$e$为$G$的中性元,则$e \in H$且$e$为$H$的中性元.
\end{prop}

\begin{prop}
    设$(G, *)$为群,$H$为$G$的非空子集,则有$H \leqslant G \iff \forall a, b \in H, a * b^{-1} \in H$.
\end{prop}

\begin{prop}
    设$(G, *)$为群,$H$为$G$的非空子集,则有$H \leqslant G \iff \forall a, b \in H, a * b \in H, a^{-1} \in H$.
\end{prop}

\begin{prop}
    设$(G, *)$为群,$(H_i)_{i \in I}$为$G$的一族子群,则$\bigcap\limits_{i \in I} H_i \leqslant G$.
\end{prop}

\section{群同态}

\begin{deff}
    设$(G, *), (H, \cdot)$为群,$\varphi \in \calF(G, H)$.称$\varphi$为\textbf{群同态},若$\forall x, y \in G, \varphi(x * y) = \varphi(x) \cdot \varphi(y)$.
\end{deff}

\begin{prop}
    设$\varphi: G \to H$为群同态,$e_G, e_H$分别为$G, H$的中性元,则$\varphi(e_G) = e_H$.
\end{prop}

\begin{prop}
    由题意知$\forall a \in G, a * e_G = a$.

    则$\forall g \in G, \varphi(a) \cdot \varphi(e_G) = \varphi(a) = \varphi(a) \cdot e_H$.

    故$\varphi(e_G)=  e_H$.
\end{prop}

\begin{prop}
    设$\varphi: G \to H$为群同态,$g \in G$,则$\varphi(g^{-1})=\varphi(g)^{-1}$.
\end{prop}

\begin{deff}
    设$\varphi: G \to H$为群同态,称$\varphi^{-1}(e_H) = \{ g \in G \mid \varphi(g) = e_H \}$为$\varphi$的\textbf{核},记为$\ker \varphi$.
\end{deff}

\begin{deff}
    设$(G, *)$为群,$H \leqslant G$.称$H$为$G$的\textbf{正规子群},若$\forall g \in G, gHg^{-1} = H$.记为$H \lhd G$.
\end{deff}

\begin{prop}
    设$(G, *)$为群,则有
    \begin{enumerate}
        \item 若$G$交换,则其子群均为正规子群.
        \item $\{ e \} \lhd G, G \lhd G$.
        \item $gH = Hg \iff gHg^{-1} = H$.
    \end{enumerate}
\end{prop}

\begin{prop}
    设$(G, *)$为群,$H$为$G$的正规子群,$h \in H$,则不一定有$g * h = h * g$.
\end{prop}

\begin{prop}
    设$\varphi: G \to H$为群同态,则$\ker \varphi \lhd G$.
\end{prop}

\begin{prop}
    设$\varphi: G \to H$为群同态,则$\varphi$为单射$\iff \ker \varphi = \{ e \} \iff \lvert \ker \varphi \rvert = 1$.
\end{prop}

\begin{deff}
    设$\varphi: G \to H$为群同态,
    \begin{enumerate}
        \item 称$\varphi$为\textbf{单同态},若$\varphi$为单射.
        \item 称$\varphi$为\textbf{满同态},若$\varphi$为满射.
        \item 称$\varphi$为\textbf{群同构},若$\varphi$为双射.
    \end{enumerate}
\end{deff}

\begin{deff}
    设$\varphi: G \to H$为群同态(群同构),称$\varphi$为\textbf{自同态}(\textbf{自同构}),若$(G, *) = (H, \cdot)$.
\end{deff}

\begin{deff}
    设$(G, *)$为群
    \begin{enumerate}
        \item 记$G$上的自同态的全体为$\mathrm{End} G$.
        \item 记$G$上的自同构的全体为$\mathrm{Aut} G$.
    \end{enumerate}
\end{deff}

\begin{prop}
    设$(G, *)$为群,$\circ$为同态的复合运算,则$(\mathrm{Aut} G, \circ)$为群.
\end{prop}

\begin{rqq}
    $(\mathrm{End} G, \circ)$不一定为群.
\end{rqq}

\begin{prop}
    设$\varphi: G \to H$为群同态,则$\varphi$为群同构$\iff$存在群同态$\psi: H \to G, \psi \circ \varphi = \mathrm{id}_G, \varphi \circ \psi = \mathrm{id}_H$.
\end{prop}

\begin{prop}
    设$G_1, G_2$为群,$p_1: G_1 \times G_2 \to G_1, p_2: G_1 \times G_2 \to G_2$为群同态,
    
    则$\ker p_1 = \{ e_1 \} \times G_2, \ker p_2 = G_1 \times \{ e_2 \}$均为$G_1 \times G_2$的正规子群.
\end{prop}

\section{商群}

\begin{deff}
    设$(G, *)$为群,$H \leqslant G, a \in G$.
    \begin{enumerate}
        \item 称$\{ a * h \mid h \in H \}$为$H$在$G$中的一个\textbf{左陪集},记为$aH$.
        \item 称$\{ h * a \mid h \in H \}$为$H$在$G$中的一个\textbf{右陪集},记为$Ha$.
    \end{enumerate}
\end{deff}

\begin{prop}
    设$H \leqslant G, a, b \in G$,则$aH$与$bH$要么相等,要么不交.
\end{prop}

\begin{deff}
    记$G / H$为$H$在$G$中的左陪集的全体.
\end{deff}

\begin{prop}\label{prop3}
    $G$为$G / H$中元素的无交并.
\end{prop}

\begin{rqq}
    Proposition \ref{prop3}给出了$G$的一个划分,即给出了$G$上的一个等价关系.
\end{rqq}

\begin{deff}
    设$(G, *)$为群,$H \leqslant G$,称$\lvert G / H \rvert$为$H$在$G$中的\textbf{指标}.若$G / H$有限,则记为$[G : H]$.
\end{deff}

\begin{thm}[Lagrange]
    设$(G, *)$为有限群,$H \leqslant G$,则$\lvert G \rvert = [G : H] \cdot \lvert H \rvert$.
\end{thm}

\begin{prop}
    设$(G, *)$为群,$H \leqslant G$,则有
    \begin{enumerate}
        \item 若$H \lhd G$,则$\forall a \in G, aH = Ha$.
        \item $H \lhd G \iff H$的任何左陪集都是右陪集.
    \end{enumerate}
\end{prop}

\begin{prop}
    设$(G, *)$为群,$H \leqslant G$,定义$G / H$上的一个内部合成法则$*'$如下:
    $$*': \begin{array}[t]{ccc} G / H \times G / H & \to & G / H \\ (aH, bH) & \mapsto & (a * b)H \end{array}$$
    
    则$*'$是良定义的,进而$(G / H, *')$为群,称为\textbf{商群}.
\end{prop}

\begin{prop}
    商群$(G / H, *')$的中性元为$eH$,元素$aH$的逆元为$a^{-1}H$.
\end{prop}

\begin{prop}
    存在自然投影$\pi: \begin{array}[t]{ccc} G & \to & G / H \\ a & \mapsto & aH \end{array}$满足$\pi(a * b) = \pi(a) *' \pi(b)$
\end{prop}

\begin{rqq}
    后面我们不再区分$G$上的内部合成法则$*$与$G / H$上的内部合成法则$*'$.
\end{rqq}

\begin{thm}[典范分解定理]\label{thm:canonical_decomposition_group}
    设$\varphi: G \to H$为群同态,则存在唯一群同构$\bar{\varphi}: G / \ker \varphi \to \Im \varphi$使得$\varphi = i \circ \bar{\varphi} \circ \pi$.
    其中$i: \Im \varphi \to H$为自然嵌入(单同态),$\pi: G \to G / \ker \varphi$为自然投影(满同态).
\end{thm}

\begin{rqq}
    Théorème \ref{thm:canonical_decomposition_group}的意义为:将群同态的研究归结为对单同态(自然嵌入$i$)与满同态(自然投影$\varphi$)的研究.
\end{rqq}

\begin{cor}
    若$G$有限,则$\lvert G \rvert = \lvert \ker \varphi \rvert \lvert \Im \varphi \rvert$.
\end{cor}

\begin{cor}
    设$\varphi: G \to H$为群同态,$G, H$均有限,$(\lvert G \rvert, \lvert H \rvert) = 1$,则$\varphi$为平凡同态.即$\forall g \in G, \varphi(g) = e_H$.
\end{cor}

\begin{thm}
    设$H, N \lhd G, N \leqslant H$,则$(G / N) / (H / N) \cong G / H$.
\end{thm}

\section{泛性质}

\begin{prop}\label{prop1}
    设群同态$\varphi: A \to B, i: K \to A$满足$\varphi \circ i = 0$,则对于任意$\psi: C \to A$且$\varphi \circ \psi = 0$,存在唯一$\alpha: C \to K$,使得$\psi = i \circ \alpha$.
\end{prop}

\begin{rqq}
    Proposition \ref{prop1}就称为核的\textbf{泛性质}.
\end{rqq}

\begin{rqq}
    Proposition \ref{prop1}唯一确定了一个$K$及一个单同态$i: K \to A$,因此可以把Proposition \ref{prop1}作为定义核的方式.
\end{rqq}

\begin{deff}
    设$\varphi: A \to B$为群同态,称$B / \Im \varphi$为$\varphi$的\textbf{余核},记为$\coker\varphi$.
\end{deff}

\begin{prop}
    设$\varphi: A \to B$为群同态,则$\varphi$为满同态$\iff \lvert \coker \varphi \rvert = 1$.
\end{prop}

\begin{prop}
    设群同态$\varphi: A \to B, \pi: B \to \coker \varphi$满足$\varphi \circ i = 0$,则对于任意$\psi: B \to C$且$\psi \circ \varphi = 0$,存在唯一$\beta: \coker \varphi \to C$,使得$\psi = \beta \circ \pi$.
\end{prop}