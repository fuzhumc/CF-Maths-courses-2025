\chapter{有理分式}

在本章中,我们记$\bbK$为$\bbR$或$\bbC$.

\section{有理分式域}

我们定义交换环$\bbK[x]$上的有理分式域$\bbK(x)$,即满足下列条件的集合.

\begin{itemize}
    \item 元素$F \in \bbK(x)$可写成形式$F = \dfrac{P}{Q}$,其中$P, Q \in \bbK[x]$且$Q \neq 0$.
    \item 设$P, Q, P_1, Q_1 \in \bbK[x]$且$Q \neq 0, Q_1 \neq 0$,
        则$$\dfrac{P}{Q} = \dfrac{P_1}{Q_1} \iff PQ_1 = P_1Q$$
    \item $\bbK(x)$包括$\bbK[x]$,多项式$P$视为有理分式$\dfrac{P}{1}$.
    \item 具有以下运算规则:
        \begin{itemize}
            \item $\dfrac{P}{Q} +\dfrac{P_1}{Q_1} = \dfrac{PQ_1 + P_1Q}{QQ_1}$.
            \item $\dfrac{P}{Q} \cdot \dfrac{P_1}{Q_1} = \dfrac{PP_1}{QQ_1}$.
        \end{itemize}
        其中$P, Q, P_1, Q_1 \in \bbK[x]$且$Q, Q_1 \neq 0$.
\end{itemize}

这样$(\bbK(x), +, \cdot)$为域,称为\textbf{有理分式域(corps des fractions rationnelles)},
其中的元素称为\textbf{有理分式(fraction rationnelle)}.

\begin{rqq}
    设$P, Q, R \in \bbK[x], Q, R \neq 0$,则$\dfrac{PR}{QR} = \dfrac{P}{Q}$.

    设$P, Q \in \bbK[x]^*$,则$\left( \dfrac{P}{Q} \right)^{- 1} = \dfrac{Q}{P}$.
\end{rqq}

\begin{deff} $\ $
    \begin{itemize}
        \item 设$F = \dfrac{P}{Q} \in \bbK(x)$,称$\dfrac{P}{Q}$为$F$的
            \textbf{不可约形式(forme irréductible)},若$P, Q$互素.
        \item 设$F = \dfrac{P}{Q} \in \bbK(x)$,称$\dfrac{P}{Q}$为$F$的
            \textbf{首一不可约形示(forme irréductible unitaire)},若$P, Q$互素且$Q$为首一多项式.
    \end{itemize}  
\end{deff}

\begin{prop} $\ $
    \begin{itemize}
        \item 设$F = \dfrac{P}{Q}$的不可约形式为$\dfrac{P_1}{Q_1}$,则存在$R \in \bbK[x]$,
            使得$P = P_1R, Q = Q_1R$.
        \item 设$F$的不可约形式为$\dfrac{P}{Q}, \dfrac{P_1}{Q_1}$,则存在$\lambda \in \bbK^*$,
            使得$P_1 = \lambda P, Q_1 = \lambda Q$.
    \end{itemize}
\end{prop}

\begin{prop}
    每个有理分式都有唯一的首一不可约形式.
\end{prop}

\begin{prop}
    设$F = \dfrac{P}{Q} \in \bbK(x)$,则$\deg P - \deg Q$的值不依赖于代表形式的选取.
\end{prop}

\begin{deff}
    设$F = \dfrac{P}{Q} \in \bbK(x)$,称$\deg P - \deg Q$为$F$的\textbf{次数(degré)}.
\end{deff}

\begin{prop}
    设$F_1, F_2 \in \bbK(x)$,
    \begin{itemize}
        \item $\deg (F_1 + F_2) \leqslant \max \{ \deg F_1, \deg F_2 \}$.
        \item $\deg F_1F_2 = \deg F_1 + \deg F_2$.
    \end{itemize}
\end{prop}

\begin{proof}
    设$F_1 = \dfrac{P_1}{Q_1}, F_2 = \dfrac{P_2}{Q_2}$.
    \begin{itemize}
        \item 我们有$$F_1 + F_2 = \dfrac{P_1}{Q_1} + \dfrac{P_2}{Q_2} = \dfrac{P_1Q_2 + P_2Q_1}{Q_1Q_2}$$
            因此$$\deg (F_1 + F_2) = \deg (P_1Q_2 + P_2Q_1) - \deg Q_1Q_2$$
            不妨设$\deg F_1 > \deg F_2$,则$\deg P_1 - \deg Q_1 > \deg P_2 - \deg Q_2$.
            因此$$\deg P_1Q_2 = \deg P_1 + \deg Q_2 > \deg P_2 + \deg Q_1 = \deg P_2Q_1$$
            由多项式的和的次数得$$\deg (P_1Q_2 + P_2Q_1) \leqslant \max \{ \deg P_1Q_2, \deg P_2Q_1 \} \leqslant \deg P_1Q_2$$
            故\begin{align*}
                \deg (F_1 + F_2) & =\deg (P_1Q_2 + P_2Q_1) - \deg Q_1Q_2 \\
                & \leqslant \deg P_1Q_2 - \deg Q_1Q_2 = \deg P_1 - \deg Q_1 \\
                & = \deg F_1 = \max \{ \deg F_1, \deg F_2 \}
            \end{align*}
        \item 同样地,我们有$F_1F_2 = \dfrac{P_1P_2}{Q_1Q_2}$,故
            \begin{align*}
                \deg F_1F_2 & = \deg P_1P_2 - \deg Q_1Q_2 \\
                & = \deg P_1 + \deg P_2 - \deg Q_1 - \deg Q_2 \\
                & = \deg F_1 + \deg F_2
            \end{align*}
    \end{itemize}
\end{proof}

\begin{deff}
    设$F$为有理分式,$\dfrac{P}{Q}$为其不可约形式.
    \begin{itemize}
        \item 称$P$的根为$F$的\textbf{根(racine)}.
        \item 称$Q$的根为$F$的\textbf{极点(pôle)}.
    \end{itemize}
\end{deff}

\begin{deff}
    设$F$为有理分式,$\dfrac{P}{Q}$为其不可约形式,$a$为其根(极点),
    称多项式$P$($Q$)的根$a$的重数为$a$的\textbf{重数(ordre de multiplicité)}.
\end{deff}

\begin{deff}
    设$F$为有理分式,$\dfrac{P}{Q}$为其不可约形式,设$\calA$为其极点的集合.
    称函数$F: \begin{array}[t]{ccc} \bbK \setminus \calA & \to & \bbK \\ \alpha & \mapsto &
    F(\alpha) \end{array}$为与$F$相伴的\textbf{有理函数(fonction rationnelle)}.
\end{deff}

\begin{rqq}
    设$F \in \bbK(x)$可以写成形式$F = \dfrac{P_1}{Q_1}$,若$Q_1(\alpha) \neq 0$,
    则$\alpha$不为$F$的极点且$F(\alpha) = \dfrac{P_1(\alpha)}{Q_1(\alpha)}$.
\end{rqq}

\begin{rqq}
    设$F, G \in \bbK(x)$,$\lambda, \mu, \alpha \in \bbK$.若$\alpha$不为$F, G$的极点,
    则$\alpha$不为$\lambda F + \mu G, FG$的极点,
    且$(\lambda F + \mu G)(\alpha) = \lambda F(\alpha) + \mu G(\alpha), (FG)(\alpha) = F(\alpha) G(\alpha)$.
\end{rqq}

\begin{deff}
    设$F$为有理分式,称$F$是\textbf{偶的(paire)}(或\textbf{奇的(impair)}),若$F(x) = F(-x)$(或$F(-x) = -F(x)$).
\end{deff}

\section{部分分式分解}

\begin{prop} \label{prop:partie_entière}
    任意有理分式$F$,都可以唯一写成一个多项式与一个次数严格负的有理分式之和.
\end{prop}

\begin{proof} $\ $
    \textbf{唯一性} \quad 设$F = E_1 + F_1 = E_2 + F_2$,其中$E_1, E_2 \in \bbK[x], F_1, F_2 \in \bbK(x), \deg F_1, \deg F_2 < 0$.
        那么$E_1 - E_2$为多项式,由于$E_1 - E_2 = F_2 - F_1$,知其次数严格负.
        因此$E_1 - E_2 = 0$,即$E_1 = E_2$,于是$F_1 = F_2$.

    \textbf{存在性} \quad 设$F = \dfrac{P}{Q}$,其中$Q \neq 0$.取$E$为$P$除以$Q$的Euclide除法商,得
        $$F = E + \dfrac{R}{Q}, \text{其中} \deg R < \deg Q$$
        这就是所要的形式.
\end{proof}

\begin{rqq}
    Proposition \ref{prop:partie_entière}中的多项式称为$F$的\textbf{整部(partie entière)}.
\end{rqq}

\begin{thm} \label{thm:décomposition_en_éléments_simples}
    设$F \in \bbK(x)$,其首一不可约形式为$\dfrac{A}{B}$,$B$分解为$\bbK[x]$中不可约因式的积$B = \prod\limits_{i = 1}^n Q_i^{r_i}$.
    其中$Q_1, Q_2, \cdots, Q_n$为首一,不可约,两两不同的多项式,$r_1, r_2, \cdots, r_n$为正整数.
    则存在唯一$E \in \bbK[x]$与多项式族$(P_{ij})_{\substack{1 \leqslant i \leqslant n \\ 1 \leqslant j \leqslant r_i}}$
    使得$$F = E + \sum\limits_{i = 1}^n \sum\limits_{j = 1}^{r_i} \dfrac{P_{ij}}{Q_i^j}$$
    其中$\deg \dfrac{P_{ij}}{Q_i} < 0$.
\end{thm}

\begin{rqq}
    Théorème \ref{thm:décomposition_en_éléments_simples}中的表示称为$F$在$\bbK$上的
    \textbf{部分分式分解(décomposition en éléments simples)}.
\end{rqq}

\begin{rqq}
    若$Q_i = x - a_i$,其中$a_i \in \bbK$,则$(P_{ij})_{1 \leqslant j \leqslant r_i}$均为常数,
    称$\sum\limits_{j = 1}^{r_i} \dfrac{P_{ij}}{(x - a_i)^j}$为$F$关于极点$a_i$的\textbf{级部(partie polaire)}.
\end{rqq}

\begin{rqq}
    多项式本身即为其部分分式分解.
\end{rqq}

\begin{prop}
    设$F$为有理分式,有单极点$a$,把$F$写成如下形式$$F = \dfrac{P}{Q} = \dfrac{P}{(x - a)Q_1}$$其中$Q_1$不以$a$为根.
    则$F$关于极点$a$的级部为$\dfrac{\lambda_1}{X - a}$其中$\lambda_1 = \dfrac{P(a)}{Q_1(a)} = \dfrac{P(a)}{Q'(a)}$.
\end{prop}

\begin{proof}
    设$F$为有理分式,有单极点$a$.记$F = \dfrac{P}{(x - a)Q_1}$,其中$Q_1$为多项式,不以$a$为根.

    我们想找标量$\lambda_1$使得$F = \dfrac{\lambda_1}{x - a} + F_0$,其中$F_0$为有理分式,不以$a$为极点.
    
    等式两边同乘$x - a$得$\dfrac{P}{Q_1} = \lambda_1 + (x - a)F_0$.
    将$a$代入$x$得$\lambda_1 = \dfrac{P(a)}{Q_1(a)}$.

    由$Q = (x - a)Q_1$,两边同时求导得$Q' = (x - a)Q_1' + Q_1$,将$a$代入$a$得$Q_1(a) = Q'(a)$,
    故$\lambda_1 = \dfrac{P(a)}{Q'(a)}$.
\end{proof}

\begin{prop}
    设$F$为有理分式,有双极点$a$,把$F$写成如下形式$$F = \dfrac{P}{Q} = \dfrac{P}{(x - a)^2 Q_2}$$其中$Q_2$不以$a$为根.
    则$F$关于极点$a$的级部为$\dfrac{\lambda_1}{x - a} + \dfrac{\lambda_2}{(x - a)^2}$
    其中$\lambda_2 = \dfrac{P(a)}{Q_2(a)}, \lambda_1 \in \bbK$.
\end{prop}

\begin{proof}
    设$F$为有理分式,有双极点$a$.记$F = \dfrac{P}{(x - a)^2Q_2}$,其中$Q_2$为多项式,不以$a$为根.

    我们想找标量$\lambda_1, \lambda_2$使得$F = \dfrac{\lambda_2}{(x - a)^2} + \dfrac{\lambda_1}{x - a} + F_0$,
    其中$F_0$为有理分式,不以$a$为极点.

    等式两边同乘$(x - a)^2$得$\dfrac{P}{Q_2} = \lambda_2 + \lambda_1 (x - a) + (x - a)^2F_0$.
    将$a$代入$x$得$\lambda_2 = \dfrac{P(a)}{Q_2(a)}$.
\end{proof}

\begin{rqq}
    这个方法容易推广.设$a$为$n$重极点,把$F$写成$F = \dfrac{P}{(x - a)^n Q_n}$其中$Q_n(a) \neq 0$.
    则$F$关于极点$a$的级部为$\sum\limits_{p = 1}^n \dfrac{\lambda_p}{(x - a)^p}$其中$\lambda_n = \dfrac{P(a)}{Q_n(a)}$.
\end{rqq}

\begin{thm}
    设$F \in \bbC(x)$,其极点为两两不同的复数$a_1, a_2, \cdots, a_n$,其重数分别为$r_1, r_2, \cdots, r_n$.
    则存在唯一$E \in \bbC[x]$与复数族$(\lambda_{kl})_{\substack{1 \leqslant k \leqslant n \\ 1 \leqslant l \leqslant r_i}}$
    使得$$F = E + \sum\limits_{k = 1}^n \sum\limits_{l = 1}^{r_k} \dfrac{\lambda_{kl}}{(x - a_k)^l}$$
\end{thm}

\begin{thm}
    设$F \in \bbR(x)$,其首一不可约形式为$\dfrac{A}{B}$,将$B$分解为$\bbR[x]$中不可约因式的积
    $$B = \prod\limits_{k = 1}^n (x - a_k)^{r_k} \prod\limits_{l = 1}^m (x^2 + p_l x + q_l)^{t_l}$$
    其中$a_1, a_2, \cdots, a_n$为$B$的两两不同的根,重数分别为$r_1, r_2, \cdots, r_n$;
    $x^2 + p_1 x + q_1, \cdots, x^2 + p_m x + q_m$为两两不同,无实根的二次多项式,$t_1, t_2, \cdots, t_m$为正整数.
    则存在唯一$E \in \bbR[x]$与实数族$(\lambda_{ku})_{\substack{1 \leqslant k \leqslant n \\ 1 \leqslant u \leqslant r_k}}$,
    实数对族$(\alpha_{lv}, \beta_{lv})_{\substack{1 \leqslant l \leqslant m \\ 1 \leqslant v \leqslant t_l}}$
    使得$$F = E + \sum\limits_{k = 1}^n \sum\limits_{u = 1}^{r_k} \dfrac{\lambda_{ku}}{(x - a_k)^u} +
    \sum\limits_{l = 1}^m \sum\limits_{v = 1}^{t_l} \dfrac{\alpha_{lv} x + \beta_{lv}}{(x^2 + p_l x + q_l)^v}$$
\end{thm}

\subsection*{部分分式分解的求法}

我们找非多项式的有理分式的分解.考虑复系数有理分式$F = \dfrac{P}{Q}$,其极点为$a_1, a_2, \cdots, a_n$,重数分别为$r_1, r_2, \cdots, r_n$.
$F$的部分分式分解形式为$$F = E + \sum\limits_{i = 1}^n \sum\limits_{j = 1}^{r_i} \dfrac{\lambda_{ij}}{(x - a_i)^j}$$

\begin{itemize}
    \item 通过$P$除以$Q$的Euclide除法得出整部$E$.
    \item 系数$\lambda_{ir_i}$通过公式$$\lambda_{ir_i} = \dfrac{P(a_i)}{Q_i(a_i)}$$计算,其中$Q = (x - a_i)^r_i Q_i$.
    \item 若每个极点都是单的,那么直接得到$F$的部分分式分解.否则,在$F$中减去每个分式$\dfrac{\lambda_{ir_i}}{(x - a_i)^{r_i}}$
        并对剩余分式重复这个过程.
\end{itemize}

但使用以下方法确定剩下的系数通常更快.

\paragraph*{实系数的情形}

若$F$为实系数有理分式,$a$为$F$的非实数$r$重极点,则$\overline{a}$也为$r$重极点,且关于$a$与$\overline{a}$的极部的系数两两共轭.

\begin{itemize}
    \item 当$a$为$Q$的非实根时,$\overline{a}$也为$Q$的根,且重数相同.
    \item 若$F = \sum\limits_{i = 1}^{r} \dfrac{\lambda_i}{(x - a)^i} + F_1$,其中$a$不为$F$的极点.
        那么对于每个复数$z$不为$F$的极点,都有
        $$F(\overline{z}) = \overline{F(z)} = \sum\limits_{i = 1}^r \dfrac{r_i}{(\overline{z} - \overline{a})^i} + \overline{F_1(z)}$$
        且有理分式$\overline{F_1}$不以$\overline{a}$为极点.
        可得$$F = \sum\limits_{i = 1}^r \dfrac{\overline{\lambda_i}}{(x - \overline{a})^i} + F_2$$
        其中$F_2$为有理分式,且不以$a$为极点.
\end{itemize}

因此,由极部的唯一性知,关于极点$\overline{a}$的极部为$\sum\limits_{i = 1}^r \dfrac{\overline{\lambda_i}}{(x - \overline{a})^i}$.

\paragraph*{奇或偶的情形}

若有理分式$F$是奇的或偶的,$a$为$F$的$n$重极点,则$-a$也为$F$的$n$重极点.
比较$F(x)$与$F(-x) = \pm F(x)$的部分分式分解,可以得出系数之间的关系.

\paragraph*{严格负次数的情形}

设$F$为次数严格负的有理分式,$f$为与$F$相伴的有理函数限制$\bbR$上除去极点外的函数,则$x^{-\deg F} f(x)$在$+\infty$处有极限.
我们也可以找出$F$的部分分式分解的项$\dfrac{1}{x - a_i}$的系数之间的关系.

\paragraph*{剩余一两个系数的情形}

当剩下一两个系数需要确定时,可以将$x$替换为一两个简单的数值.

\section{有理分式的原函数}

有理函数在$\bbR$中不包含极点的区间上连续,因此在这些区间上有原函数.
为了求其原函数,我们可以将其分解为部分分式并求出每个部分分式的原函数.

\begin{prop}
    设$a \in \bbR, n \in \bbZ \setminus \{ -1 \}$,则在$\bbR$中不包含$a$的区间上,
    函数$(x - a)^n$的一个原函数为$\dfrac{1}{n + 1} (x - a)^{n + 1}$.
\end{prop}

\begin{prop}
    设$a \in \bbR$,则在$\bbR$中不包含$a$的区间上,函数$\dfrac{1}{x - a}$的一个原函数为$\ln \lvert x - a \rvert$.
\end{prop}

\begin{prop}
    设$a = \alpha + i \beta \in \bbC$,其中$\alpha \in \bbR, \beta \in \bbR^*$,则在$\bbR$上,
    函数$\dfrac{1}{x - a} = \dfrac{1}{x - \alpha - i \beta}$的一个原函数为
    $\dfrac{1}{2} \ln \left( (x - \alpha)^2 + \beta^2 \right) + i \arctan \dfrac{x - \alpha}{\beta}$.
\end{prop}

\begin{proof}
    $a$为实数的情形是显然的.对于另一种情况,只需求导验证公式.但我们也可以使用以下方法.

    \begin{align*}
        \dfrac{1}{x - a} & = \dfrac{x - \overline{a}}{(x - a)(x - \overline{a})} \\
        & = \dfrac{x - \alpha}{(x - \alpha)^2 + \beta^2} + i \dfrac{\beta}{(x - \alpha)^2 + \beta^2} \\
        & = \dfrac{x - \alpha}{(x - \alpha)^2 + \beta^2} + i \dfrac{1}{\beta} \dfrac{1}{(\frac{x - \alpha}{\beta})^2 + 1}
    \end{align*}

    由常用积分知$\dfrac{1}{x - a}$的一个原函数为
    $\dfrac{1}{2} \ln \left( (x - \alpha)^2 + \beta^2 \right) + i \arctan \dfrac{x - \alpha}{\beta}$.
\end{proof}

\begin{prop}
    设$b, c \in \bbR$满足$b^2 - 4c < 0$,则函数$\dfrac{2x + b}{x^2 + bx + c}$的一个原函数为$\ln (x^2 + bx +c)$.
\end{prop}

\begin{proof}
    注意到$\dfrac{2x + b}{x^2 + bx + c}$具有形式$\dfrac{u'(x)}{u(x)}$,其中$u(x) = x^2 + bx + c$.
    故$$\int \dfrac{2x + b}{x^2 + bx + c} \dif x = \int \dfrac{1}{u} \dif u = \ln u(x) = \ln (x^2 + bx + c) \qedhere$$
\end{proof}

\begin{prop}
    设$b, c \in \bbR$满足$b^2 - 4c < 0$,则函数$\dfrac{1}{x^2 + bx + c}$的一个原函数为
    $\dfrac{1}{\omega} \arctan \dfrac{x + \frac{b}{2}}{\omega}$,其中$\omega = \sqrt{c - \dfrac{b^2}{4}}$.
\end{prop}

\begin{proof}
    配方得$\dfrac{1}{x^2 - bx + c} = \dfrac{1}{(x - \frac{b}{2})^2 + \omega^2}$,其中$\omega = \sqrt{c - \dfrac{b^2}{4}}$.

    对$a \in \bbR^*$,通过求导可以验证$$\int_0^x \dfrac{\dif t}{t^2 + a^2} = \dfrac{1}{a} \arctan \dfrac{x}{a}$$

    令$t = x - \dfrac{b}{2}$,得$$\int \dfrac{\dif x}{x^2 - bx + c} = \int \dfrac{\dif t}{t^2 + \omega^2} =
    \dfrac{1}{\omega} \arctan \dfrac{t}{\omega} = \dfrac{1}{\omega} \arctan \dfrac{x - \frac{b}{2}}{\omega} \qedhere$$
\end{proof}

\begin{rqq}
    为了求$\dfrac{\lambda x + \mu}{x^2 + bx + c}$在$\bbR$上的原函数,其中$b, c \in \bbR$满足$b^2 - 4c < 0$,我们可以

    \begin{itemize}
        \item 在分子中表示分母的导数$$\int \dfrac{\lambda t + \mu}{t^2 + bt + c} \dif t =
            \dfrac{\lambda}{2} \int \dfrac{2t + b}{t^2 + bt + c} \dif t -
            \left( \dfrac{b \lambda}{2} - \mu \right) \int \dfrac{1}{t^2 + bt + c} \dif t$$
            再使用前面的结论即可.
        \item 注意到原函数形如$\alpha f + \beta g$,其中$f(x) = \ln (x^2 + bx + c), g(x) = \arctan \dfrac{2x + b}{\sqrt{4c - b^2}}$.
            通过求导解出$\alpha, \beta$即可.
    \end{itemize}
\end{rqq}

\begin{rqq}
    对于某些有理分式,事先进行换元可以避免繁琐的部分分式分解.
\end{rqq}

