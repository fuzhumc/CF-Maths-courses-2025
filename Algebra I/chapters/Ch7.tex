\chapter{有理分式}

在本章中,我们记$\mathbb{K}$为$\mathbb{R}$或$\mathbb{C}$.

\section{有理分式域}

我们定义交换环$\mathbb{K}[x]$上的有理分式域$\mathbb{K}(x)$,即满足下列条件的集合.

\begin{itemize}
    \item 元素$F \in \mathbb{K}(x)$可写成形式$F = \dfrac{P}{Q}$,其中$P, Q \in \mathbb{K}[x]$且$Q \neq 0$.
    \item 设$P, Q, P_1, Q_1 \in \mathbb{K}[x]$且$Q \neq 0, Q_1 \neq 0$,
        则$$\dfrac{P}{Q} = \dfrac{P_1}{Q_1} \iff PQ_1 = P_1Q$$.
    \item $\mathbb{K}(x)$包括$\mathbb{K}[x]$,多项式$P$视为有理分式$\dfrac{P}{1}$.
    \item 具有以下运算规则:
        \begin{itemize}
            \item $\dfrac{P}{Q} +\dfrac{P_1}{Q_1} = \dfrac{PQ_1 + P_1Q}{QQ_1}$.
            \item $\dfrac{P}{Q} \cdot \dfrac{P_1}{Q_1} = \dfrac{PP_1}{QQ_1}$.
        \end{itemize}
        其中$P, Q, P_1, Q_1 \in \mathbb{K}[x]$且$Q, Q_1 \neq 0$.
\end{itemize}

这样$(\mathbb{K}(x), +, \cdot)$为域,称为\textbf{有理分式域(corps des fractions rationnelles)},
其中的元素称为\textbf{有理分式(fraction rationnelle)}.

\begin{rqq}
    设$P, Q, R \in \mathbb{K}[x], Q, R \neq 0$,则$\dfrac{PR}{QR} = \dfrac{P}{Q}$.

    设$P, Q \in \mathbb{K}[x]^*$,则$\left( \dfrac{P}{Q} \right)^{- 1} = \dfrac{Q}{P}$.
\end{rqq}

\begin{deff}
    设$F = \dfrac{P}{Q} \in \mathbb{K}(x)$,称$\dfrac{P}{Q}$为$F$的
    \textbf{不可约形式(forme irréductible)},若$P, Q$互素.

    设$F = \dfrac{P}{Q} \in \mathbb{K}(x)$,称$\dfrac{P}{Q}$为$F$的
    \textbf{首一不可约表示(forme irréductible unitaire)},若$P, Q$互素且$Q$为首一多项式.
\end{deff}

\begin{prop}
    设$F = \dfrac{P}{Q}$的不可约形式为$\dfrac{P_1}{Q_1}$,则存在$R \in \mathbb{K}[x]$,
    使得$P = P_1R, Q = Q_1R$.

    设$F$的不可约形式为$\dfrac{P}{Q}, \dfrac{P_1}{Q_1}$,则存在$\lambda \in \mathbb{K}^*$,
    使得$P_1 = \lambda P, Q_1 = \lambda Q$.
\end{prop}

\begin{prop}
    每个有理分式都有唯一的首一不可约形式.
\end{prop}

\begin{prop}
    设$F = \dfrac{P}{Q} \in \mathbb{K}(x)$,则$\deg P - \deg Q$的值不依赖于代表形式的选取.
\end{prop}

\begin{deff}
    设$F = \dfrac{P}{Q} \in \mathbb{K}(x)$,称$\deg P - \deg Q$为$F$的\textbf{次数(degré)}.
\end{deff}

\begin{prop}
    设$F_1, F_2 \in \mathbb{K}(x)$,
    \begin{itemize}
        \item $\deg (F_1 + F_2) \leqslant \max \{ \deg F_1, \deg F_2 \}$.
        \item $\deg F_1F_2 = \deg F_1 + \deg F_2$.
    \end{itemize}
\end{prop}

\begin{deff}
    设$F$为有理分式,$\dfrac{P}{Q}$为其不可约形式.
    \begin{itemize}
        \item 称$P$的根为$F$的\textbf{根(racine)}.
        \item 称$Q$的根为$F$的\textbf{极点(pôle)}.
    \end{itemize}
\end{deff}

\begin{deff}
    设$F$为有理分式,$\dfrac{P}{Q}$为其不可约形式,$a$为其根(极点),称多项式$P$($Q$)的根$a$的重数为$a$的\textbf{重数(ordre de multiplicité)}.
\end{deff}

\begin{deff}
    设$F$为有理分式,$\dfrac{P}{Q}$为其不可约形式,设$\mathcal{A}$为其极点的集合.
    称函数$F: \begin{array}[t]{ccc} \mathbb{K} \setminus \mathcal{A} & \to & \mathbb{K} \\ \alpha & \mapsto &
    F(\alpha) \end{array}$为与$F$相伴的\textbf{有理函数(fonction rationnelle)}.
\end{deff}

