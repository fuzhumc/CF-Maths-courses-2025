\chapter{基础知识}

\section{集合}

集合部分的基础定义参照高中数学(这里我们先不介绍公理化集合论).

\begin{deff}
    集合$E$与$F$的\textbf{差集}是指$\{ x \mid x \in E, x \notin F \}$,记为$E \setminus F$.
\end{deff}

\begin{deff}
    集合$E$与$F$的\textbf{对称差}是指$(E \setminus F) \cup (F \setminus E)$,记为$E \triangle F$.
\end{deff}

下面给出有关集合的一系列性质.

\begin{prop}
    设$A,B,C$为$E$的子集,则
    \begin{enumerate}
        \item $A \subseteq A \cup B$.
        \item $A \cap B \subseteq A$.
        \item $A \cup A = A$.
        \item $A \cap A = A$.
        \item $A \cup E = E$.
        \item $A \cap \varnothing = \varnothing$.
        \item $A \cap E = A$.
        \item $A \cup \varnothing = A$.
        \item $A \cup B = B \cup A$.
        \item $A \cap B = B \cap A$.
        \item $(A \cup B) \cup C = A \cup (B \cup C)$.
        \item $(A \cap B) \cap C = A \cap (B \cap C)$.
        \item $A \cap (B \cup C)=(A \cap B) \cup (A \cap C)$.
        \item $A \cup (B \cap C)=(A \cup B) \cap (A \cup C)$.
        \item $A \cup \complement_E A = E$.
        \item $A \cap \complement_E A = \varnothing$.
        \item $\complement_E E = \varnothing$.
        \item $\complement_E \varnothing = E$.
        \item $\complement_E (A \cup B) = \complement_E A \cap \complement_E B$.
        \item $\complement_E (A \cap B) = \complement_E A \cup \complement_E B$.
        \item $E \subseteq A \Rightarrow A = E$.
        \item $A \subseteq \varnothing \Rightarrow A = \varnothing$.
        \item $A \cup C \subseteq B \cup C \Leftarrow A \subseteq B$.
        \item $A \cap C \subseteq B \cap C \Leftarrow A \subseteq B$.
        \item $A \subseteq B \iff \complement_E B \subseteq \complement_E A$.
        \item $A \subseteq B \iff A \cup B = B$.
        \item $A \subseteq B \iff A \cap B = A$.
        \item $(A \triangle B) \triangle C = A \triangle (B \triangle C)$.
        \item $A \triangle \varnothing = \varnothing \triangle A =A$.
        \item $A \triangle \complement_E A = \complement_E A \triangle A = E$.
    \end{enumerate}
\end{prop}

我们在这里不给出证明,这些性质最好自己逐个证一遍,毕竟一辈子只用做一次(除非你当大学教授).

\begin{deff}
    称二元组$(x,y)$为一个\textbf{数对}.

    称$(x,y) = (x',y')$,若$x = x', y = y'$.
\end{deff}

\begin{deff}
    集合$E$与$F$的\textbf{笛卡尔积}是指$\{ (x,y) \mid x \in E, y \in F \}$,记为$E \times F$.
\end{deff}

\section{映射}

\begin{deff}
    \textbf{映射}是指一个三元组$u = (E, F, \Gamma)$,其中$E, F$为集合,$\Gamma \subseteq E \times F$满足:对于任意$x \in E$,存在唯一$y \in F$使得$(x, y) \in \Gamma$.

    此时记$u:\begin{array}[t]{ccc} E & \to & F \\ x & \mapsto & y \end{array}$,也记$u(x)=y$.

    称$E$为\textbf{出发集},$F$为\textbf{到达集},$y$为$x$的\textbf{像},$x$为$y$的\textbf{原像},$u$为$E$到$F$的映射.

    $E$到$F$的映射的全体记为$\calF(E,F)$.
\end{deff}

\begin{deff}
    称$\Gamma_u = \{ (x, y) \in E \times F \mid u(x) = y \}$为$u$的\textbf{图像}.
\end{deff}

\begin{deff}
    设$u = (E, F, \Gamma), u' = (E', F', \Gamma')$为两映射,称$u$与$u'$\textbf{相等},若$E = E', F = F', \Gamma = \Gamma'$.
\end{deff}

\begin{deff}
    称$\{ u(x) \mid x \in E \}$为$u$的\textbf{像集},记为$\im u$或$u(E)$.
\end{deff}

\begin{deff}
    设$E$为集合,称$E$到其自身的映射为$E$上的一个\textbf{置换}.

    $E$上的置换的全体记为$\calS(E)$.
\end{deff}

\begin{exer}
    设$E, F, G, H$为集合,$u \in \calF(E, F), v \in \calF(F, G), w \in \calF(G, H)$,则$w \circ (v \circ u) = (w \circ v) \circ u$.
\end{exer}

\begin{rqq}
    设$u, v \in \calF(E, E)$,则不一定有$u \circ v = v \circ u$.
\end{rqq}

\begin{exer}
    设$u: E \to F, v: F \to G$,证明:
    \begin{enumerate}
        \item 若$v \circ u$为单射,则$u$为单射.
        \item 若$v \circ u$为满射,则$v$为满射.
    \end{enumerate}
\end{exer}

\begin{prop}
    设$u: E \to F$,则$u \circ u^{-1} = \mathrm{id}_F, u^{-1} \circ u = \mathrm{id}_E$.
\end{prop}

\begin{rqq}
    一般没有$u \circ u^{-1} = u^{-1} \circ u$.
\end{rqq}

\begin{lem}
    设$u: E \to F, v: F \to E$,若$u \circ v = \mathrm{id}_F, v \circ u = \mathrm{id}_E$,则$u, v$均为双射且互为逆映射.
\end{lem}

\begin{proof}
    由$u \circ v = \mathrm{id}_F$为双射知$u$为单射且$v$为满射,

    由$v \circ u = \mathrm{id}_E$为双射知$v$为单射且$u$为满射.

    故$u,v$为双射且互为逆映射.
\end{proof}

\begin{prop}
    设$u$为双射,则$u^{-1}$也为双射,且$\left( u^{-1} \right)^{-1} = u$.
\end{prop}

\begin{prop}
    设$u: E \to F, v: F \to G$为双射,则$v \circ u$为双射,且$(v \circ u)^{-1} = u^{-1} \circ v^{-1}$.
\end{prop}

\begin{deff}
    设$u: E \to E$,称$u$为\textbf{对合},若$u \circ u = \mathrm{id}_E$.
\end{deff}

\begin{deff}
    设$u: E \to F, A \subseteq E, B \subseteq F$.

    称$\{ u(x) \mid x \in A \}$为$A$在$u$下的\textbf{像},记为$u(A)$.

    称$\{ x \in E \mid u(x) \in B \}$为$B$在$u$下的\textbf{原像},记为$u^{-1}(B)$.
\end{deff}

\begin{rqq}
    这里虽然出现了记号$u^{-1}$,但不意味着$u$可逆.
\end{rqq}

\begin{prop}
    设$u: E \to F$,则$B, B' \subseteq F$,则有
    \begin{enumerate}
        \item 若$B \subseteq B'$,则$u^{-1}(B) \subseteq u^{-1}(B')$
        \item $u^{-1}(B \cup B') = u^{-1}(B) \cup u^{-1}(B')$.
        \item $u^{-1}(B \cap B') = u^{-1}(B) \cap u^{-1}(B')$.
        \item $u^{-1}(B^{\complement})=\left( u^{-1}(B) \right)^{\complement}$.
    \end{enumerate}
\end{prop}

\begin{deff}
    设$A \subseteq E$,称$$\bbI_A(x) = \begin{cases} 1, & x \in A \\ 0, & x \notin A \end{cases}$$为$A$的\textbf{特征函数}.
\end{deff}

\begin{prop}
    设$E$为集合,则映射$$\begin{array}[t]{ccc} \calP(E) & \to & \calF(E, \{ 0, 1 \}) \\ A & \mapsto & \bbI_A \end{array}$$为双射.
\end{prop}

\begin{prop}
    设$A, B \subseteq E$,则
    \begin{enumerate}
        \item $\bbI_{A \cap B} = \bbI_A \cdot \bbI_B$.
        \item $\bbI_{A \cup B} = \bbI_A + \bbI_B - \bbI_{A \cap B}$.
        \item $\bbI_{\complement_E A} = 1 - \bbI_A$.
    \end{enumerate}
\end{prop}

\section{族}

\begin{prop}
    设$x: I \to E$,称$x(I)$为$E$中由$I$标记的一\textbf{族}元素,$I$称为\textbf{指标集}.
\end{prop}

\begin{rqq}
    在使用``族''时常把$x(i)$记为$x_i$.
\end{rqq}

\begin{rqq}
    $E$中由$\bbN$标记的一族元素常称为$E$中的一个序列.
\end{rqq}

\begin{deff}
    设$I_1, I_2 \subseteq I, A_i \subseteq E (i \in I)$,则
    \begin{enumerate}
        \item $$\left( \bigcap\limits_{i \in I_1} A_i \right) \cap \left( \bigcap\limits_{i \in I_2} A_i \right) = \bigcap\limits_{i \in I_1 \cap I_2} A_i$$
        \item $$\left( \bigcup\limits_{i \in I_1} A_i \right) \cup \left( \bigcup\limits_{i \in I_2} A_i \right) = \bigcup\limits_{i \in I_1 \cup I_2} A_i$$
        \item $$\complement_E \left( \bigcup\limits_{i \in I} A_i \right) = \bigcap\limits_{i \in I} \complement_E A_i$$
        \item $$\complement_E \left( \bigcap\limits_{i \in I} A_i \right) = \bigcup\limits_{i \in I} \complement_E A_i$$
    \end{enumerate}
\end{deff}

\begin{prop}
    设$u: E \to F$,$(B_i)_{i \in I}$为$F$的一族子集,则
    \begin{enumerate}
        \item $$u^{-1} \left( \bigcup\limits_{i \in I} B_i \right) = \bigcup\limits_{i \in I} u^{-1}(B_i)$$
        \item $$u^{-1} \left( \bigcap\limits_{i \in I} B_i \right) = \bigcap\limits_{i \in I} u^{-1}(B_i)$$
    \end{enumerate}
\end{prop}

\begin{deff}
    设$u: E \to F, A \subseteq E$,称$u \vert_A: \begin{array}[t]{ccc} A & \to & F \\ x & \mapsto & u(x) \end{array}$为$u$在$A$上的\textbf{限制}.
\end{deff}

\begin{deff}
    设$x: I \to E, J \subseteq I$,称$x \vert_J: J \to E$为$x$的一个\textbf{子族}.
\end{deff}

\section{二元关系}

\begin{deff}
    集合$E$上的一个\textbf{二元关系}是指集合$\calR \subseteq E \times E$.

    当$(x, y) \in \calR$时,记$x \calR y$.
\end{deff}

\begin{deff}
    设$R$为$E$上的一个二元关系,称$R$为\textbf{等价关系},若$R$满足:
    \begin{enumerate}
        \item 若$x \in E$,则$x \calR x$.
        \item 设$x, y \in E$,若$x \calR y$,则$y \calR x$.
        \item 设$x, y, z \in E$,若$x \calR y$且$y \calR z$,则$x \calR z$.
    \end{enumerate}
\end{deff}

\begin{deff}
    设$\calR$为$E$上的一个等价关系,对于$x \in E$,称$\{ y \mid x \calR y \}$为$x$的\textbf{等价类},记为$cl(x)$或$\bar{x}$.

    等价类的全体记为$E / \calR$,称为\textbf{商集}.

    称$x$为等价类$\bar{x}$的一个\textbf{代表元}.
\end{deff}

\begin{rqq}
    存在从$E$到$E / \calR$的满射$\pi: \begin{array}[t]{ccc} E & \to & E / \calR \\ x & \mapsto & \bar{x} \end{array}$.
\end{rqq}

\begin{rqq}
    等价类的代表元不一定唯一.
\end{rqq}

\begin{prop}
    设$\calR$为$E$上的一个等价关系,$x, y \in E$,则$\bar{x}$与$\bar{y}$要么相等,要么不相交.
\end{prop}

\begin{rqq}
    $E$上的一个等价关系$\calR$等价于$E$的分割$E / \calR$.
\end{rqq}

\begin{rqq}
    等价关系常记为$\sim$.
\end{rqq}

\section{射影空间(示例)}

待整理.

\section{序关系}

\begin{deff}
    设$\calR$为$E$上的一个二元关系,称$\calR$为\textbf{序关系},若$\calR$满足:
    \begin{enumerate}
        \item 若$x \in E$,则$x \calR x$.
        \item 设$x, y \in E$,若$x \calR y$且$y \calR x$,则$x = y$.
        \item 设$x, y, z \in E$,若$x \calR y$且$y \calR z$,则$x \calR z$.
    \end{enumerate}
    此时称$(E, \calR)$为\textbf{偏序集},序关系常记为$\leqslant$.
\end{deff}

\begin{deff}
    设$(E, \leqslant)$为偏序集,$x, y \in E$,称$x, y$\textbf{可比较},若$x \leqslant y$或$y \leqslant x$.
\end{deff}

\begin{deff}
    设$(E, \leqslant)$为偏序集,称$(E, \leqslant)$为\textbf{全序集},若对任意$x, y \in E$,都有$x, y$可比较.
\end{deff}

\begin{deff}
    设$(E, \leqslant)$为偏序集,$A \subseteq E, a \in A$.
    \begin{itemize}
        \item 称$a$为$A$的\textbf{最大元},若$\forall x \in A, x \leqslant a$.此时记$a = \max A$.
        \item 称$a$为$A$的\textbf{最小元},若$\forall x \in A, a \leqslant x$.此时记$a = \min A$.
    \end{itemize}
\end{deff}

\begin{deff}
    设$(E, \leqslant)$为偏序集,$A \subseteq E, a \in E$.
    \begin{itemize}
        \item 称$a$为$A$的\textbf{上界},若$\forall x \in A, x \leqslant a$.
        \item 称$a$为$A$的\textbf{下界},若$\forall x \in A, a \leqslant x$.
    \end{itemize}
\end{deff}

\begin{rqq}
    上下界不一定存在,存在也不一定唯一.
\end{rqq}

\begin{deff}
    设$(E, \leqslant)$为偏序集,$A \subseteq E, a \in A$.
    \begin{itemize}
        \item 称$a$为$A$的\textbf{极大元},若$\forall x \in A, (a \leqslant x \Rightarrow x = a)$.
        \item 称$a$为$A$的\textbf{极小元},若$\forall x \in A, (x \leqslant a \Rightarrow x = a)$.
    \end{itemize}
\end{deff}

\begin{rqq}
    极大小元不一定存在,存在也不一定唯一.

    若最大(小)元存在,则最大(小)元必为极大(小)元.
\end{rqq}

\section{自然数的集合}

下面这段话来自法文书
\begin{itemize}
    \item 自然数,及其集合$\bbN$,是所有数学的基础,人类最早的数学活动无疑是计数.因此$\bbN$的大多数性质看起来如此自然,在相较近的时期之前,即使是在最伟大的数学家之间,也没有引起任何疑问.
    \item 直到19世纪,人们才处理$\bbN$的构造问题.对全体整数的严格构造超出了本课程的范围,我们将遵循数学家Kronecker所说:``上帝赐予我们整数,人类完成其余.''然而,在本书中列出详尽的公认性质必不可少的,这使得我们能构建后续内容.
\end{itemize}

自然数集$\bbN$是一个非空集合,配上一个全序关系,以及与该序关系相容的一个加法运算和一个乘法运算,即:$$\forall x, y ,z \in \bbN, (x \leqslant y \Rightarrow x + z \leqslant y + z \land x \leqslant y \Rightarrow xz \leqslant yz)$$

此外:
\begin{itemize}
    \item $\bbN$的非空子集都有最小元.
    \item $\bbN$的非空有上界子集都有最大元.
    \item $\bbN$没有上界.
\end{itemize}

直接结果:
\begin{itemize}
    \item $\bbN$有最小元,记为0,记$\bbN^* = \bbN \setminus \{ 0 \}$.
    \item 设$n \in \bbN$,则严格大于$n$的自然数构成的集合非空(否则$n$为$\bbN$的最大元);称其最小元为$n$的\textbf{后继};
        \begin{itemize}
            \item 记1为0的后继.
            \item $n$的后继也为$n+1$,因此$$\forall m \in \bbN, (m > n \Rightarrow m \geqslant n + 1)$$
        \end{itemize}
    \item 同理,若$n \in \bbN^*$,则严格小于$n$的自然数构成的集合非空(它包含0)且以$n$为上界;称其最大元为$n$的\textbf{前驱},等于$n - 1$,且$$\forall m \in \bbN, (m < n \Rightarrow m \leqslant n - 1)$$
\end{itemize}

作为$\bbN$的定义的应用,我们给出下面的定理.
\begin{thm}[Eucilde除法]
    设$a \in \bbN, b \in \bbN^*$,则存在唯一$q, r \in \bbN$使得$$a = bq + r, r < b$$
    \begin{itemize}
        \item $q$称为$a$对$b$的Euclide除法商.
        \item $r$称为$a$对$b$的Euclide除法余数.
    \end{itemize}
\end{thm}

\section{归纳推理}

我们上面列出的$\bbN$的性质的一个重要应用是\textbf{归纳原理},它断言:
\begin{thm}[归纳原理]
    若$\bbN$的子集$A$包含0,且满足$$\forall n \in \bbN, (n \in A \Rightarrow n + 1 \in A)$$则$A = \bbN$
\end{thm}
我们通常通过下面的定理来运用这一原理,从而进行\textbf{归纳推理}.
\begin{thm}
    设$P$为$\bbN$上的一个叙述,若$P(0)$为真且$\forall n \in \bbN, (P(n) \Rightarrow P(n + 1))$则对于每个自然数$n$,$P(n)$都为真.
\end{thm}
\begin{rqq}
    我们常将$P(n)$记为$P_n$.
\end{rqq}

\section{有限集}

\begin{deff}
    称$E$为\textbf{有限集},若存在$n \in \bbN$及从$\{ 1, 2, \cdots, n \}$到$E$的双射.

    整数$n$称为$E$的\textbf{势},或者$E$的元素个数.通常记为$\mathrm{card} E$,也可以是$\lvert E \rvert$或$\# E$
\end{deff}

\begin{rqq}
    势为1的集合称为\textbf{单点集}.
\end{rqq}

\begin{rqq}
    不是有限集的集合称为\textbf{无限集}.
\end{rqq}

\begin{prop}
    设$E$为有限集,势为$n$,若集合$F$与$E$直接存在双射,则$\lvert F \rvert = n$.
\end{prop}

\begin{prop}
    设$E$为有限集,$F$为$E$的子集,则
    \begin{enumerate}
        \item $F$为有限集且$\lvert F \rvert \leqslant \lvert E \rvert$.
        \item $\lvert F \rvert = \lvert E \rvert$当且仅当$F = E$.
    \end{enumerate}
\end{prop}

\begin{prop}
    设$E, F$为非空有限集,$u: E \to F$,则
    \begin{enumerate}
        \item $u(E)$为有限集且$\lvert u(E) \rvert \leqslant \lvert E \rvert$.
        \item $\lvert u(E) \rvert = \lvert E \rvert$当且仅当$u$为单射.
    \end{enumerate}
\end{prop}

\begin{rqq}
    从这个形式可以推出以下结果.
    \begin{enumerate}
        \item 若存在从有限集$E$到有限集$F$的满射,则$\lvert F \rvert \leqslant \lvert E \rvert$且$F = u(E)$.
        因此,若$\lvert F \rvert > \lvert E \rvert$,则不存在从有限集$E$到有限集$F$的满射.
        \item 若存在从有限集$E$到有限集$F$的单射,则$\lvert E \rvert \leqslant \lvert F \rvert$.
        因此,若$\lvert E \rvert > \lvert F \rvert$,则不存在从有限集$E$到有限集$F$的单射.
    \end{enumerate}
\end{rqq}

\begin{thm}
    设$E, F$为非空有限集,且具有相同的势,$u$为$E$到$F$的映射,则下列三个等价:
    \begin{enumerate}
        \item $u$为单射.
        \item $u$为满射.
        \item $u$为双射.
    \end{enumerate}
\end{thm}

\begin{rqq}
    对无限集不成立.
\end{rqq}

\section{选择公理和超限归纳法}

\begin{thm}[选择公理]
    设$E$为非空集合,则存在映射$f: \calP(E) \setminus \{ \varnothing \} \to E$使得$f(A) \in A$.这个映射称为\textbf{选择函数}.
\end{thm}

\begin{rqq}
    选择公理意味着,可以从非空集合的每个子集中同时选出一个元素.
\end{rqq}

\begin{deff}
    设$(A, \leqslant)$为偏序集,$B \subseteq A$.称$B$为一个\textbf{链},若$(B, \leqslant)$为全序集.
\end{deff}

\begin{thm}[Zorn引理]
    设$A$非空,$(A, \leqslant)$为偏序集.若$A$中所有链都有上界,则$A$有极大元.
\end{thm}

\begin{deff}
    设$(A, \leqslant)$为偏序集,称它是\textbf{良序集},若$A$的任何非空子集都有最小元.
\end{deff}

\begin{prop}
    良序集一定是全序集.
\end{prop}

\begin{thm}[良序原理]
    设$A$非空,则存在$A$上的一个序关系$\leqslant$,使得$(A, \leqslant)$为良序集.
\end{thm}

\begin{thm}[超限归纳法]
    设$(A, \leqslant)$为良序集,$A$的非空子集$B$满足$$\forall a \in A ( \{ c \in A \mid c < a \} \subseteq B \Rightarrow a \in B )$$则$A = B$.
\end{thm}

\section{线性方程组}

待整理.