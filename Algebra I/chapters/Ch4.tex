\chapter{环论}

\section{环}

\begin{deff}
    设$A$为非空集合,$+, *$为$A$上的两个内部合成法则.称$(A, +, *)$为一个\textbf{环},若
    \begin{enumerate}
        \item $(A, +)$为交换群.
        \item $(A, *)$为幺半群.
        \item $*$对$+$分配.
    \end{enumerate}
\end{deff}

\begin{deff}
    设$(A, + ,*)$为环,称$(A, +, *)$为一个\textbf{交换环},若$*$交换.
\end{deff}

\begin{rqq}
    设$A={0}, 0 + 0 = 0, 0 * 0 = 0$,则$(A, +, *)$为环,称为平凡环.
\end{rqq}

\begin{rqq}
    设$(A, +, *)$为环,则交换群$(A, +)$的中性元一般记为$0_A$,幺半群$(A, *)$中性元一般记为$1_A$.
\end{rqq}

\begin{prop}
    设$(A, +, *)$为环,则有$\lvert A \rvert = 1 \iff 0_A = 1_A$.
\end{prop}

\begin{deff}
    设$(A, +, *)$为环,$B \subseteq A$.称$B$为$A$的\textbf{子环},若
    \begin{enumerate}
        \item $1 \in B$.
        \item $(B, +)$为群.
        \item $\forall x, y \in B, x * y \in B$.
    \end{enumerate}.
    
    记为$B \leqslant A$.
\end{deff}

\begin{prop}
    设$(A, +, *)$为环,则
    \begin{enumerate}
        \item $\forall x \in A, 0 * x = x * 0 = 0$.
        \item $\forall x, y \in A, (- x) * y = - (x * y) = x * (- y)$.
    \end{enumerate}
\end{prop}

\begin{prop}
    设$(A, +, *)$为环,$x, y \in A$.若$x * y = y * x$,则$(x + y)^n = \sum\limits_{k=0}^n \binom{n}{k} x^k y^{n-k}$.
\end{prop}

\begin{deff}
    设$(A, +, *)$为环,$a \in A^*$.称$a$为$A$的\textbf{零因子},若存在$b \in A$,使得$a * b = b * a = 0$.
\end{deff}

\begin{deff}
    设$(A, +, *)$为环.称$A$为一个\textbf{整环},若$A$为交换环且无零因子.
\end{deff}

\begin{deff}
    设$(A, +, *)$为环,$a \in A$.称$a$为$A$的一个\textbf{单位},若$a$可逆.

    $A$中单位的全体记为$A^{\times}$.
\end{deff}

\begin{prop}
    设$(A, +, *)$为环,则$(A^{\times}, *)$为群.
\end{prop}

\begin{prop}
    零因子一定不是单位.
\end{prop}

\section{商环与理想}

\begin{deff}
    设$(A, +, *)$为交换环,$I \subseteq A$.称$I$为$A$的一个\textbf{理想},若
    \begin{enumerate}
        \item $(I, +)$为$(A, +)$的子群.
        \item $\forall a \in A, i \in I, a * i \in I$.
    \end{enumerate}
\end{deff}

\begin{prop}
    设$(A, +, *)$为交换环,$I$为$A$的理想,则$(A/I, +)$为$(A, +)$的商群.
\end{prop}

\begin{prop}
    设$(A, +, *)$为交换环,$I$为$A$的理想,则$(A/I, +, *)$为环.
\end{prop}

\begin{deff}
    设$(A, +, *)$为交换环,$I$为$A$的理想,称$(A/I, +, *)$为$A$的一个\textbf{商环}.
\end{deff}

\begin{rqq}
    若$A$不交换,则这样定义的$I$称为\textbf{左理想},类似地可以定义\textbf{右理想}.
\end{rqq}

\begin{prop}
    只有$A$本身既为理想,又为子环.
\end{prop}

\section{环同态}

\begin{deff}
    设$(A, +_A, *_A), (B, +_B, *_B)$为环,$\varphi \in \mathcal{F}(A, B)$.称$\varphi$为\textbf{环同态},若$\varphi(1_A) = 1_B$且$\forall x, y \in A, \varphi(x +_A y) = \varphi(x) +_B \varphi(y), \varphi(x *_A y) = \varphi(x) *_B \varphi(y)$.
\end{deff}

\begin{deff}
    设$\varphi: A \to B$为环同态,称$\varphi^{-1}(1_B) = \{ a \in A \mid \varphi(a) = 1_B \}$为$\varphi$的\textbf{核},记为$\mathrm{Ker} \varphi$.
\end{deff}

\begin{deff}
    设$\varphi: A \to B$为环同态,
    \begin{enumerate}
        \item 称$\varphi$为\textbf{单同态},若$\varphi$为单射.
        \item 称$\varphi$为\textbf{满同态},若$\varphi$为满射.
        \item 称$\varphi$为\textbf{环同构},若$\varphi$为双射.
    \end{enumerate}
\end{deff}