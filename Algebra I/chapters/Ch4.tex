\chapter{环论}

\section{环}

\begin{deff}
    设$A$为非空集合,$+, *$为$A$上的两个内部合成法则.称$(A, +, *)$为一个\textbf{环},若
    \begin{enumerate}
        \item $(A, +)$为交换群.
        \item $(A, *)$为幺半群.
        \item $*$对$+$分配.
    \end{enumerate}
\end{deff}

\begin{deff}
    设$(A, + ,*)$为环,称$(A, +, *)$为一个\textbf{交换环},若$*$交换.
\end{deff}

\begin{rqq}
    设$A={0}, 0 + 0 = 0, 0 * 0 = 0$,则$(A, +, *)$为环,称为平凡环.
\end{rqq}

\begin{rqq}
    设$(A, +, *)$为环,则交换群$(A, +)$的中性元一般记为$0_A$,幺半群$(A, *)$中性元一般记为$1_A$.
\end{rqq}

\begin{prop}
    设$(A, +, *)$为环,则有$\lvert A \rvert = 1 \iff 0_A = 1_A$.
\end{prop}

\begin{deff}
    设$(A, +, *)$为环,$B \subseteq A$.称$B$为$A$的\textbf{子环},若
    \begin{enumerate}
        \item $1 \in B$.
        \item $(B, +)$为群.
        \item $\forall x, y \in B, x * y \in B$.
    \end{enumerate}
    
    记为$B \leqslant A$.
\end{deff}

\begin{prop}
    设$(A, +, *)$为环,则
    \begin{enumerate}
        \item $\forall x \in A, 0 * x = x * 0 = 0$.
        \item $\forall x, y \in A, (- x) * y = - (x * y) = x * (- y)$.
    \end{enumerate}
\end{prop}

\begin{prop}
    设$(A, +, *)$为环,$x, y \in A$.若$x * y = y * x$,则$(x + y)^n = \sum\limits_{k=0}^n \binom{n}{k} x^k y^{n-k}$.
\end{prop}

\begin{deff}
    设$(A, +, *)$为环,$a \in A^*$.称$a$为$A$的\textbf{零因子},若存在$b \in A$,使得$a * b = b * a = 0$.
\end{deff}

\begin{deff}
    设$(A, +, *)$为环.称$A$为一个\textbf{整环},若$A$为交换环且无零因子.
\end{deff}

\begin{deff}
    设$(A, +, *)$为环,$a \in A$.称$a$为$A$的一个\textbf{单位},若$a$可逆.

    $A$中单位的全体记为$A^{\times}$.
\end{deff}

\begin{prop}
    设$(A, +, *)$为环,则$(A^{\times}, *)$为群.
\end{prop}

\begin{prop}
    零因子一定不是单位.
\end{prop}

\section{环同态}

\begin{deff}
    设$(A, +_A, *_A), (B, +_B, *_B)$为环,$\varphi \in \mathcal{F}(A, B)$.称$\varphi$为\textbf{环同态},若$\varphi(1_A) = 1_B$且$\forall x, y \in A, \varphi(x +_A y) = \varphi(x) +_B \varphi(y), \varphi(x *_A y) = \varphi(x) *_B \varphi(y)$.
\end{deff}

\begin{deff}
    设$\varphi: A \to B$为环同态,称$\varphi^{-1}(1_B) = \{ a \in A \mid \varphi(a) = 1_B \}$为$\varphi$的\textbf{核},记为$\mathrm{Ker} \varphi$.
\end{deff}

\begin{deff}
    设$\varphi: A \to B$为环同态,
    \begin{enumerate}
        \item 称$\varphi$为\textbf{单同态},若$\varphi$为单射.
        \item 称$\varphi$为\textbf{满同态},若$\varphi$为满射.
        \item 称$\varphi$为\textbf{环同构},若$\varphi$为双射.
    \end{enumerate}
\end{deff}

\begin{prop}
    设$\varphi: A \to B$为环同态,则$\varphi$为环同构$\iff$存在环同态$\psi: B \to A, \psi \circ \varphi = \mathrm{id}_A, \varphi \circ \psi = \mathrm{id}_B$.
\end{prop}

\begin{prop}
    设$\varphi: A \to B$为环同态,则$\varphi$为单同态$\iff \mathrm{Ker} \varphi = \{ e \} \iff \lvert \mathrm{Ker} \varphi \rvert = 1$.
\end{prop}

\section{理想}

\begin{deff}
    设$(A, +, *)$为交换环,$I \subseteq A$.称$I$为$A$的一个\textbf{理想},若
    \begin{enumerate}
        \item $(I, +)$为$(A, +)$的子群.
        \item $\forall a \in A, i \in I, a * i \in I$.
    \end{enumerate}
\end{deff}

\begin{prop}
    设$(A, +, *)$为交换环,$I$为$A$的理想,则$(A/I, +)$为$(A, +)$的商群.
\end{prop}

\begin{prop}
    设$(A, +, *)$为交换环,$I$为$A$的理想,则$(A/I, +, *)$为环.
\end{prop}

\begin{deff}
    设$(A, +, *)$为交换环,$I$为$A$的理想,称$(A/I, +, *)$为$A$的一个\textbf{商环}.
\end{deff}

\begin{rqq}
    若$A$不交换,则这样定义的$I$称为\textbf{左理想},类似地可以定义\textbf{右理想}.
\end{rqq}

\begin{prop}
    只有$A$本身既为理想,又为子环.
\end{prop}

\begin{thm}
    设$\varphi: A \to B$为环同态,则存在唯一环同构$\bar{\varphi}: A / \mathrm{Ker} \varphi \to \mathrm{Im} \varphi$使得$\varphi = i \circ \bar{\varphi} \circ \pi$.
    其中$i: \mathrm{Im} \varphi \to B$为自然嵌入(单同态),$\pi: A \to B / \mathrm{Ker} \varphi$为自然投影(满同态).
\end{thm}

\begin{prop}
    设$\varphi: A \to B$为满同态,则$A / \mathrm{Ker} \varphi \cong B$.
\end{prop}

\begin{deff}
    设$(A, +, *)$为交换环,$I, J$为$A$的理想.称$\left\{ \sum\limits_{\lambda \in \Lambda} i_{\lambda} * j_{\lambda} \mid i_{\lambda} \in I, j_{\lambda} \in J, \Lambda \text{为有限集} \right\}$为$I, J$\textbf{生成}的理想,记为$IJ$.
\end{deff}

\begin{prop}
    $IJ$为包含$\{ i * j \mid i \in I, j \in J \}$的最小理想.
\end{prop}

\begin{deff}
    设$(A, +, *)$为交换环,$P$为$A$的理想.称$P$为$A$的一个\textbf{素理想},若$P \neq A$且$\forall x, y \in P, x * y \in P \Rightarrow x \in P \lor y \in P$
\end{deff}

\begin{prop}
    $\{ 0 \}$为素理想当且仅当$A$为整环.
\end{prop}

\begin{thm}[Dedekind]
    整环中非零非平凡的理想均可唯一分解成素理想的积.
\end{thm}

\begin{deff}
    设$A$为交换环,$M$为$A$的理想.称$M$为$A$的\textbf{极大理想},若$M$为偏序集$(\{ I \mid I \text{为} A \text{的真理想} \}, \subseteq)$的极大元.
\end{deff}

\begin{thm}
    承认选择公理时,极大理想总是存在.
\end{thm}

\begin{rqq}
    极大理想不一定唯一.
\end{rqq}

\begin{prop}
    设$A$为交换环,$I$为其理想.
    \begin{enumerate}
        \item $I$为极大理想$\iff A / I$为域.
        \item $I$为素理想$\iff A / I$为整环.
    \end{enumerate}
\end{prop}

\section{模与线性空间}

\begin{deff}
    设$(A, +, *)$为环,$E$为非空集合.称映射$*: \begin{array}[t]{ccc} A \times E & \to & E \\ (\lambda, x) & \mapsto & \lambda * x \end{array}$为$A$在$E$上的一个\textbf{外部合成法则}或\textbf{外部运算}.
\end{deff}

\begin{deff}
    设$(A, +, *)$为环,$E$为非空集合.称$E$为一个(左)$A$-\textbf{模},若
    \begin{enumerate}
        \item $(E, +)$为交换群.
        \item 存在外部运算$*: A \times E \to E$,满足:
        \begin{enumerate}
            \item $\forall x \in E, 1 * x = x$.
            \item $\forall x \in E, \lambda, \mu \in A, (\lambda + \mu) * x = \lambda * x + \mu * x$.
            \item $\forall x, y \in E, \lambda \in A, \lambda * (x + y) = \lambda * x + \lambda * y$.
            \item $\forall x \in E, \lambda, \mu \in A, (\lambda * \mu) * x = \lambda * (\mu * x)$.
        \end{enumerate}
    \end{enumerate}
\end{deff}

\begin{rqq}
    类似地地可定义(右)$A$-\textbf{模}.
\end{rqq}

\begin{deff}
    设$K$为域,则称$K$-模为$K$-\textbf{向量空间}($K$-e.v.).
\end{deff}