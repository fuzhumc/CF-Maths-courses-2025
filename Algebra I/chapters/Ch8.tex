\chapter{向量空间}

在本章中,$\bbK$表示$\bbR$或$\bbC$.

\section{向量空间}

\begin{deff}
    设$E$为非空集合,称映射$$\cdot: \begin{array}[t]{ccc} \bbK \times E & \to & E \\ (\alpha, x) & \mapsto &
    \alpha \cdot x \end{array}$$为$E$上的一个\textbf{外部法则(loi externe)}.
\end{deff}

\begin{deff}
    设$E$为非空集合,具有一个内部合成法则$+$与一个外部法则$\cdot$.
    称$(E, +, \cdot)$为一个\textbf{$\boldsymbol{\bbK}$-向量空间($\boldsymbol{\bbK}$-espace vectoriel)},若
    \begin{itemize}
        \item $(E, +)$为交换群.
        \item 对任意$x, y \in E, \alpha, \beta \in \bbK$,
            \begin{enumerate}
                \item $1 \cdot x = x$.
                \item $\alpha \cdot (\beta \cdot x) = (\alpha \beta) \cdot x$.
                \item $(\alpha + \beta) \cdot x = \alpha \cdot x + \beta \cdot x$.
                \item $\alpha \cdot (x + y) = \alpha \cdot x + \alpha \cdot y$.
            \end{enumerate}
    \end{itemize}
    此时称$E$的元素为\textbf{向量(vecteur)},称$\bbK$的元素为\textbf{标量(scalaire)}.
\end{deff}

\begin{rqq}
    对$\alpha \in \bbK, x \in E$,记$\alpha x = \alpha \cdot x$;
    对$\alpha \in \bbK^*, x \in E$,记$\dfrac{x}{\alpha} = \dfrac{1}{\alpha} \cdot x$.
\end{rqq}

\begin{rqq}
    当内部法则与外部法则明确时,我们也可以说``$E$为一个$\bbK$-向量空间'',而不是``$(E, +, \cdot)$为一个$\bbK$-向量空间''.
\end{rqq}

\begin{prop}
    设$\alpha \in \bbK, x \in E$,
    \begin{enumerate}
        \item $0_{\bbK} x = 0_E$.
        \item $\alpha 0_E = 0_E$.
        \item $\alpha (- x) = - (\alpha x) = (-\alpha) x$.
    \end{enumerate}
\end{prop}

\begin{prop}
    设$\alpha \in \bbK, x \in E, \alpha x = 0_E$,则$\alpha = 0_{\bbK}$或$x = 0_E$.
\end{prop}

\begin{deff}
    设$E, F$为$\bbK$-向量空间,则集合$E \times F$配备运算
    \begin{itemize}
        \item $(x', y') + (x'', y'') = (x' + x'', y' + y'')$.
        \item $\lambda (x, y) = (\lambda x, \lambda y)$.
    \end{itemize}
    成为一个$\bbK$-向量空间,称为\textbf{向量积空间(espace vectoriel produit)$\boldsymbol{E \times F}$}.
\end{deff}

\begin{prop}
    设$E$为$\bbK$-向量空间,$X$为集合.则集合$\calF(X, E)$配备运算
    \begin{itemize}
        \item $f + g: \begin{array}[t]{ccc} X & \to & E \\ x & \mapsto & f(x) + g(x) \end{array}$
        \item $\alpha f: \begin{array}[t]{ccc} X & \to & E \\ x & \mapsto & \alpha f(x) \end{array}$
    \end{itemize}
    成为一个$\bbK$-向量空间.
\end{prop}

\section{向量子空间}

\begin{deff}
    设$E$为$\bbK$-向量空间,$x, y, z \in E$,称$z$为$x$与$y$的\textbf{线性组合(combinaison linéaire)},
    若存在$\lambda, \mu \in \bbK$,使得$z = \lambda x + \mu y$.
\end{deff}

\begin{deff}
    设$E$为$\bbK$-向量空间,$F$为$E$的非空子集.称$F$为$E$的\textbf{向量子空间(sous-espace vectoriel)},
    若$0_E \in F$且$$\forall x, y \in F, \lambda, \mu \in \bbK, \lambda x + \mu y \in F$$
\end{deff}

\begin{rqq}
    $E$与$\{ 0_E \}$均为$E$的向量子空间,称为$E$的\textbf{平凡子空间(sous-espaces vectoriels triviaux)}.
\end{rqq}

\begin{prop}
    $E$的每个向量子空间均为$\bbK$-向量空间.
\end{prop}

\begin{prop}
    设$E$为$\bbK$-向量空间,$(F_i)_{i \in I}$为$E$的一族向量子空间,则$\bigcup\limits_{i \in I} F_i$也为$E$的向量子空间.
\end{prop}

\begin{prop}
    设$\calA$为$E$的子集,则存在包含$\calA$的最小向量子空间,
    称为\textbf{$\boldsymbol{\calA}$生成的向量子空间(sous-espace vectoriel engendré par $\boldsymbol{\calA}$)},
    记为$\vect \calA$.
\end{prop}

\begin{deff}
    设$A = (a_i)_{i \in I}$为$E$中一族元素,称$\calA = \{ a_i \mid i \in I \}$生成的向量子空间为
    \textbf{$\boldsymbol{A}$生成的向量子空间(sous-espace vectoriel engendré par $\boldsymbol{A}$)},
    记为$\vect A, \vect \calA, \vect (a_i)_{i \in I}$,
    或$\vect \{ a_i \mid i \in I \}$.
\end{deff}

\begin{prop}\label{prop:sous_espace_engendré}
    设$E$为$\bbK$-向量空间,$n \in \bbN^*, A = (a_1, a_2, \cdots, a_n)$为$E$中一族元素,
    则$\vect A = \left\{ \sum\limits_{i=1}^n \lambda_i a_i \mid \lambda_i \in \bbK \right\}$.
\end{prop}

\begin{deff}
    设$E$为$\bbK$-向量空间,$n \in \bbN^*, x_1, x_2, \cdots, x_n \in E$.
    称$\sum\limits_{i = 1}^n \lambda_i x_i, \lambda_i \in \bbK$为$x_1, x_2, \cdots, x_n$或
    族$(x_1, x_2, \cdots, x_n)$的\textbf{线性组合(combinaison linéaire)}.
\end{deff}

\begin{rqq}
    Proposition \ref{prop:sous_espace_engendré}也可表述为:
    向量$x_1, x_2, \cdots, x_n$生成的向量子空间为它们的线性组合的全体.
\end{rqq}

\section{线性映射}

在本节中,$E, F, G$为$\bbK$-向量空间.

\begin{deff}
    设$u \in \calF(E, F)$,称$u$为线性映射,
    若$$\forall x, y \in E, \alpha, \beta \in \bbK, u(\alpha x + \beta y) = \alpha u(x) + \beta u(y)$$

    $u$也称为\textbf{向量空间的同态(morphisme d'espaces vectoriels)}.

    记$\calL(E, F)$为$E$到$F$的线性映射的全体.
\end{deff}

\begin{rqq}
    称$u$为
    \begin{itemize}
        \item \textbf{自同态(endomorphisme)},若$E = F$.
        \item \textbf{同构(isomorphisme)},若$u$为双射.
        \item \textbf{自同构(automorphisme)},若$E = F$且$u$为双射.
    \end{itemize}
\end{rqq}

\begin{prop}
    设$u \in \calL(E, F)$,则$u(0_E) = 0_F$.
\end{prop}

\begin{rqq}
    我们一般用相同的方式表示向量$0_E$与$0_F$,则上述结论即为$u(0) = 0$.
\end{rqq}

\begin{rqq}
    设$u \in \calL(E, F), n \in \bbN^*, x_1, x_2, \cdots, x_n \in E, \lambda_1, \lambda_2, \cdots,
    \lambda_n \in \bbK$,则$$u \left( \sum\limits_{i = 1}^n \lambda_i x_i \right) = \sum\limits_{i = 1}^n
    \lambda_i u(x_i)$$

    上述结论翻译为:\textit{线性映射下,线性组合的像为像的线性组合}.
\end{rqq}

\begin{prop}
    设$u \in \calL(E, F)$.
    \begin{itemize}
        \item 设$E'$为$E$的向量子空间,则$u(E')$为$F$的向量子空间.
        \item 设$F'$为$F$的向量子空间,则$u^{-1}(F')$为$E$的向量子空间.
    \end{itemize}
\end{prop}

\begin{deff}
    设$u \in \calL(E, F)$.
    \begin{itemize}
        \item 记$\ker u = \{ x \in E \mid u(x) = 0_F \}$,为$E$的向量子空间.
        \item 记$\im u = u(E) = \{ u(x) \mid x \in E \}$,为$F$的向量子空间.
    \end{itemize}
\end{deff}

\begin{thm}
    设$u \in \calL(E, F)$,则$u$为单射当且仅当$\ker u = \{ 0_E \}$.
\end{thm}

\begin{prop}
    设$u, v \in \calL(E, F), \lambda, \mu \in \bbK$,则$\lambda u + \mu v$为线性映射.
\end{prop}

\begin{prop}
    $\calL(E, F)$为$\calF(E, F)$的向量子空间.
\end{prop}

\begin{prop}
    设$u \in \calL(E, F), v \in \calL(F, G)$,则$v \circ u \in \calL(E, G)$.
\end{prop}

\begin{prop}
    映射$\begin{array}[t]{ccc} \calL(F, G) \times \calL(E, F) & \to & \calL(E, G) \\
    (v, u) & \mapsto & v \circ u \end{array}$是双线性的.

    也就是说,映射$R_u: \begin{array}[t]{ccc} \calL(F, G) & \to & \calL(E, G) \\
    \psi & \mapsto & \psi \circ u \end{array}$与$L_v: \begin{array}[t]{ccc} \calL(E, F) &
    \to & \calL(E, G) \\ \varphi & \mapsto & v \circ \varphi \end{array}$均是线性的.
\end{prop}

\begin{prop}
    $(\calL(E, E), +, \circ)$为环.
\end{prop}

\begin{prop}
    设$u \in \calL(E, F)$为同构,则逆映射$u^{-1}$是线性的.
\end{prop}

记$\mathcal{GL}(E)$为$E$上的自同构的全体.

\begin{prop}
    $(\mathcal{GL}(E), \circ)$为群,称为$E$的线性群.
\end{prop}

\section{向量空间的仿射子空间}

\begin{deff}
    设$a \in E$,称从$E$到$E$的映射$t_a: x \mapsto x + a$为沿$a$的\textbf{平移(translation)},记为$t_a$.
\end{deff}

\begin{rqq}
    $t_a \in \calL(E, E) \iff a = 0$.
\end{rqq}

\begin{prop}
    \begin{itemize}
        \item 对任意$a, b \in E$,有$t_a \circ t_b = t_{a + b} = t_b \circ t_a$.
        \item 对任意$a \in E$,平移$t_a$为双射且$t_a^{-1} = t_{-a}$.
    \end{itemize}
\end{prop}

对$a \in E$及$E$的向量子空间$F$,我们记$a + F = \{ a + x \mid x \in F \} = t_a(F)$.

\begin{rqq}
    \begin{enumerate}
        \item 集合$\calF = a + F$非空,因为$a \in a + F$.
        \item 对$x \in E$,$x \in a + F$当且仅当$x - a \in F$.
    \end{enumerate}
\end{rqq}

\begin{lem}
    设$a, a' \in E$, $F, F'$为$E$的向量子空间,则$a + F = a' + F'$当且仅当$F = F'$且$a' - a \in F$.
\end{lem}

\begin{deff}
    \begin{itemize}
        \item 设$\calF \subseteq E$,称$\calF$为$E$的\textbf{仿射子空间(sous-espace affine)},
            若存在$a \in E$及$E$的向量子空间$F$,使得$\calF = a + F$.
        \item 设$\calF \subseteq E$,称$\calF$为\textbf{过点$\boldsymbol{a}$,方向为$\boldsymbol{F}$的仿射子空间
            (sous-espace affine passant par $\boldsymbol{a}$ et dirigé par $\boldsymbol{F}$)},若有$x \in \calF \iff x - a \in F$.
        \item 向量子空间$F$称为$\calF$的\textbf{方向(direction)}.
    \end{itemize}
\end{deff}

\begin{rqq}
    仿射子空间非空.
\end{rqq}

\begin{rqq}
    单点的仿射子空间方向为0.
\end{rqq}

\begin{prop}
    设$F$为$E$的向量子空间,$\calF$为$E$的仿射子空间,过点$a$,方向为$F$.则对任意$a' \in \calF$,都有$\calF = a' + F$.
\end{prop}

\begin{prop}
    设$\calF, \calG$为$E$的仿射子空间,方向分别为$F, G$,则
    \begin{itemize}
        \item $\calF \cap \calG$非空.
        \item $\calF \cap \calG$为$E$的仿射子空间,其方向为$F \cap G$.
    \end{itemize}
\end{prop}

\section{任意族生成的子空间}

\begin{deff}
    设$X$为$E$中任意子集或族,称一个向量为$X$中元素的线性组合,若其为$X$中有限元素的线性组合.
\end{deff}

\begin{rqq}
    设$x \in E$,则
    \begin{itemize}
        \item 若$A$为$E$中一族元素,则$x$为$A$中元素的线性组合,当且仅当存在$A$的有限子族$A'$使得$x \in \vect(A')$.
        \item 若$\calA$为$E$中子集,则$x$为$\calA$中元素的线性组合,当且仅当存在$\calA$的有限子集$\calA'$使得$x \in \vect \calA'$.
    \end{itemize}
\end{rqq}

\begin{prop}
    设$\calA$为$E$的非空子集,则向量子空间$\vect \calA$为$\calA$中元素的线性组合的全体.
\end{prop}

\begin{prop}
    设$A = (a_i)$为$E$中序列,则$$\vect A = \left\{ \sum\limits_{i = 0}^n \lambda_i a_i \mid \lambda_i \in \bbK \right\}$$
\end{prop}

\begin{deff}
    设$(\lambda_i)_{i \in I}$为一族标量,以$I$为指标集.
    \begin{itemize}
        \item 称集合$\{ j \in I \mid \lambda_j \neq 0 \}$为族$(\lambda_i)_{i \in I}$的\textbf{支集(support)}.
        \item 称族$(\lambda_i)_{i \in I}$是\textbf{几乎处处为0的(presque nulle)},若其支集为有限集,
            此时称该族是\textbf{有限支撑的(ɑ̀ support fini)}.
    \end{itemize}
\end{deff}

\begin{deff}
    设$(\lambda_i)_{i \in I}$为几乎处处为0的标量集,记$\sum\limits_{i \in I} \lambda_ia_i = \sum\limits_{i \in J} \lambda_ia_i$,
    其中$J$为$(\lambda_i)_{i \in I}$的支集.
\end{deff}