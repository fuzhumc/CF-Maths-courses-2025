\chapter{$\bbZ$上的算术}

\section{整除}

\begin{deff}
    设$a, b \in \bbZ$,称$b$整除$a$,若存在$k \in \bbZ$,使得$a = kb$,记为$b \mid a$.

    此时称$b$为$a$的因数,$a$为$b$的倍数.

    对于$a \in \bbZ$,$a$的因数的全体记为$\calD(a)$.

    对于$b \in \bbZ$,记$b \bbZ = \{ kb \mid k \in \bbZ \}$.
\end{deff}

\begin{rqq}
    \begin{enumerate}
        \item $\pm 1$整除任何整数.
        \item 任何整数都整除$0$.
    \end{enumerate}
\end{rqq}

\begin{prop}
    设$b \mid a$且$a \neq 0$,则$\left\lvert b \right\rvert \leqslant \left\lvert a \right\rvert$.
\end{prop}

\begin{deff}
    称$a$与$b$\textbf{相伴},若$a \mid b$且$b \mid a$.
\end{deff}

\begin{prop}
    若$a$与$b$相伴,则$a = \pm b$.
\end{prop}

\begin{rqq}
    $\bbZ$上的整除关系不是偏序,但$\bbN$上的整除关系是偏序,$1$是其最小元,$0$是其最大元.
\end{rqq}

\begin{prop}
    设$a, b \in \bbZ$,则有
    \begin{enumerate}
        \item 设$u, v, d \in \bbZ, d \mid a,b$,则$d \mid au + bv$.
        \item 设$n \in \bbZ^*$,则$a \mid b \iff na \mid nb$.
    \end{enumerate}
\end{prop}

\begin{thm}[Euclide除法]
    设$a \in \bbZ, b \in \bbN^*$,则存在唯一$q, r \in \bbZ$使得$$a = bq + r, 0 \leqslant r < b$$
\end{thm}

\begin{prop}
    设$a, b \in \bbZ$,则$b \mid a$当且仅当Euclid除法余数为0.
\end{prop}

\section{最大公因数和最小公倍数}

\begin{deff}
    设$a, b \in \bbN$,记$a$与$b$的\textbf{最大公因数}为$a \land b, \pgcd(a, b)$或$(a, b)$;记$a$与$b$的\textbf{最小公倍数}为$a \lor b, \ppcm(a, b)$或$[a, b]$.
\end{deff}

\begin{lem}
    设$a \in \bbN, b \in \bbN^*, q, r \in \bbN, a = bq + r$,则$(a, b) = (b, r)$.
\end{lem}

\begin{thm}
    设$a, b \in \bbN^*, d = (a, b), n \in \bbN$,则有$n \mid d \iff n \mid a \text{且} n \mid b$.
\end{thm}

从$\bbN$扩展到$\bbZ$
\begin{deff}
    设$a, b \in \bbZ$,记$a$与$b$的最大公因数$(a, b) = (\left\lvert a \right\rvert, \left\lvert b \right\rvert)$;记$a$与$b$的最小公倍数$[a, b] = [\left\lvert a \right\rvert, \left\lvert b \right\rvert]$.
\end{deff}

\begin{deff}
    设$a, b \in \bbZ$,称$a$与$b$\textbf{互素},若$(a, b) = 1$.
\end{deff}

\begin{thm}[Bézout]
    设$a, b \in \bbZ$,则存在$u, v \in \bbZ$使得$au + bv = (a, b)$.
\end{thm}

\begin{rqq}
    这样的$(u, v)$并不唯一.用辗转相除法可求$(u, v)$.
\end{rqq}

\begin{thm}
    设$a, b \in \bbZ$,则$(a, b) = 1 \iff \exists u, v \in \bbZ, au + bv = 1$.
\end{thm}

\begin{prop}
    设$a, b \in \bbZ, d = (a, b)$.则存在$a', b' \in \bbZ$使得$a = a'd, b = b'd, (a', b') = 1$.
\end{prop}

\begin{prop}
    设$a, b, c \in \bbZ$满足$(a, b) = (a, c) = 1$,则$(a, bc) = 1$.
\end{prop}

\begin{cor}
    设$a, b \in \bbZ$满足$(a, b) = 1$,则有$\forall m, n \in \bbN, (a^m, b^n) = 1$.
\end{cor}

\begin{deff}
    设$a_1, a_2, \cdots, a_n \in \bbZ$.
    \begin{enumerate}
        \item 称$a_1, a_2, \cdots, a_n$\textbf{互素},若$(a_1, a_2, \cdots, a_n) = 1$.
        \item 称$a_1, a_2, \cdots, a_n$\textbf{两两互素},若$\forall i, j \in \{ 1, 2, \cdots, n \}, (a_i, a_j) = 1$.
    \end{enumerate}
\end{deff}

\begin{prop}
    设$a_1, a_2, \cdots, a_n \in \bbZ$,则$(a_1, a_2, \cdots, a_n) = 1 \iff \exists u_1, u_2, \cdots, u_n \in \bbZ, a_1u_1 + a_2u_2 + \cdots + a_nu_n = 1$.
\end{prop}

\begin{prop}
    设$a, b \in \bbZ, m = [a, b]$,则有$a \mid n \text{且} b \mid n \iff m \mid n$.
\end{prop}

\begin{prop}
    设$a, b, c \in \bbN^*$,则$[ac, bc] = [a, b]c$.
\end{prop}

\begin{lem}[Gauss]
    设$a, b, c \in \bbN^*, a \mid bc$,若$(a, b) = 1$,则$a \mid c$.
\end{lem}

\begin{prop}
    设$a, b \in \bbN^*$,则有
    \begin{enumerate}
        \item 若$(a, b) = 1$,则$[a, b] = ab$.
        \item 一般地,则$(a, b)[a, b] = ab$.
    \end{enumerate}
\end{prop}

\section{素数}

\begin{deff}
    设$p \in \bbN^*, p \neq 1$.称$p$为\textbf{素数},若$p$的因数只有$1$与$p$.

    素数的全体记为$\calP$.
\end{deff}

\begin{prop}
    设$p \in \calP, n \in \calN$,则要么$p \mid n$,要么$(p, n) = 1$.

    特别地,对于$k = 1, 2, \cdots, p-1$,有$(p, k) = 1$.
\end{prop}

\begin{prop}
    严格大于1的自然数必有素因子.
\end{prop}

\begin{deff}
    设$n \in \bbN^*, p \in \calP$.称满足$p^k \mid n$的最大的$k$为$n$的$p$-\textbf{进赋值},记为$V_p(n)$.约定$V_p(0) = + \infty$.
\end{deff}

\begin{deff}
    设$n \in \bbN, p \in \calP$,记$V_p(n) = V_p(\left\lvert n \right\rvert)$.
\end{deff}

\begin{prop}
    设$k = V_p(n)$,则$p^k \mid n, p^{k+1} \nmid n$.
\end{prop}

\begin{prop}
    设$n \in \bbN^*, p \in \calP, k = V_p(n)$,则存在$q \in \bbN^*, n = p^kq, (p, q) = 1$.
\end{prop}

\begin{prop}
    设$a, b \in \bbN^*, p \in \calP$,则$V_p(ab) = V_p(a) + V_p(b)$.
\end{prop}

\begin{exer}
    设$a, b \in \bbN^*, p \in \calP$.
    \begin{enumerate}
        \item 证明:$V_p(a + b) \geqslant \min \{ V_p(a), V_p(b) \}$.
        \item 找一组$(a, b, p)$使上式不取等.
        \item 证明:若$V_p(a) \neq V_p(b)$,则上式取等.
    \end{enumerate}
\end{exer}

\begin{deff}
    设$r=\dfrac{a}{b} \in \bbQ, p \in \calP$,记$V_p(r) = V_p(a) - V_p(b)$.
\end{deff}

\begin{exer}
    设$r \in \bbQ, p \in \calP$.记$\left\lvert r \right\rvert_p = p^{- V_p(r)}$,约定$p^{- \infty} = 0$.证明:
    \begin{enumerate}
        \item $\left\lvert r \right\rvert_p \geqslant 0$,当且仅当$r=0$时取等.
        \item $\left\lvert r_1 + r_2 \right\rvert_p \leqslant \max \{ \left\lvert r_1 \right\rvert_p, \left\lvert r_2 \right\rvert_p \}$.
        \item $\left\lvert r_1r_2 \right\rvert_p = \left\lvert r_1 \right\rvert_p \left\lvert r_2 \right\rvert_p$.
    \end{enumerate}
\end{exer}