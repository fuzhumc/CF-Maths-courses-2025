\chapter{序列极限}

让我们回顾一下序列收敛的定义.

\begin{deff}
    设$(x_n)$为$\mathbb{R}$中序列,$x \in \mathbb{R}$.称$(x_n)$收敛于$x$,若
    $$
    \forall \varepsilon>0,\exists N \in \mathbb{N},\text{当}n>N\text{时,有}\left\lvert x_n - x \right\rvert < \varepsilon.
    $$
    记为$\lim\limits_{n \to \infty}x_n=x$.
\end{deff}

\section{收敛序列的性质}

\begin{thm}
    收敛序列的极限唯一.
\end{thm}

\begin{pf}
    设$\lim\limits_{n \to \infty} u_n = l$且$\lim\limits_{n \to \infty} u_n = \tilde{l}$.
    令$\varepsilon=\dfrac{\left\lvert l-\tilde{l} \right\rvert}{47}$.
    则存在$N_1$,当$n>N_1$时,$\left\lvert u_n - l \right\rvert < \varepsilon$.
    存在$N_2$,当$n>N_2$时,$\left\lvert u_n - \tilde{l} \right\rvert < \varepsilon$.
    令$N=\max\{N_1,N_2\}$,当$n>N$时,$\left\lvert l - \tilde{l} \right\rvert \leq \left\lvert l - u_n \right\rvert + \left\lvert u_n - \tilde{l} \right\rvert < 2\varepsilon = \dfrac{2\left\lvert l-\tilde{l} \right\rvert}{47}$.
    因此$l=\tilde{l}$,即极限唯一.
\end{pf}

\begin{prop}
    若$\lim\limits_{n\to\infty}u_n=u$,则$\lim\limits_{n\to\infty} \left\lvert u_n \right\rvert = \left\lvert u \right\rvert$.
\end{prop}

\begin{pf}
    由$\left\lvert \left\lvert u_n \right\rvert - \left\lvert u \right\rvert \right\rvert \leq \left\lvert u_n - u \right\rvert$易证.
\end{pf}

\begin{prop}
    收敛序列一定有界.
\end{prop}

\begin{pf}
    收敛序列必为Cauchy列,而Cauchy列必有界.
\end{pf}

\begin{prop}
    设序列$(u_n)$收敛且$u_n \geqslant l$,则$\lim\limits_{n\to\infty}u_n \geqslant l$.
\end{prop}

\begin{prop}
    设序列$(u_n)$收敛且$\lim\limits_{n\to\infty} u_n \geqslant l$,
    则存在$N \in \mathbb{N}$,当$n>N$时,$u_n \geqslant l$.
\end{prop}

\begin{deff}
    称序列$(u_n)$发散至$+\infty(-\infty)$,若对于任意$M \in \mathbb{R}$,存在$N \in \mathbb{N}$,
    当$n>N$时,$u_n>M(u_n<M)$.
\end{deff}

\begin{rqq}
    此时也称$(u_n)$极限不存在.
\end{rqq}

\begin{prop}
    若序列$(u_n)$收敛于$l$,则$(u_n)$的任何子列都收敛于$l$.
\end{prop}

\begin{prop}
    若序列$(u_n)$单调增且无上界,则$(u_n)$发散至$+\infty$.
\end{prop}

\begin{deff}
    称两序列$(u_n),(v_n)$是相邻的,若$(u_n)$单调增,$(v_n)$单调减,
    且$\lim\limits_{n\to\infty}(u_n-v_n)=0$.
\end{deff}

\begin{thm}
    相邻序列均收敛,且极限相同.
\end{thm}

\begin{thm}[序列极限的运算]
    设$\lim\limits_{n\to\infty}u_n=u,\lim\limits_{n\to\infty}v_n=v$,则有
    \begin{enumerate}
        \item $\lim\limits_{n\to\infty}(u_n+v_n)=u+v$.
        \item $\lim\limits_{n\to\infty}(u_n-v_n)=u-v$.
        \item $\lim\limits_{n\to\infty}(u_nv_n)=uv$.
        \item 若$v_n \neq 0,v \neq 0$,则$\lim\limits_{n\to\infty}\dfrac{u_n}{v_n}=\dfrac{u}{v}$.
    \end{enumerate}
\end{thm}

\begin{deff}
    设$(u_n)$为实序列,
    记$\varlimsup\limits_{n\to\infty}u_n=\lim\limits_{n\to\infty}\sup\limits_{m \geqslant n}u_n$
    称为$(u_n)$的上极限,
    记$\varliminf\limits_{n\to\infty}u_n=\lim\limits_{n\to\infty}\inf\limits_{m \geqslant n}u_n$
    称为$(u_n)$的下极限.
\end{deff}

\begin{thm}
    $\lim\limits_{n\to\infty}u_n=l \Leftrightarrow 
    \varlimsup\limits_{n\to\infty}u_n=\varliminf\limits_{n\to\infty}u_n=l.$
\end{thm}

\begin{prop}
    若$\varlimsup\limits_{n\to\infty}u_n=l$,
    则存在$(u_n)$的一个子列收敛于$l$.
\end{prop}

\begin{pf}
    记$b_n=\sup\limits_{m \geqslant n}u_n \to l$.
    则对于任意$\varepsilon>0$,存在$N\in\mathbb{N}$,使得$\left\lvert b_N-l \right\rvert \leqslant \varepsilon$.
    取$\varepsilon=\frac{1}{2n},\varphi(n) \geqslant N$满足$b_n-\frac{1}{2n}<a_{\varphi(n)}<b_N$.
    则$\lim\limits_{n\to\infty}a_{\varphi(n)}=l$.
\end{pf}

\section{$\mathbb{R}$中稠密类}

\begin{deff}
    设$A \subset \mathbb{R}$,称$A$在$\mathbb{R}$中稠密,若对于任意$a,b \in \mathbb{R},a<b$,
    存在$x \in A$,使得$a<x<b$.
\end{deff}

\begin{prop}
    $A$在$\mathbb{R}$中稠密$\Leftrightarrow$对于任意$x \in \mathbb{R},\varepsilon>0,
    (x-\varepsilon,x+\varepsilon) \cap A \neq \varnothing$.
\end{prop}

\begin{prop}
    $A$在$\mathbb{R}$中稠密$\Leftrightarrow$对于任意$x \in \mathbb{R}$,存在$A$中序列收敛于$x$.
\end{prop}

\begin{prop}
    设$A\subseteq\mathbb{R}$有上界$s$,且$A$在$\mathbb{R}$中稠密.
    则$s=\sup A$当且仅当存在$A$中序列收敛于$s$.
\end{prop}

\section{序列之间的比较关系}

\begin{deff}
    设序列$u=(u_n),v=(v_n)$从某项开始非$0$.
    称$(u_n)$等价于$(v_n)$,若$\lim\limits_{n\to\infty}\dfrac{u_n}{v_n}=1$.
    记为$u \sim v$或$u_n \sim v_n$.
\end{deff}

\begin{prop}
    设$u \sim u',v \sim v'$,则$uu' \sim vv'$.
\end{prop}

\begin{deff}
    设序列$u=(u_n),v=(v_n)$从某项开始非$0$.
    称$u$受$v$控制,若存在$N \in \mathbb{N}$,当$n>N$时,$\left\lvert \dfrac{u_n}{v_n} \right\rvert$有界.
    记为$u = O(v)$或$u_n = O(v_n)$.
\end{deff}

\begin{deff}
    设序列$u=(u_n),v=(v_n)$从某项开始非$0$.
    称$u$在$v$前可忽略,若$\lim\limits_{n\to\infty}\dfrac{u_n}{v_n}=0$.
    记为$u = o(v)$或$u_n = o(v_n)$.
\end{deff}

\begin{prop}
    设序列$u=(u_n),v=(v_n)$从某项开始非$0$.
    则$u \sim v \Leftrightarrow u-v=o(v)$.
\end{prop}