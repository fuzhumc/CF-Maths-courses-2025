\chapter{函数积分}

\section{阶梯函数}

\begin{deff}
    $[a,b]$的\textbf{分割}是指一个有限序列$\sigma=(u_k)$,其中$a=u_0<u_1<u_2<\cdots<u_n=b$.
\end{deff}

\begin{deff}
    分割$\sigma=(u_k)$的\textbf{支集}$\mathrm{supp}(\sigma)=\{u_k \mid k=0,1,2,\cdots,n\}$
\end{deff}

\begin{deff}
    分割$\sigma=(u_k)$的\textbf{宽度}$\delta(\sigma)=\max\limits_{1 \leqslant k \leqslant n}(u_k-u_{k-1})$
\end{deff}

\begin{deff}
    设$\sigma,\sigma'$为$[a,b]$的两个分割,称$\sigma'$比$\sigma$\textbf{更细},若$\mathrm{supp}(\sigma)\subseteq\mathrm{supp}(\sigma')$
\end{deff}

\begin{prop}
    设$\sigma,\sigma'$为$[a,b]$的两个分割,则存在$[a,b]$的另一个分割$\sigma''$,他比$\sigma,\sigma'$都更细.
\end{prop}

\begin{deff}
    设$f:[a,b]\to\mathbb{R}$.称$f$为\textbf{阶梯函数},若存在$[a,b]$的一个分割$\sigma=(u_k)$,使得$f\vert_{(u_k-1,u_k)}$为常值函数.
    此时称分割$\sigma$适应于$f$.$[a,b]$上阶梯函数的全体记为$\mathcal{E}([a,b],\mathbb{R})$.
\end{deff}

\begin{prop}
    设$f\in\mathcal{E}([a,b],\mathbb{R})$,$\sigma$适应于$f$,则存在比$\sigma$更细的分割$\sigma'$也适应于$f$.
\end{prop}

\begin{prop}
    设$f,g\in\mathcal{E}([a,b],\mathbb{R})$,则存在$[a,b]$的一个分割$\sigma$,它同时适应于$f,g$.
\end{prop}

\begin{prop}
    设$f\in\mathcal{E}([a,b],\mathbb{R})$,$\sigma=(u_k)$适应于$f$.
    记$f_k$为$f$在$(u_{k-1},u_k)$上的取值,则$S_\sigma(f)=\sum\limits_{k=1}^n(u_{k}-u_{k-1})f_k$的取值与$\sigma$的选取无关.
\end{prop}

\begin{deff}
    记$\int_{[a,b]}f=S_\sigma(f)$称为$f$的\textbf{积分}.
\end{deff}

\begin{prop}
    设$f\in\mathcal{E}([a,b],\mathbb{R}),\lambda\in\mathbb{R}$,则:
    \begin{enumerate}
        \item $$\int_{[a,b]}\lambda f=\lambda\int_{[a,b]}f$$
        \item $$\int_{[a,b]}f+g=\int_{[a,b]}f+\int_{[a,b]}g$$
    \end{enumerate}
\end{prop}

\begin{rqq}
    $$\int_{[a,b]}:\begin{array}[t]{ccc}\mathcal{E}([a,b],\mathbb{R}) & \to &\mathbb{R} \\ f & \mapsto & \int_{[a,b]}f \end{array}$$
    可以看作$\mathcal{E}([a,b],\mathbb{R})$上的一个``函数''.
\end{rqq}

\begin{prop}
    设$f,g\in\mathcal{E}([a,b],\mathbb{R})$,则有:
    \begin{enumerate}
        \item 若$f\geqslant0$,则$\int_{[a,b]}f\geqslant0$.
        \item 若$f \geqslant g$,则$\int_{[a,b]}f \geqslant \int_{[a,b]}g$.
    \end{enumerate}
\end{prop}

\begin{prop}
    设$f\in\mathcal{E}([a,b],\mathbb{R}),c\in[a,b]$,则
    $$\int_{[a,b]}f=\int_{[a,c]}f\vert_{[a,c]}+\int_{[c,b]}f\vert_{[c,b]}$$
\end{prop}

\section{分段连续函数}

\begin{deff}
    设$f:[a,b]\to\mathbb{R}$.称$f$为\textbf{分段连续函数},若存在$[a,b]$的一个分割$\sigma=(u_k)$,使得$f\vert_{(u_k-1,u_k)}$为连续函数,且$f(u_k)$有界.
    此时称分割$\sigma$适应于$f$.$[a,b]$上分段连续函数的全体记为$\mathcal{CM}([a,b],\mathbb{R})$.
\end{deff}

\begin{prop}
    设$f\in\mathcal{CM}([a,b],\mathbb{R})$,则$f$有界.
\end{prop}

\begin{prop}
    设$f,g\in\mathcal{CM}([a,b],\mathbb{R}),\lambda,\mu\in\mathbb{R}$,则$\lambda f+\mu g\in\mathcal{CM}([a,b],\mathbb{R})$.
\end{prop}

\begin{prop}
    设$f,g\in\mathcal{CM}([a,b],\mathbb{R})$,则$fg\in\mathcal{CM}([a,b],\mathbb{R})$.
\end{prop}

\begin{prop}
    设$f\in\mathcal{CM}([a,b],\mathbb{R}),g\in\mathcal{CM}([c,d],\mathbb{R}),f([a,b])\subseteq[c,d]$,则$g\circ f\in\mathcal{CM}([a,b],\mathbb{R})$.
\end{prop}

\begin{lem}
    设$f\in\mathcal{CM}([a,b],\mathbb{R})$,给定$\varepsilon>0$,
    则存在$\varphi\in\mathcal{E}([a,b],\mathbb{R})$使得$\sup\left\lvert f-\varphi \right\rvert < \varepsilon$.
\end{lem}

\begin{prop}
    设$f\in\mathcal{CM}([a,b],\mathbb{R})$,则存在$\varphi_n\in\mathcal{E}([a,b],\mathbb{R}),n\in\mathbb{N}$,
    使得$\lim\limits_{n\to\infty}\sup\left\lvert f-\varphi_n \right\rvert = 0$.
\end{prop}

\begin{thm}
    设$f\in\mathcal{CM}([a,b],\mathbb{R}),\varphi_n\in\mathcal{E}([a,b],\mathbb{R})$满足$\lim\limits_{n\to\infty}\sup\left\lvert f-\varphi_n \right\rvert = 0$,
    则$\int_{[a,b]}\varphi_n$收敛,且极限值与$(\varphi_n)$的选取无关.
\end{thm}

\begin{deff}
    设$f\in\mathcal{CM}([a,b],\mathbb{R}),\varphi_n\in\mathcal{E}([a,b],\mathbb{R})$满足$\lim\limits_{n\to\infty}\sup\left\lvert f-\varphi_n \right\rvert = 0$,
    称$\lim\limits_{n\to\infty}\int_{[a,b]}\varphi_n$为$f$的积分,记为$\int_{[a,b]}f$.
\end{deff}

\begin{prop}
    设$f,g\in\mathcal{CM}([a,b],\mathbb{R}),\lambda\in\mathbb{R}$,则:
    \begin{enumerate}
        \item $$\int_{[a,b]}\lambda f=\lambda\int_{[a,b]}f$$
        \item $$\int_{[a,b]}f+g=\int_{[a,b]}f+\int_{[a,b]}g$$
    \end{enumerate}
\end{prop}

\begin{prop}
    设$f,g\in\mathcal{CM}([a,b],\mathbb{R})$仅在有限个点上取不同值,则$\int_{[a,b]}f=\int_{[a,b]}g$.
\end{prop}

\begin{prop}
    设$f\in\mathcal{CM}([a,b],\mathbb{R})$,则$\left\lvert f \right\rvert \in\mathcal{CM}([a,b],\mathbb{R})$,
    且$\int_{[a,b]} \left\lvert f \right\rvert \geqslant \left\lvert \int_{[a,b]}f \right\rvert$.
\end{prop}

\begin{prop}
    设$f,g\in\mathcal{CM}([a,b],\mathbb{R})$,则
    \begin{enumerate}
        \item 若$f\geqslant0$,则$\int_{[a,b]}f\geqslant0$.
        \item 若$f \geqslant g$,则$\int_{[a,b]}f \geqslant \int_{[a,b]}g$.
    \end{enumerate}
\end{prop}

\begin{prop}
    设$f\in\mathcal{CM}([a,b],\mathbb{R}),\left\lvert f \right\rvert \leqslant k$,
    则$$\int_{[a,b]}f\leqslant k(b-a)$$
\end{prop}

\begin{prop}
    设$f\in\mathcal{CM}([a,b],\mathbb{R}),c\in[a,b]$,则
    $$\int_{[a,b]}f=\int_{[a,c]}f\vert_{[a,c]}+\int_{[c,b]}f\vert_{[c,b]}$$
\end{prop}

\begin{thm}
    设$f\in\mathcal{C}([a,b],\mathbb{R}_+)$,若$f$不恒为0,则$\int_{[a,b]}f>0$.
\end{thm}

\begin{proof}
    由题意知,存在$x_0\in[a,b],f(x_0)>0$,因此$\lim\limits_{x\to x_0}f(x)>0$.

    即,存在$\delta>0$,当$x\in[x_0-\delta,x_0+\delta]$时,有$f(x)>0$.

    故$\int_{[a,b]}f\geqslant\int_{[x_0-\delta,x_0+\delta]}f>0$.
\end{proof}

\begin{cor}
    设$f\in\mathcal{C}([a,b],\mathbb{R}_+)$,若$\int_{[a,b]}f=0$,则$f$恒为0.
\end{cor}

\begin{prop}
    设$f\in\mathcal{C}([a,b],\mathbb{R})$,若$\int_{[a,b]}f=0$,则$f$在$[a,b]$上有零点.
\end{prop}

\begin{thm}[Cauchy不等式]
    设$f,g\in\mathcal{CM}([a,b],\mathbb{R})$,则
    $$\int_{[a,b]}f^2\int_{[a,b]}g^2\geqslant\left(\int_{[a,b]}fg\right)^2$$
\end{thm}

\begin{thm}[Hölder不等式]
    设$f,g\in\mathcal{CM}{[a,b],\mathbb{R}_+},p,q>1$满足$\frac{1}{p}+\frac{1}{q}=1$,则
    $$\left(\int_{[a,b]}f^p\right)^{\frac{1}{p}}\left(\int_{[a,b]}g^q\right)^{\frac{1}{q}}\geqslant\int_{[a,b]}fg$$
\end{thm}

\section{Riemann可积函数}

\begin{deff}
    设$f:[a,b]\to\mathbb{R}$.称$f$在$[a,b]$上\textbf{Riemann可积},若存在$I\in\mathbb{R}$满足
    对任意$\varepsilon>0$,存在$\delta>0$,当$\delta(\sigma)<\delta$时,有
    $\forall \xi_k \in (u_{k-1},u_k),\left\lvert \sum\limits_{k=1}^n f(\xi_k)(u_k-u_{k-1})-I\right\rvert<\varepsilon$.
    
    此时记$\int_{[a,b]}f=I$.

    $[a,b]$上可积函数的全体记为$\mathcal{R}([a,b],\mathbb{R})$.
\end{deff}

\begin{deff}
    设$\sigma=(u_k)$为$[a,b]$的一个分割,$f:[a,b]\to\mathbb{R}$,

    记$M_k=\sup\limits_{x\in[x_{k-1},x_k]}f(x),m_k=\inf\limits_{x\in[x_{k-1},x_k]}f(x)$.
    
    称$\sum\limits_{k=1}^n M_k(u_k-u_{k-1})$为\textbf{Darboux上和},记为$\bar{S}(f,\sigma)$.
    
    称$\sum\limits_{k=1}^n m_k(u_k-u_{k-1})$为\textbf{Darboux下和},记为$\underline{S}(f,\sigma)$.
\end{deff}

\begin{prop}
    设$\sigma,\sigma'$为$[a,b]$的分割,$f;[a,b]\to\mathbb{R}$,若$\sigma'$比$\sigma$更细,则
    $\bar{S}(f,\sigma')\leqslant\bar{S}(f,\sigma),\underline{S}(f,\sigma')\geqslant\underline{S}(f,\sigma)$
\end{prop}

\begin{prop}
    设$\sigma,\sigma'$为$[a,b]$的分割,$f:[a,b]\to\mathbb{R}$,则
    $\bar{S}(f,\sigma)\geqslant\underline{S}(f,\sigma')$.
\end{prop}

\begin{deff}
    设$\sigma$为$[a,b]$的分割,$f:[a,b]\to\mathbb{R}$.
    \begin{enumerate}
        \item 称$\inf\limits_{\sigma}\bar{S}(f,\sigma)$为\textbf{上积分},记为$\bar{I}$.
        \item 称$\sup\limits_{\sigma}\underline{S}(f,\sigma)$为\textbf{下积分},记为$\underline{I}$.
    \end{enumerate}
\end{deff}

\begin{prop}
    设$\sigma$为$[a,b]$的分割,$f:[a,b]\to\mathbb{R}$,则
    $\bar{S}(f,\sigma)\geqslant\bar{I}\geqslant\underline{I}\geqslant\underline{S}(f,\sigma)$.
\end{prop}

\begin{thm}
    设$f:[a,b]\to\mathbb{R}$,则$f$在$[a,b]$上Riemann可积$\iff\bar{I}=\underline{I}$.
\end{thm}

\begin{prop}
    \begin{enumerate}
        \item 单调函数Riemann可积.
        \item 闭区间上连续函数Riemann可积.
        \item 只有有限个间断点的函数Riemann可积.
    \end{enumerate}
\end{prop}

\begin{deff}
    记$$\int_a^b f(x)\mathrm{d}x=\begin{cases} \int_{[a,b]}f, & a<b \\ 0, & a=b \\ -\int_{[b,a]}f, & a>b \end{cases}$$
\end{deff}

\begin{prop}
    $$\int_a^b f(x)\mathrm{d}x=-\int_b^a f(x)\mathrm{d}x$$
\end{prop}

\begin{prop}
    $$\int_a^b f(x)\mathrm{d}x=\int_a^c f(x)\mathrm{d}x+\int_c^b f(x)\mathrm{d}x$$
\end{prop}

\begin{prop}
    设$f\in\mathcal{R}(I,\mathbb{C})$,则$\int_a^b \left\lvert f \right\rvert \mathrm{d}x \geqslant \left\lvert \int_a^b f(x)\mathrm{d}x \right\rvert$.
\end{prop}

\begin{deff}
    设$f,F:[a,b]\to\mathbb{R}$,称$F$为$f$的\textbf{原函数},若$F'=f$.
\end{deff}

\begin{prop}
    设$F,G$分别为$f,g$的原函数,$\lambda,\mu\in\mathbb{R}$,则$\lambda F+\mu G$为$\lambda f+\mu g$的原函数.
\end{prop}

\begin{thm}
    设$f\in\mathcal{C}([a,b],\mathbb{R})$,则
    $F:\begin{array}[t]{ccc} [a,b] & \to & \mathbb{R} \\ x & \mapsto & \displaystyle\int_a^x f(t)\mathrm{d}t \end{array}$
    为$f$的一个原函数.
\end{thm}

\begin{rqq}
    上述函数是$f$的原函数中唯一一个在$a$处取0的.
\end{rqq}

\begin{thm}[Newton-Leibniz]
    设$f\in\mathcal{C}([a,b],\mathbb{R})$,$F$为$f$的原函数,则$$\int_a^b f(x)\mathrm{d}x=F(b)-F(a)$$
\end{thm}

\begin{rqq}
    设$f\in\mathcal{R}([a,b],\mathbb{R})$,则$f$不一定有原函数.
\end{rqq}

\begin{thm}
    设$f\in\mathcal{R}([a,b],\mathbb{R})$,若$f$有原函数$F$,则$$\int_a^b f(x)\mathrm{d}x=F(b)-F(a)$$
\end{thm}

\begin{prop}
    设$f\in\mathcal{C}(I,\mathbb{R}),\alpha,\beta\in\mathcal{C}^1(J,I)$.
    令$g(x)=\int_{\alpha(x)}^{\beta(x)}f(t)\mathrm{d}t$,则$g$在$J$上可导,
    且$g'(x)=f'(\beta(x))\beta'(x)-f'(\alpha(x))\alpha'(x)$.
\end{prop}

\begin{prop}
    设$u,v\in\mathcal{C}^1(I,\mathbb{R})$,则$$\int u'(x)v(x)\mathrm{d}x=u(x)v(x)-\int u(x)v'(x)\mathrm{d}x$$
\end{prop}

\begin{prop}
    设$f\in\mathcal{C}(I,\mathbb{R}),u\in\mathcal{C}^1(J,I),a,b\in J$,则
    $$\int_{u(a)}^{u(b)}f(x)\mathrm{d}x=\int_a^b f(u(t))u'(t)$$
\end{prop}

\begin{prop}
    设$f\in\mathcal{C}([a,b],\mathbb{R}),T\in\mathbb{R}$,
    则$\int_{a+T}^{b+T} f(x-T)\mathrm{d}x=\int_a^b f(x)\mathrm{d}x$.
\end{prop}

\begin{prop}
    设$f:\mathbb{R}\to\mathbb{R}$为以$T$的周期,$a\in\mathbb{R}$,则$\int_a^{a+T}f(x)\mathrm{d}x=\int_0^T f(x)\mathrm{d}x$.
\end{prop}

\begin{thm}[第一积分中值]
    设$f\in\mathcal{C}([a,b],\mathbb{R})$,则存在$\xi\in[a,b]$使得$\int_a^b f(x)\mathrm{d}x=f(\xi)(b-a)$.
\end{thm}

\begin{proof}
    设$m=\min\limits_{x\in[a,b]}f(x),M=\max\limits_{x\in[a,b]}f(x)$,
    则$m(b-a)\leqslant\int_a^bf(x)\mathrm{d}x\leqslant M(b-a)$,
    即$\dfrac{\int_a^b f(x)\mathrm{d}x}{b-a}\in[m,M]$.

    由介值定理知,存在$\xi\in[a,b]$使得$\dfrac{\int_a^b f(x)\mathrm{d}x}{b-a}=f(\xi)$,
    即$\int_a^b f(x)\mathrm{d}x=f(\xi)(b-a)$.
\end{proof}

\begin{thm}[第二积分中值]
    设$f\in\mathcal{C}^1([a,b],\mathbb{R}),g\in\mathcal{C}([a,b],\mathbb{R}_+)$,$f$单调减
    则存在$c\in[a,b]$使得$\int_a^b f(x)g(x)\mathrm{d}x=f(a)\int_a^c g(x)\mathrm{d}x$.
\end{thm}