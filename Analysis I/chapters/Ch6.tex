\chapter{函数积分}

\section{阶梯函数}

\begin{deff}
    $[a,b]$的分割是指一个有限序列$\sigma=(u_k)$,其中$a=u_0<u_1<u_2<\cdots<u_n=b$.
\end{deff}

\begin{deff}
    分割$\sigma=(u_k)$的支集$\mathrm{supp}(\sigma)=\{u_k \mid k=0,1,2,\cdots,n\}$
\end{deff}

\begin{deff}
    分割$\sigma=(u_k)$的宽度$\delta(\sigma)=\max\limits_{1 \leqslant k \leqslant n}(u_k-u_{k-1})$
\end{deff}

\begin{deff}
    设$\sigma,\sigma'$为$[a,b]$的两个分割,称$\sigma'$比$\sigma$更细,若$\mathrm{supp}(\sigma)\subseteq\mathrm{supp}(\sigma')$
\end{deff}

\begin{prop}
    设$\sigma,\sigma'$为$[a,b]$的两个分割,则存在$[a,b]$的另一个分割$\sigma''$,他比$\sigma,\sigma'$都更细.
\end{prop}

\begin{deff}
    设$f:[a,b]\to\mathbb{R}$.称$f$为阶梯函数,若存在$[a,b]$的一个分割$\sigma=(u_k)$,使得$f\vert_{(u_k-1,u_k)}$为常值函数.
    此时称分割$\sigma$适应于$f$.$[a,b]$上阶梯函数的全体记为$\mathcal{E}([a,b],\mathbb{R})$.
\end{deff}

\begin{prop}
    设$f\in\mathcal{E}([a,b],\mathbb{R})$,$\sigma$适应于$f$,则存在比$\sigma$更细的分割$\sigma'$也适应于$f$.
\end{prop}

\begin{prop}
    设$f,g\in\mathcal{E}([a,b],\mathbb{R})$,则存在$[a,b]$的一个分割$\sigma$,它同时适应于$f,g$.
\end{prop}

\begin{prop}
    设$f\in\mathcal{E}([a,b],\mathbb{R})$,$\sigma=(u_k)$适应于$f$.
    记$f_k$为$f$在$(u_{k-1},u_k)$上的取值,则$S_\sigma(f)=\sum\limits_{k=1}^n(u_{k}-u_{k-1})f_k$的取值与$\sigma$的选取无关.
\end{prop}

\begin{deff}
    记$\int_{[a,b]}f=S_\sigma(f)$称为$f$的积分.
\end{deff}

\begin{prop}
    设$f\in\mathcal{E}([a,b],\mathbb{R}),\lambda\in\mathbb{R}$,则:
    \begin{enumerate}
        \item $$\int_{[a,b]}\lambda f=\lambda\int_{[a,b]}f$$
        \item $$\int_{[a,b]}f+g=\int_{[a,b]}f+\int_{[a,b]}g$$
    \end{enumerate}
\end{prop}

\begin{rqq}
    $$\int_{[a,b]}:\begin{array}[t]{ccc}\mathcal{E}([a,b],\mathbb{R}) & \to &\mathbb{R} \\ f & \mapsto & \int_{[a,b]}f \end{array}$$
    可以看作$\mathcal{E}([a,b],\mathbb{R})$上的一个``函数''.
\end{rqq}

\begin{prop}
    设$f,g\in\mathcal{E}([a,b],\mathbb{R})$,则有:
    \begin{enumerate}
        \item 若$f\geqslant0$,则$\int_{[a,b]}f\geqslant0$.
        \item 若$f \geqslant g$,则$\int_{[a,b]}f \geqslant \int_{[a,b]}g$.
    \end{enumerate}
\end{prop}

\begin{prop}
    设$f\in\mathcal{E}([a,b],\mathbb{R}),c\in[a,b]$,则
    $$\int_{[a,b]}f=\int_{[a,c]}f\vert_{[a,c]}+\int_{[c,b]}f\vert_{[c,b]}$$
\end{prop}

\section{分段连续函数}

\begin{deff}
    设$f:[a,b]\to\mathbb{R}$.称$f$为分段连续函数,若存在$[a,b]$的一个分割$\sigma=(u_k)$,使得$f\vert_{(u_k-1,u_k)}$为连续函数,且$f(u_k)$有界.
    此时称分割$\sigma$适应于$f$.$[a,b]$上分段连续函数的全体记为$\mathcal{CM}([a,b],\mathbb{R})$.
\end{deff}

\begin{prop}
    设$f\in\mathcal{CM}([a,b],\mathbb{R})$,则$f$有界.
\end{prop}

\begin{prop}
    设$f,g\in\mathcal{CM}([a,b],\mathbb{R}),\lambda,\mu\in\mathbb{R}$,则$\lambda f+\mu g\in\mathcal{CM}([a,b],\mathbb{R})$.
\end{prop}

\begin{prop}
    设$f,g\in\mathcal{CM}([a,b],\mathbb{R})$,则$fg\in\mathcal{CM}([a,b],\mathbb{R})$.
\end{prop}

\begin{prop}
    设$f\in\mathcal{CM}([a,b],\mathbb{R}),g\in\mathcal{CM}([c,d],\mathbb{R}),f([a,b])\subseteq[c,d]$,则$g\circ f\in\mathcal{CM}([a,b],\mathbb{R})$.
\end{prop}

\begin{lem}
    设$f\in\mathcal{CM}([a,b],\mathbb{R})$,给定$\varepsilon>0$,
    则存在$\varphi\in\mathcal{E}([a,b],\mathbb{R})$使得$\sup\left\lvert f-\varphi \right\rvert < \varepsilon$.
\end{lem}

\begin{prop}
    设$f\in\mathcal{CM}([a,b],\mathbb{R})$,则存在$\varphi_n\in\mathcal{E}([a,b],\mathbb{R}),n\in\mathbb{N}$,
    使得$\lim\limits_{n\to\infty}\sup\left\lvert f-\varphi_n \right\rvert = 0$.
\end{prop}

\begin{thm}
    设$f\in\mathcal{CM}([a,b],\mathbb{R}),\varphi_n\in\mathcal{E}([a,b],\mathbb{R})$满足$\lim\limits_{n\to\infty}\sup\left\lvert f-\varphi_n \right\rvert = 0$,
    则$\int_{[a,b]}\varphi_n$收敛,且极限值与$(\varphi_n)$的选取无关.
\end{thm}

\begin{deff}
    设$f\in\mathcal{CM}([a,b],\mathbb{R}),\varphi_n\in\mathcal{E}([a,b],\mathbb{R})$满足$\lim\limits_{n\to\infty}\sup\left\lvert f-\varphi_n \right\rvert = 0$,
    称$\lim\limits_{n\to\infty}\int_{[a,b]}\varphi_n$为$f$的积分,记为$\int_{[a,b]}f$.
\end{deff}

\begin{prop}
    设$f,g\in\mathcal{CM}([a,b],\mathbb{R}),\lambda\in\mathbb{R}$,则:
    \begin{enumerate}
        \item $$\int_{[a,b]}\lambda f=\lambda\int_{[a,b]}f$$
        \item $$\int_{[a,b]}f+g=\int_{[a,b]}f+\int_{[a,b]}g$$
    \end{enumerate}
\end{prop}

\begin{prop}
    设$f,g\in\mathcal{CM}([a,b],\mathbb{R})$仅在有限个点上取不同值,则$\int_{[a,b]}f=\int_{[a,b]}g$.
\end{prop}

\begin{prop}
    设$f\in\mathcal{CM}([a,b],\mathbb{R})$,则$\left\lvert f \right\rvert \in\mathcal{CM}([a,b],\mathbb{R})$,
    且$\int_{[a,b]} \left\lvert f \right\rvert \geqslant \left\lvert \int_{[a,b]}f \right\rvert$.
\end{prop}

\begin{prop}
    设$f,g\in\mathcal{CM}([a,b],\mathbb{R})$,则
    \begin{enumerate}
        \item 若$f\geqslant0$,则$\int_{[a,b]}f\geqslant0$.
        \item 若$f \geqslant g$,则$\int_{[a,b]}f \geqslant \int_{[a,b]}g$.
    \end{enumerate}
\end{prop}

\begin{prop}
    设$f\in\mathcal{CM}([a,b],\mathbb{R}),\left\lvert f \right\rvert \leqslant k$,
    则$$\int_{[a,b]}f\leqslant k(b-a)$$
\end{prop}

\begin{prop}
    设$f\in\mathcal{CM}([a,b],\mathbb{R}),c\in[a,b]$,则
    $$\int_{[a,b]}f=\int_{[a,c]}f\vert_{[a,c]}+\int_{[c,b]}f\vert_{[c,b]}$$
\end{prop}

\begin{thm}
    设$f\in\mathcal{C}([a,b],\mathbb{R}_+)$,若$f$不恒为0,则$\int_{[a,b]}f>0$.
\end{thm}

\begin{proof}
    由题意知,存在$x_0\in[a,b],f(x_0)>0$,因此$\lim\limits_{x\to x_0}f(x)>0$.

    即,存在$\delta>0$,当$x\in[x_0-\delta,x_0+\delta]$时,有$f(x)>0$.

    故$\int_{[a,b]}f\geqslant\int_{[x_0-\delta,x_0+\delta]}f>0$.
\end{proof}

\begin{cor}
    设$f\in\mathcal{C}([a,b],\mathbb{R}_+)$,若$\int_{[a,b]}f=0$,则$f$恒为0.
\end{cor}

\begin{prop}
    设$f\in\mathcal{C}([a,b],\mathbb{R})$,若$\int_{[a,b]}f=0$,则$f$在$[a,b]$上有零点.
\end{prop}

\begin{thm}[Cauchy不等式]
    设$f,g\in\mathcal{CM}([a,b],\mathbb{R})$,则
    $$\int_{[a,b]}f^2\int_{[a,b]}g^2\geqslant\left(\int_{[a,b]}fg\right)^2$$
\end{thm}

\begin{thm}[Hölder不等式]
    设$f,g\in\mathcal{CM}{[a,b],\mathbb{R}_+},p,q>1$满足$\frac{1}{p}+\frac{1}{q}=1$,则
    $$\left(\int_{[a,b]}f^p\right)^{\frac{1}{p}}\left(\int_{[a,b]}g^q\right)^{\frac{1}{q}}\geqslant\int_{[a,b]}fg$$
\end{thm}