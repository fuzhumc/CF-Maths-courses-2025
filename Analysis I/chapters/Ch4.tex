\chapter{函数极限和函数连续性}

\section{函数极限}

\begin{deff}
    称$\lim\limits_{x \to a} f(x)=l$,若对于任意$\varepsilon>0$,存在$\delta>0$,
    当$0< \left\lvert x-a \right\rvert \leqslant \delta$,时,
    有$\left\lvert f(x)-l \right\rvert \leqslant \varepsilon$.
\end{deff}

\begin{deff}
    称$\lim\limits_{x \to +\infty} f(x)=l$,若对于任意$\varepsilon>0$,存在$M>0$,
    当$x>M$时,有$\left\lvert f(x)-l \right\rvert \leqslant \varepsilon$.

    称$\lim\limits_{x \to -\infty} f(x)=l$,若对于任意$\varepsilon>0$,存在$M<0$,
    当$x<M$时,有$\left\lvert f(x)-l \right\rvert \leqslant \varepsilon$.
\end{deff}

\begin{deff}
    称$\lim\limits_{x \to a} f(x)=+\infty$,若对于任意$M>0$,存在$\delta>0$,
    当$0< \left\lvert x-a \right\rvert \leqslant \delta$,时,有$f(x) \geqslant M$.

    称$\lim\limits_{x \to a} f(x)=-\infty$,若对于任意$M<0$,存在$\delta>0$,
    当$0< \left\lvert x-a \right\rvert \leqslant \delta$,时,有$f(x) \leqslant M$.
\end{deff}

\begin{prop}
    若$\lim\limits_{x \to a} f(x)=l$,则$f$在$a$附近有界.
\end{prop}

\begin{thm}
    函数极限若存在则必唯一.
\end{thm}

\begin{prop}[局部性]
    设$f:I\to\mathbb{R},J \subseteq I$,$a$为$J$的内点或端点.若$\lim\limits{x\to a}f(x)=l$,则$\lim\limits_{x\to a}f(x)\vert_J=l$
\end{prop}

\begin{thm}
    设$f:I \to \mathbb{R},a \in I,l \in \mathbb{R}$.则$\lim\limits_{x \to a}=l \Leftrightarrow$
    对于$I$上任意收敛于$a$的序列$(u_n)$,都有$\lim\limits_{n \to \infty}f(u_n)=l$.
\end{thm}

\begin{thm}[函数极限的运算]
    设$\lim\limits_{x \to a}f(x)=l,\lim\limits_{x \to a}g(x)=r$,则有
    \begin{enumerate}
        \item $\lim\limits_{x \to a}f(x)+g(x)=l+r$.
        \item $\lim\limits_{x \to a}f(x)-g(x)=l-r$.
        \item $\lim\limits_{x \to a}f(x)g(x)=lr$.
        \item 若$g(x)\neq0,r\neq0$,则$\lim\limits_{x \to a}\frac{f(x)}{g(x)}=\frac{l}{r}$.
    \end{enumerate}
\end{thm}

\begin{deff}[上下极限]
    设$f:I\to\mathbb{R},a\in I$.
    记$\varlimsup\limits_{x\to a}f(x)=
    \lim\limits_{\delta\to0}\sup\limits_{0<\left\lvert x-a \right\rvert < \delta}f(x)$
    称为$f$在$a$处的上极限.
    记$\varliminf\limits_{x\to a}f(x)=
    \lim\limits_{\delta\to0}\sup\limits_{0<\left\lvert x-a \right\rvert < \delta}f(x)$
    称为$f$在$a$处的下极限.
\end{deff}

\begin{thm}
    $$\lim\limits_{x\to a}f(x)=l \Leftrightarrow
    \varlimsup\limits_{x\to a}f(x)=\varliminf\limits_{x\to a}f(x)=l$$
\end{thm}

\begin{prop}[序关系]
    设$f:I\to\mathbb{R},a\in I,l\in\mathbb{R}$,则有:
    \begin{enumerate}
        \item 若$\lim\limits_{x\to a}f(x) \geqslant l$,则存在$\delta>0$,当$0<\left\lvert x-a \right\rvert \leqslant \delta$时,有$f(x) \geqslant l$.
        \item 若存在$\delta>0$,当$0<\left\lvert x-a \right\rvert \leqslant \delta$时,有$f(x) \geqslant l$,且$\lim\limits_{x\to a}f(x)$存在,则$\lim\limits_{x\to a}f(x) \geqslant l$.
    \end{enumerate}
\end{prop}

\begin{prop}
    若$\lim\limits_{x \to a}f(x)=0$,$g$在$a$附近有界,则$\lim\limits_{x \to a}f(x)g(x)=0$.
\end{prop}

\begin{prop}
    若$\lim\limits_{x \to a}f(x)=+\infty$,则$\lim\limits_{x \to a}\frac{1}{f(x)}=0$.
\end{prop}

\section{函数连续}

\begin{deff}
    称$f$在$a$处连续,若$\lim\limits_{x \to a} f(x) = f(a)$.
\end{deff}

\begin{prop}
    $f$在$a$处连续的充要条件为:对任意收敛于$a$的序列$(u_n)$,
    都有$\lim\limits_{n \to \infty}f(u_n)=f(a)$.
\end{prop}

\begin{prop}
    $f$在$a$处连续的充要条件是:
    对于$f(a)$的开任意邻域$\tilde{O}$,存在$a$的一个开邻域$O$,使得$O\subseteq f^{-1}(\tilde{O})$.
\end{prop}

\begin{prop}
    若$f$在$a$处连续,则$f$在$a$附近有界.
\end{prop}

\begin{thm}[介值定理]
    设$f:I\to\mathbb{R},[a,b]\subseteq I$.若$f(a)f(b)\leqslant0$,
    则存在$c\in[a,b]$,使得$f(c)=0$.
\end{thm}

\begin{prop}
    区间在连续函数下的像为区间.
\end{prop}

\begin{prop}
    设$f:I\to\mathbb{R}$连续且为单射,则有:
    \begin{enumerate}
        \item $g:I\to f(I)$为双射.
        \item $f$严格单调.
        \item $f^{-1}$存在且连续,严格单调.
    \end{enumerate}
\end{prop}

\begin{thm}
    设$I$为闭区间,$f:I\to\mathbb{R}$连续,
    则$f$在$I$上有上下确界,且能取到.
\end{thm}

\begin{deff}[左右连续]
    称$f$在$a$处左连续,若$\lim\limits_{x \to a^-} f(x) = f(a)$.

    称$f$在$a$处右连续,若$\lim\limits_{x \to a^+} f(x) = f(a)$.
\end{deff}

\begin{deff}
    设$I$为区间,$a$为$I$内点,$f:I\setminus\{a\}\to\mathbb{R}$.
    称$\tilde{f}:I\to\mathbb{R}$为$f$的连续延拓,若$\tilde{f}$满足
    $\tilde{f}\vert_{I\setminus\{a\}}=f$且$\tilde{f}(x)$在$a$处连续.
\end{deff}

\begin{thm}
    设$I$为区间,$a$为$I$内点,$f:I\setminus\{a\}\to\mathbb{R}$.
    若$f$单调增,则$\lim\limits_{x\to a^+}f(x)$存在.
\end{thm}

\begin{deff}[Lipschitz函数]
    设$f:I\to\mathbb{R},k>0$.称$f$为$k$-Lipschitz函数,若对于任意$x,y\in I$,
    都有$\left\lvert f(x)-f(y) \right\rvert \leqslant k \left\lvert x-y \right\rvert$.
\end{deff}

\begin{deff}[一致连续]
    设$I$为区间,$f:I\to \mathbb{R}$.称$f$在$I$上一致连续,
    若对于任意$\varepsilon>0$,存在$\delta>0$,
    当$x,y\in I$满足$\left\lvert x-y \right\rvert \leqslant \delta$时,
    有$\left\lvert f(x)-f(y) \right\rvert \leqslant \varepsilon$.
\end{deff}

\begin{thm}
    闭区间上的连续函数一定一致连续.
\end{thm}

\begin{pf}
    设$I$为闭区间,$f:I\to\mathbb{R}$连续.若$f$在$I$上不一致连续,即
    $$\exists\varepsilon_0>0,\forall\delta>0,\exists x,y \in I,
    \left\lvert x-y \right\rvert \leqslant \delta,
    \left\lvert f(x)-f(y) \right\rvert > \varepsilon_0$$

    对于$n\in\mathbb{N}^*$,取$\delta=\frac{1}{n}$,则存在$x=x_n,y=y_n$满足上述条件,
    易知序列$(x_n),(y_n)$有界.

    因此存在$(x_n),(y_n)$的收敛子列,不妨记为$(x_{\varphi(n)}),(y_{\varphi(n)})$.

    设$\lim\limits_{n\to\infty}x_{\varphi(n)}=x_0,\lim\limits_{n\to\infty}y_{\varphi(n)}=y_0$.

    则$\left\lvert x_{\varphi(n)}-y_{\varphi(n)} \right\rvert \leqslant \dfrac{1}{\varphi(n)} \to 0$.

    因此$x_0=y_0$,而$0<\varepsilon_0 \leqslant \left\lvert f(x_{\varphi(n)})-f(y_{\varphi(n)})
    \right\rvert \to 0$,矛盾!
    
    故$f$在$I$上一致连续.
\end{pf}