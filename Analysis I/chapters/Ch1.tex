\chapter{集合与映射}

\begin{deff} 
    设$S,T$为两集合,称$S$与$T$\textbf{等势},若存在双射$f:S \to T$.记为$\left\lvert S \right\rvert = \left\lvert T \right\rvert$.
\end{deff}

\begin{deff}
    设$S$为集合,称$S$为\textbf{有限集},若$S$为空集或存在$n\in\mathbb{N}^*$,使得${1,2,\cdots,n}$与$S$等势.记为$\left\lvert S \right\rvert = n$.
\end{deff}

\begin{exer}
$\left\lvert (0,1) \right\rvert = \left\lvert [0,1] \right\rvert$.
\end{exer}

\begin{thm}
    设$S,T$为两集合,若存在单射$f:S \to T,g:T \to S$,则$\left\lvert S \right\rvert = \left\lvert T \right\rvert$.
\end{thm}

\begin{pf}
    先证明一个引理.

    \begin{lem}
        若$B \subseteq A$,存在单射$f:A \to B$,则$\left\lvert A \right\rvert = \left\lvert B \right\rvert$.
    \end{lem}

    \begin{pf}
        记$Y=A \setminus B,X=Y \cup f(Y) \cup f(f(Y)) \cup \cdots$.

        则$X=Y \cup f(X),A=Y \cup B$.

        因此$A \setminus X = B \setminus f(X)$.

        进而$A \setminus X \subseteq B$.

        而$A=X \cup (A \setminus X),B=f(X) \cup (B \setminus f(X))$.

        记映射$F:A \to B,F(X)=\begin{cases} f(x), & x \in X \\ x, & x \in A \setminus X \end{cases}$
        
        则$F$为双射,故$\left\lvert A \right\rvert = \left\lvert B \right\rvert$.
    \end{pf}

    由题意知$f(S) \subseteq T$,存在单射$f \circ g:T \to S \to f(S)$.

    由引理知$\left\lvert T \right\rvert = \left\lvert f(S) \right\rvert$.

    显然$\left\lvert S \right\rvert = \left\lvert f(S) \right\rvert$.

    故$\left\lvert S \right\rvert = \left\lvert T \right\rvert$.
\end{pf}

\begin{deff}
    设$S$为集合,记$2^S$或$\mathcal{P}(S)$为$S$的子集的全体,称为$S$的\textbf{幂集}.
\end{deff}

\begin{thm}
    $\left\lvert S \right\rvert \neq \left\lvert \mathcal{P}(S) \right\rvert$.
\end{thm}

\begin{pf}
    若存在$f:S \to \mathcal{P}(S)$为双射.
    设$A=\{ x \in S \mid x \notin f(x) \} \in \mathcal{P}(S)$.
    则存在$a \in S$使得$f(a)=A$.
    若$a \in A$,则$a \notin f(a)=A$,矛盾.
    若$a \notin A$,则$a \in f(a)=A$,矛盾.
    故不存在这样的双射$f$.即$\left\lvert S \right\rvert \neq \left\lvert \mathcal{P}(S) \right\rvert$.
\end{pf}