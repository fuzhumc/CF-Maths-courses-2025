\chapter{渐进分析}

\begin{deff}
    设$f,\varphi:I\to\mathbb{R},a\in I$.称$f$在$a$附近被$\varphi$\textbf{控制},若存在$u:I\to\mathbb{R}$在$a$附近有且$f=u\varphi$.此时记$f=O(\varphi)$.
\end{deff}

\begin{deff}
    设$f,\varphi:I\to\mathbb{R},a\in i$.称$f$在$a$附近相对于$\varphi$\textbf{可忽略},若存在$\varepsilon:I\to\mathbb{R},\lim\limits_{x\to a}\varepsilon(x)=0$且$f=\varepsilon\varphi$.记为$f=o(\varphi)$.
\end{deff}

\begin{prop}
    $f$在$a$附近有界$\iff f=O(1)$;$\lim\limits_{x\to a}f(x)=0\iff f=o(1)$.
\end{prop}

\begin{prop}
    \begin{enumerate}
        \item 若$f=o(\varphi)$,则$f=O(\varphi)$.
        \item 若$f_1=O(\varphi_1),f_2=O(\varphi_2)$,则$f_1f_2=O(\varphi_1\varphi_2)$.
        \item 若$f_1=o(\varphi),f_2=o(\varphi)$,则$f_1+f_2=o(\varphi)$.
        \item 若$f=o(\varphi_1),\varphi_1=o(\varphi_2)$,则$f=o(\varphi_2)$.
    \end{enumerate}
\end{prop}

\begin{deff}
    设$f,g:I\to\mathbb{R},a\in I$,称$f$在$a$附近与$g$\textbf{等价},若存在$u:I\to\mathbb{R},\lim\limits_{x\to a}u(x)=1$且$f=ug$.记为$f\sim g$.
\end{deff}

\begin{prop}
    $f\sim g\iff f-g=o(g)$.
\end{prop}

\begin{prop}
    设$f:I\to\mathbb{R},a\in I$,$f$在$a$处可导,则$f(x)\sim f(a)+f'(a)(x-a)$.
\end{prop}

\begin{rqq}
    若$g$在$a$附近非0,则$f\sim g\iff\lim\limits_{x\to a}\dfrac{f(x)}{g(x)}=1$.
\end{rqq}

\begin{prop}
    设$f\sim g$,则有
    \begin{enumerate}
        \item 若$g$在$a$附近非负,则$f$在$a$附近非负.
        \item 若$g$在$a$附近非0,则$f$在$a$附近非0.
    \end{enumerate}
\end{prop}

\begin{prop}
    设$f_1\sim g_1,f_2\sim g_2$,则有
    \begin{enumerate}
        \item $f_1f_2\sim g_1g_2$.
        \item 若$f_2,g_2$在$a$附近非0,则$\dfrac{f_1}{f_2}\sim\dfrac{g_1}{g_2}$.
    \end{enumerate}
\end{prop}

\begin{prop}
    设$f\sim g,\alpha\in\mathbb{R}$,则$f^{\alpha}\sim g^{\alpha}$.
\end{prop}

\begin{exer}
    设$f,g:I\to\mathbb{R}$,则$e^f\sim e^g\iff f-g\to0$.
\end{exer}

\begin{prop}
    设$f:I\to\mathbb{R},a\in I,f\sim g,b=f(a)=g(a)\in U,\varphi:U\to I,\lim\limits_{x\to b}\varphi(x)=a$,则$f\circ\varphi\sim g\circ\varphi$.
\end{prop}

\section{极限展开}

\begin{deff}
    设$0\in I,f:I\to\mathbb{R}$.称$f$在$0$处有$n$阶展开,若存在$a_0,a_1,a_2,\cdots,a_n\in\mathbb{R}$,使得$f(x)=\sum\limits_{k=0}^na_kx^k+o(x^n)$.
\end{deff}

\begin{deff}
    设$x_0\in I,f:I\to\mathbb{R}$.称$f$在$x_0$处有$n$阶展开,若存在$a_0,a_1,a_2,\cdots,a_n\in\mathbb{R}$,使得$f(x)=\sum\limits_{k=0}^na_k(x-x_0)^k+o((x-x_0)^n)$.
\end{deff}

\begin{prop}
    函数在一点处的$n$阶展开唯一.
\end{prop}

\begin{deff}
    设$f$在$0$处的$n$阶展开为$f(x)=\sum\limits_{k=0}^na_kx^k+o(x^n)$.称多项式$P(x)=\sum\limits_{k=0}^na_kx^k$为其\textbf{正则部分}.
\end{deff}

\begin{prop}
    设$P$为$f$在0处的$n$阶展开的正则部分.若$f$为偶函数,则$P$只有偶次项;若$f$为奇函数,则$P$只有寄次项.
\end{prop}

\begin{prop}
    $f$在0处有0阶展开当且仅当$f(x)=l+o(1)$,其中$l\in\mathbb{R}$.
\end{prop}

\begin{prop}
    $f$在0处可导当且仅当$f$在0处有$1$阶展开.
\end{prop}

\begin{prop}
    设$f:I\to\mathbb{R}$,$F$为其原函数,$f$在0处有$n$阶展开$f(x)=\sum\limits_{k=0}^na_kx^k+o(x^n)$,则$F(x)-F(0)=\sum\limits_{x=0}^n\dfrac{a_k}{k+1}x^{k+1}+o(x^{n+1})$.
\end{prop}

\begin{deff}
    设$f$在0处有$n$阶展开$f(x)=\sum\limits_{k=0}^na_kx^k+o(x^n)$,记$p=\min\{p\in[0,n]\mid a_p\neq0\}$,称$f(x)=x^p(a_p+a_{p+1}x+\cdots+a_nx^{n-p}+o(x^{n-p}))$为$f$在0处的$n$阶展开的正规形式.
\end{deff}

\begin{prop}
    设$f,g$在0处有$n$阶展开,则$f+g,fg$在0处也有$n$阶展开.
\end{prop}

\begin{prop}
    设$f$在0处有$n$阶展开,$\lim\limits_{x\to0}u(x)=0$,则$\dfrac{1}{1-u(x)}$在0处也有$n$阶展开,且其正则部分与$\sum\limits_{k=0}^nu(x)^k$在0处的$n$阶展开的正则部分相同.
\end{prop}