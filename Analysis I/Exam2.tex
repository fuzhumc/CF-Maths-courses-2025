% Options for packages loaded elsewhere
\PassOptionsToPackage{unicode}{hyperref}
\PassOptionsToPackage{hyphens}{url}
\documentclass[
  a4paper,
]{article}

\usepackage{xcolor}
\usepackage{amsmath,amssymb}

\usepackage{ctex}
\usepackage{xeCJK}
\setCJKmainfont{SimSun}[BoldFont=SimHei,ItalicFont=KaiTi]


\setcounter{secnumdepth}{-\maxdimen} % remove section numbering

\usepackage{bookmark}
\IfFileExists{xurl.sty}{\usepackage{xurl}}{} % add URL line breaks if available
\urlstyle{same}

\usepackage{iftex}
\defaultfontfeatures{Scale=MatchLowercase}
\defaultfontfeatures[\rmfamily]{Ligatures=TeX,Scale=1}
\usepackage{lmodern}

\makeatletter
\@ifundefined{KOMAClassName}{% if non-KOMA class
  \IfFileExists{parskip.sty}{%
    \usepackage{parskip}
  }{% else
    \setlength{\parindent}{0pt}
    \setlength{\parskip}{6pt plus 2pt minus 1pt}}
}{% if KOMA class
  \KOMAoptions{parskip=half}}
\makeatother

\setlength{\emergencystretch}{3em} % prevent overfull lines
\providecommand{\tightlist}{\setlength{\itemsep}{0pt}\setlength{\parskip}{0pt}}

\usepackage{bookmark}
\IfFileExists{xurl.sty}{\usepackage{xurl}}{} % add URL line breaks if available
\urlstyle{same}
\hypersetup{hidelinks,pdfcreator={cjx}}


\usepackage{titling}
\title{Exam2}
\author{}
\date{2026年1月21日}

\usepackage{geometry}
\geometry{a4paper,left=1.27cm,right=1.27cm,top=1.7cm,bottom=1.7cm}

\usepackage{enumitem}
\usepackage{etoolbox}
\AtBeginEnvironment{enumerate}{
  \apptocmd{\tightlist}{\def\labelenumii{(\arabic{enumii})}}{}{}
}


\usepackage{fancyhdr}

\begin{document}
\maketitle

\pagestyle{fancy}
\fancyhf{}

\lhead{2026年1月21日}
\chead{Exam2}

\renewcommand{\headrulewidth}{1pt}
\rfoot{\thepage}

\begin{enumerate}
\def\labelenumi{\arabic{enumi}.}
\tightlist
\item
  (10分)判断极限是否存在,存在的话计算极限.

  \begin{enumerate}
  \def\labelenumii{\arabic{enumii}.}
  \tightlist
  \item
    \(\lim\limits_{x \to +\infty} \left( \sin^2 \dfrac{1}{x} + \cos \dfrac{1}{x} \right)^{x^2}\).
  \item
    \(\lim\limits_{x \to 0} \dfrac{(1 + \ln (1 + x))^{\frac{1}{\tan x}} - e(1 - x)}{\tan^2 x}\).
  \end{enumerate}
\item
  (20分)

  \begin{enumerate}
  \def\labelenumii{\arabic{enumii}.}
  \tightlist
  \item
    计算积分\(\int_0^1 \dfrac{\mathrm{d} t}{\sqrt{1 + t^2} + \sqrt{1 - t^2}}\).
  \item
    设\(f_n(t) = \dfrac{n}{n^2 t^2 + 1}\)且函数\(g\)为紧区间\([a, b]\)上的连续函数.求极限\(\lim\limits_{n \to \infty} \dfrac{1}{\pi} \int_a^b f_n(t)g(t) \mathrm{d} t\).
  \end{enumerate}
\item
  (20分)判断研究下列级数的敛散性并说明理由.

  \begin{enumerate}
  \def\labelenumii{\arabic{enumii}.}
  \tightlist
  \item
    设\(x > 0\),通项为\(u_n = \sqrt{n!} \sin x \cdot \sin \dfrac{x}{\sqrt{2}} \cdot \sin \dfrac{x}{\sqrt{3}} \cdots \sin \dfrac{x}{\sqrt{n}}\)的级数.
  \item
    通项为\(v_n = e^{an^2} \left( 1 - \dfrac{a}{n} \right)^{n^3}\)的级数.
  \end{enumerate}
\item
  (15分)设\((f_n)\)是一列从区间\([-1, 1]\)到实数\(\mathbb{R}\)的连续函数,满足以下条件:对于所有\(1 > a > 0\),有\(\int_0^1 f_n(t) \mathrm{d}t = 1\),且\(\lim\limits_{n \to \infty} \sup\limits_{\lvert t \rvert > a} \lvert f_n(t) \rvert = 0\),且\(g\)是一个连续函数,定义在区间\([-1, 1]\)上.

  \begin{enumerate}
  \def\labelenumii{\arabic{enumii}.}
  \tightlist
  \item
    若对所有\(n\)都有\(f_n \geqslant 0\),证明:\(\lim\limits_{n \to \infty} \int_{-1}^1 g(t) f_n(t) \mathrm{d} t = g(0)\).
  \item
    若没有\(f_n \geqslant 0\)的条件,(1)中的结论是否成立?请证明或给出反例.
  \end{enumerate}
\item
  (15分)设\[u_n = \dfrac{n}{2} - \sum\limits_{k = 1}^n \dfrac{n^2}{(n + k)^2}\]求数列\((u_n)\)的极限.
\item
  (10分)设\(f\)是\(\mathbb{R}\)上的二阶连续可微函数,且满足\(f(x) + f''(x) > 0, \forall x \in \mathbb{R}\).证明:\(f(x) + f(x + \pi) > 0, \forall x \in \mathbb{R}\).
\item
  (10分)考虑微分方程\((E)\):\[y'' + p(x + y' + y''')y = 0\]其中\(p: \mathbb{R} \to \mathbb{R}\)是一个连续函数,这里\(p(x + y' + y''')\)指的是\(p\)在\(x + y'(x) + y'''(x)\)处的值.

  \begin{enumerate}
  \def\labelenumii{\arabic{enumii}.}
  \tightlist
  \item
    设\(y\)是\((E)\)在\(\mathbb{R}\)上的一个非零解,\(y\)的零点集合\(X\)是否没有聚点?证明或给出反例.
  \item
    设\(X^+ = \{ x \in X \mid x \geqslant 0 \}\),证明可以将\(X^+\)中元素按递增顺序排成一个子序列.
  \end{enumerate}
\end{enumerate}

\end{document}
