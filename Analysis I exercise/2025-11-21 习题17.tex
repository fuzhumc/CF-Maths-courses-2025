% Options for packages loaded elsewhere
\PassOptionsToPackage{unicode}{hyperref}
\PassOptionsToPackage{hyphens}{url}
\documentclass[
  a4paper,
]{article}

\usepackage{xcolor}
\usepackage{amsmath,amssymb}

\usepackage{ctex}
\usepackage{xeCJK}
\setCJKmainfont{SimSun}[BoldFont=SimHei,ItalicFont=KaiTi]


\setcounter{secnumdepth}{-\maxdimen} % remove section numbering

\usepackage{bookmark}
\IfFileExists{xurl.sty}{\usepackage{xurl}}{} % add URL line breaks if available
\urlstyle{same}

\usepackage{iftex}
\defaultfontfeatures{Scale=MatchLowercase}
\defaultfontfeatures[\rmfamily]{Ligatures=TeX,Scale=1}
\usepackage{lmodern}

\makeatletter
\@ifundefined{KOMAClassName}{% if non-KOMA class
  \IfFileExists{parskip.sty}{%
    \usepackage{parskip}
  }{% else
    \setlength{\parindent}{0pt}
    \setlength{\parskip}{6pt plus 2pt minus 1pt}}
}{% if KOMA class
  \KOMAoptions{parskip=half}}
\makeatother

\setlength{\emergencystretch}{3em} % prevent overfull lines
\providecommand{\tightlist}{\setlength{\itemsep}{0pt}\setlength{\parskip}{0pt}}

\usepackage{bookmark}
\IfFileExists{xurl.sty}{\usepackage{xurl}}{} % add URL line breaks if available
\urlstyle{same}
\hypersetup{hidelinks,pdfcreator={cjx}}


\usepackage{titling}
\title{习题17}
\author{}
\date{2025年11月21日}

\usepackage{geometry}
\geometry{a4paper,left=1.27cm,right=1.27cm,top=1.7cm,bottom=1.7cm}

\usepackage{enumitem}
\usepackage{etoolbox}
\AtBeginEnvironment{enumerate}{
  \apptocmd{\tightlist}{\def\labelenumii{(\arabic{enumii})}}{}{}
}


\usepackage{fancyhdr}

\begin{document}
\maketitle

\pagestyle{fancy}
\fancyhf{}

\lhead{2025年11月21日}
\chead{习题17}

\renewcommand{\headrulewidth}{1pt}
\rfoot{\thepage}

\textbf{定义}
称\(f \sim g(x \to a)\),若存在函数\(h:X\to\mathbb{R}\)以及\(a\)的一个邻域\(U\),使得\(f(x)=g(x)h(x),x\in X \cap U\)且\(\lim\limits_{x\to a}h(x)=1\).
1. 证明:\(e^f \sim e^g\)当且仅当\(\lim\limits_{x\to a}(f(x)-g(x))=0\).
2. 设\(f_1 \sim g_1,f_2 \sim g_2(x\to a)\).证明: 1.
若\(g_2=o(g_1)\),则\(f_1+f_2 \sim g_1(x\to a)\). 2.
若存在\(\alpha \neq -1\)满足\(g_2 \sim \alpha g_1\),则\(f_1+f_2 \sim (1+\alpha)g_1\).
3. 若\(g_2 \sim -g_1\),则\(f_1+f_2=o(g_1)\). 4.
\(\cosh x \sim \frac{e^x}{2},\cosh^{-1} x \sim \ln x\). 3.
设函数\[f(x)=\begin{cases} 1+x+\frac{x^2}{3}+x^3\sim\frac{1}{2}, & x\neq0 \\ 1, & x=0 \end{cases}\]证明:\(f\)在\(x=0\)处有二阶限定展开,但\(f\)在\(x=0\)处二阶不可导.
4.
设函数\(f\in\mathcal{C}^{\infty}\)且对任意\(n\),\(f(x)=o(x^n)\).问:\(f\)在\(0\)附近是否恒为\(0\)?
5.
设\(f\)满足\(f(x)=o(x^n),\forall n\),\(f\)一定是\(\mathcal{C}^{\infty}\)吗?
6. 我们称\(f,g\)在\(x=a\)处\(n\)阶相等,若\(f-g=o((x-a)^n)\). 1.
证明:若\(P,Q\)是关于\((x-a)\)的次数不大于\(n\)的多项式,且\(P,Q\)在\(a\)处\(n\)阶相等,则\(P=Q\).
2.
若\(f\)在\(a\)处\(n\)次可导,且\(P\)是次数不大于\(n\)的关于\((x-a)\)的多项式,且与\(f\)在\(a\)处\(n\)阶相等,则\[P(x)=f(a)+f’(a)(x-a)+\cdots+\frac{f^{(n)}(a)}{n!}(x-a)^n\]
7.
设\(f\in\mathcal{C}^2([a,b],\mathbb{R})\)且\(f’(a)=f’(b)=0\).证明:存在\(\xi\in(a,b)\)使得\[\left\lvert f’’(\xi) \right\rvert \geqslant \frac{4}{(b-a)^2} \left\lvert f(b)-f(a) \right\rvert\]
8.
设\(f\in\mathcal{C}^2([0,2],\mathbb{R})\)且\(\left\lvert f(x) \right\rvert \leqslant 1,\left\lvert f’’(x) \right\rvert \leqslant 1\).证明\(\left\lvert f’(x) \right\rvert \leqslant 2\).
9. 1.
证明:若\(f’(x_0)\)存在,则\[f’’(x_0)=\lim\limits_{h\to0}\frac{f(x_0+h)+f(x_0-h)-2f(x_0)}{h^2}\]
2.
设函数\[f(x)=\begin{cases} x^2, & x\geqslant 0 \\ -x^2, & x\leqslant 0 \end{cases}\]证明:\(f’’(x_0)=\lim\limits_{h\to0}\dfrac{f(x_0+h)+f(x_0-h)-2f(x_0)}{h^2}\)存在但\(f’’(0)\)不存在.
10. 设\(f\in\mathcal{C}^2\)有上界,证明:\(f’\)是常值函数. 11.
设\(f\in\mathcal{C}^2\)且\(f,f’’\)有上界,证明:\(f’\)有界且满足\(\max \left\lvert f’(x) \right\rvert \leqslant \max \left\lvert f(x) \right\rvert \max \left\lvert f’’(x) \right\rvert\).

\end{document}
