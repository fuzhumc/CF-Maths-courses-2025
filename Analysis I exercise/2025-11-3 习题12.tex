% Options for packages loaded elsewhere
\PassOptionsToPackage{unicode}{hyperref}
\PassOptionsToPackage{hyphens}{url}
\documentclass[
  a4paper,
]{article}

\usepackage{xcolor}
\usepackage{amsmath,amssymb}

\usepackage{ctex}
\usepackage{xeCJK}
\setCJKmainfont{SimSun}[BoldFont=SimHei,ItalicFont=KaiTi]


\setcounter{secnumdepth}{-\maxdimen} % remove section numbering

\usepackage{bookmark}
\IfFileExists{xurl.sty}{\usepackage{xurl}}{} % add URL line breaks if available
\urlstyle{same}

\usepackage{iftex}
\defaultfontfeatures{Scale=MatchLowercase}
\defaultfontfeatures[\rmfamily]{Ligatures=TeX,Scale=1}
\usepackage{lmodern}

\makeatletter
\@ifundefined{KOMAClassName}{% if non-KOMA class
  \IfFileExists{parskip.sty}{%
    \usepackage{parskip}
  }{% else
    \setlength{\parindent}{0pt}
    \setlength{\parskip}{6pt plus 2pt minus 1pt}}
}{% if KOMA class
  \KOMAoptions{parskip=half}}
\makeatother

\setlength{\emergencystretch}{3em} % prevent overfull lines
\providecommand{\tightlist}{\setlength{\itemsep}{0pt}\setlength{\parskip}{0pt}}

\usepackage{bookmark}
\IfFileExists{xurl.sty}{\usepackage{xurl}}{} % add URL line breaks if available
\urlstyle{same}
\hypersetup{hidelinks,pdfcreator={cjx}}


\usepackage{titling}
\title{习题12}
\author{}
\date{2025年11月3日}

\usepackage{geometry}
\geometry{a4paper,left=1.27cm,right=1.27cm,top=1.7cm,bottom=1.7cm}

\usepackage{enumitem}
\usepackage{etoolbox}
\AtBeginEnvironment{enumerate}{
  \apptocmd{\tightlist}{\def\labelenumii{(\arabic{enumii})}}{}{}
}


\usepackage{fancyhdr}

\begin{document}
\maketitle

\pagestyle{fancy}
\fancyhf{}

\lhead{2025年11月3日}
\chead{习题12}

\renewcommand{\headrulewidth}{1pt}
\rfoot{\thepage}

本周我们开始学习积分,在接触严格的理论证明前,我们先建立起对于积分的感性认识.
(非常不严格的)\textbf{定义}
对于函数\(f:[a,b] \to \mathbb{R}\),如果\({} f(x)\geqslant0 {}\)我们称\(f\)的图像与\(x\)轴所围成的面积为\(f\)从\(a\)到\(b\)的积分,记为\(\int_a^b f(x)\mathrm{d}x\).如果\(f\)是变号的,则记\(\int_a^b f(x)\mathrm{d}x=\int_a^b f^+(x)\mathrm{d}x-\int_a^b f^-(x)\mathrm{d}x\).
1. 依据上面的定义,计算下面的积分: 1. \(\int_{-2}^2 2x+3\mathrm{d}x\). 2.
\(\int_{-\frac{\sqrt{2}}{2}}^1 \sqrt{1-x^2}\mathrm{d}x\). 3.
\(\int_0^{\pi} \cos x \mathrm{d}x\). 2.
对于不规则的区域我们显然不能用上面的方法计算``积分''.事实上,我们从未严格定义过``面积''这一概念.但目前不妨仍秉持着对``面积''朴素的认知,来计算一下\(\int_0^1 x^2\mathrm{d}x\).
先把\([0,1]\)区间平均分成\(n\)等份,记分割点为\(0=x_0<x_1<x_2<\cdots<x_{n-1}<x_n=1\).
1.
对于区间\([x_j,x_{j+1}]\),作一个矩形,宽为区间长度\(\frac{1}{n}\),高为函数在\(x_j\)处的取值,即\(x_j^2\).试计算这\(n\)个矩形的面积之和(记为\(L_n\)).
2.
在上一问中稍作改变,把每一个矩形的高度改为函数在\(x_{j+1}\)处的取值,即\(x_{j+1}^2\).试再计算这\(n\)个矩形的面积之和(记为\(R_n\)).
3.
计算\(\lim\limits_{n \to \infty}L_n\)与\(\lim\limits_{n \to \infty}R_n\),你发现了什么?这是巧合吗?

\begin{verbatim}
我们把这个极限称为$f(x)=x^2$再$[0,1]$上的积分.
我们把这个积分定义为$f(x)=x^2$与$x$轴所围成的面积!
试想一下,这个定义有没有什么问题?
\end{verbatim}

\begin{enumerate}
\def\labelenumi{\arabic{enumi}.}
\setcounter{enumi}{2}
\tightlist
\item
  利用上面的办法,试着计算一下下列函数的积分:

  \begin{enumerate}
  \def\labelenumii{\arabic{enumii}.}
  \tightlist
  \item
    \(\int_0^1 e^x\mathrm{d}x\).
  \item
    \(\int_1^2 \frac{1}{x}\mathrm{d}x\).
  \item
    \({} \int_{-1}^1 x^2-x+1 \mathrm{d}x {}\).
  \item
    \(\int_0^{\pi} \sin x \mathrm{d}x\).
  \end{enumerate}
\item
  显然我们不能每次计算积分都把上述操作做一遍,重新考虑\(\int_0^1 x^2 \mathrm{d}x\).

  \begin{enumerate}
  \def\labelenumii{\arabic{enumii}.}
  \tightlist
  \item
    问自己一个问题,什么样的函数\(F\),满足\(F’(x)=x^2\).
  \item
    对这样的\(F\),计算一下\(F(1)-F(0)\).你发现了什么?这是巧合吗?
  \end{enumerate}
\item
  对于\(f\geqslant0\),验证下面三件事:

  \begin{enumerate}
  \def\labelenumii{\arabic{enumii}.}
  \tightlist
  \item
    \(\int_a^a f(x)\mathrm{d}x=0\).
  \item
    对任意区间\([a,b]\),\(\int_a^b f(x) \mathrm{d}x \geqslant 0\).
  \item
    考虑一串区间\(a_0 \leqslant a_1 \leqslant a_2 \leqslant \cdots \leqslant a_n\),有\(\sum\limits_{j=0}^{n-1} \int_{a_j}^{a_{j+1}} f(x)\mathrm{d}x=\int_{a_0}^{a_n}f(x)\mathrm{d}x\).
  \item
    令\(n \to \infty\),是否仍然成立?
  \item
    这是在说什么事情?
  \end{enumerate}
\item
  若\(f:\mathbb{R}\to\mathbb{R}\)连续,利用微分和积分的定义(目前还不太严格)证明:\(\frac{\mathrm{d}}{\mathrm{d}x}\int_0^xf(t)\mathrm{d}t=f(x)\).
\end{enumerate}

\end{document}
