% Options for packages loaded elsewhere
\PassOptionsToPackage{unicode}{hyperref}
\PassOptionsToPackage{hyphens}{url}
\documentclass[
  a4paper,
]{article}

\usepackage{xcolor}
\usepackage{amsmath,amssymb}

\usepackage{ctex}
\usepackage{xeCJK}
\setCJKmainfont{SimSun}[BoldFont=SimHei,ItalicFont=KaiTi]


\setcounter{secnumdepth}{-\maxdimen} % remove section numbering

\usepackage{bookmark}
\IfFileExists{xurl.sty}{\usepackage{xurl}}{} % add URL line breaks if available
\urlstyle{same}

\usepackage{iftex}
\defaultfontfeatures{Scale=MatchLowercase}
\defaultfontfeatures[\rmfamily]{Ligatures=TeX,Scale=1}
\usepackage{lmodern}

\makeatletter
\@ifundefined{KOMAClassName}{% if non-KOMA class
  \IfFileExists{parskip.sty}{%
    \usepackage{parskip}
  }{% else
    \setlength{\parindent}{0pt}
    \setlength{\parskip}{6pt plus 2pt minus 1pt}}
}{% if KOMA class
  \KOMAoptions{parskip=half}}
\makeatother

\setlength{\emergencystretch}{3em} % prevent overfull lines
\providecommand{\tightlist}{\setlength{\itemsep}{0pt}\setlength{\parskip}{0pt}}

\usepackage{bookmark}
\IfFileExists{xurl.sty}{\usepackage{xurl}}{} % add URL line breaks if available
\urlstyle{same}
\hypersetup{hidelinks,pdfcreator={cjx}}


\usepackage{titling}
\title{习题15}
\author{}
\date{2025年11月14日}

\usepackage{geometry}
\geometry{a4paper,left=1.27cm,right=1.27cm,top=1.7cm,bottom=1.7cm}

\usepackage{enumitem}
\usepackage{etoolbox}
\AtBeginEnvironment{enumerate}{
  \apptocmd{\tightlist}{\def\labelenumii{(\arabic{enumii})}}{}{}
}


\usepackage{fancyhdr}

\begin{document}
\maketitle

\pagestyle{fancy}
\fancyhf{}

\lhead{2025年11月14日}
\chead{习题15}

\renewcommand{\headrulewidth}{1pt}
\rfoot{\thepage}

先定义\(\int_a^{b^-}f(x)\mathrm{d}x\)型广义积分:设函数\(f\)在\([a,b)\)上任意有界闭区间\([\alpha,\beta]\)上可积,满足\(\lim\limits_{x\to b^{-}}f(x)=\pm\infty\)(此时称\(b\)为\(f\)的\textbf{奇点}).则定义\textbf{广义积分}\(\int_a^bf(x)\mathrm{d}x=\lim\limits_{\beta\to b^-}\int_a^{\beta}f(x)\mathrm{d}x\)(倘若极限存在).
类似地,我们可以定义\(\int_{a^+}^bf(x)\mathrm{d}x\)型广义积分:\(\int_a^bf(x)\mathrm{d}x=\lim\limits_{\alpha\to a^+}\int_{\alpha}^bf(x)\mathrm{d}x\).

\begin{enumerate}
\def\labelenumi{\arabic{enumi}.}
\tightlist
\item
  利用上述定义,判断下列积分是否有限.若有限,试计算之.

  \begin{enumerate}
  \def\labelenumii{\arabic{enumii}.}
  \tightlist
  \item
    \[\int_0^{+\infty}e^{-x}\mathrm{d}x\]
  \item
    \[\int_0^1\frac{1}{x}\mathrm{d}x\]
  \item
    \[\int_0^1\frac{1}{\sqrt{1-x}}\mathrm{d}x\]
  \item
    \[\int_0^1\ln x\mathrm{d}x\]
  \item
    \[\int_0^{+\infty}e^{-x}\cos x\mathrm{d}x\]
  \end{enumerate}
\end{enumerate}

现在定义\(\int_{a^+}^{b^-}f(x)\mathrm{d}x\)型积分:设\(f\)在\((a,b)\)上任意有界闭区间\([\alpha,\beta]\)上可积,满足\(\lim\limits_{x\to a^+}f(x)=\pm\infty,\lim\limits_{a\to b^-}f(x)=\pm\infty\),且对于某个\(c\in(a,b)\),广义积分\(\int_a^cf(x)\mathrm{d}x\)与\(\int_c^bf(x)\mathrm{d}x\)都存在(要么有限要么无限,不可以振荡!).则定义\textbf{广义积分}\(\int_a^bf(x)\mathrm{d}x=\int_a^cf(x)\mathrm{d}x+\int_c^bf(x)\mathrm{d}x\).

\begin{enumerate}
\def\labelenumi{\arabic{enumi}.}
\setcounter{enumi}{1}
\tightlist
\item
  利用上述定义,判定并计算下列积分:

  \begin{enumerate}
  \def\labelenumii{\arabic{enumii}.}
  \tightlist
  \item
    \[\int_{-1}^1\frac{1}{\sqrt{1-x^2}}\mathrm{d}x\]
  \item
    \[\int_{-\infty}^{+\infty}\frac{1+x}{1+x^2}\mathrm{d}x\]
  \item
    对于上一问中的函数,若考虑\(\lim\limits_{c\to+\infty}(\int_{-c}^0\frac{1+x}{1+x^2}+\int_0^c\frac{1+x}{1+x^2})\),你会得到什么?
  \end{enumerate}
\end{enumerate}

为了区别上题中的(2)(3),我们定义\textbf{主值积分}:\[p.v.\int_{-\infty}^{+\infty}f(x)\mathrm{d}x=\lim\limits_{c\to+\infty}\int_{-c}^cf(x)\mathrm{d}x,p.v.\int_a^bf(x)\mathrm{d}x=\lim\limits_{\varepsilon\to0^+}\int_{a+\varepsilon}^{b-\varepsilon}f(x)\mathrm{d}x\]

\begin{enumerate}
\def\labelenumi{\arabic{enumi}.}
\setcounter{enumi}{2}
\tightlist
\item
  计算下列广义积分:

  \begin{enumerate}
  \def\labelenumii{\arabic{enumii}.}
  \tightlist
  \item
    \[\int_0^{+\infty}\frac{1}{x^4+1}\mathrm{d}x\]
  \item
    \[\int_0^1\frac{1}{x\sqrt{1+x^2}}\mathrm{d}x\]
  \item
    \[p.v.\int_{-\infty}^{+\infty}\sin x\mathrm{d}x\]
  \item
    \[p.v.\int_0^4\frac{1}{x^2-4x+3}\mathrm{d}x\]
  \item
    \[\int_0^{+\infty}\frac{1}{(x^2+1)(x^{\lambda}+1)}\mathrm{d}x,(\lambda>0)\]
  \end{enumerate}
\item
  证明:广义积分\({} \int_1^{+\infty}\dfrac{x\ln(1+\frac{1}{x})}{1+x^2}\mathrm{d}x {}\)收敛.
\item
  考虑广义积分\({} \int_0^{\frac{\pi}{2}}\cos x\ln\tan x\mathrm{d}x {}\)的敛散性.若收敛,试计算之.
\item
  设\(I=\int_0^{\frac{\pi}{2}}\ln\sin x\mathrm{d}x,J=\int_0^{\frac{\pi}{2}}\ln\cos x\mathrm{d}x\).

  \begin{enumerate}
  \def\labelenumii{\arabic{enumii}.}
  \tightlist
  \item
    证明:\(I,J\)都收敛.
  \item
    计算\(I,J\).
  \end{enumerate}
\item
  设\(f\in\mathcal{C}^2([0,+\infty),\mathbb{R})\),满足\({} \int_0^{+\infty}(f(x))^2\mathrm{d}x {}\)与\({} \int_0^{+\infty}(f’’(x))^2\mathrm{d}x {}\)收敛.证明:\(\int_0^{+\infty}(f’(x))^2\mathrm{d}x\)收敛.
\item
  设广义积分\(I=\int_0^{+\infty}\frac{\sin x}{x}\mathrm{d}x\).

  \begin{enumerate}
  \def\labelenumii{\arabic{enumii}.}
  \tightlist
  \item
    定义\(f:[0,\pi]\to\mathbb{R}\)如下:\[f(x)=\begin{cases} \dfrac{1}{x}-\dfrac{1}{2\sin\dfrac{x}{2}}, & x\in(0,\pi] \\ 0, & x=0 \end{cases}\]证明\(f\)是\(\mathcal{C}^1\)的.
  \item
    计算积分\[I_n=\int_0^{\pi}\dfrac{\sin(n+\frac{1}{2})x}{2\sin\dfrac{x}{2}}\mathrm{d}x\]
  \item
    对于\(\varphi\in\mathcal{C}^1([0,1],\mathbb{R})\),证明:\[\lim\limits_{n\to\infty}\int_0^{\pi}\varphi(x)\sin(n+\dfrac{1}{2})x\mathrm{d}x=0\]
  \item
    计算\(I\).
  \item
    利用上一问计算\(\int_0^{+\infty}\dfrac{\sin x\cos x}{x}\mathrm{d}x\).
  \item
    利用上一问计算\(\int_0^{+\infty}\frac{\sin^2x}{x}\mathrm{d}x\).
  \end{enumerate}
\end{enumerate}

\end{document}
