% Options for packages loaded elsewhere
\PassOptionsToPackage{unicode}{hyperref}
\PassOptionsToPackage{hyphens}{url}
\documentclass[
  a4paper,
]{article}

\usepackage{xcolor}
\usepackage{amsmath,amssymb}

\usepackage{ctex}
\usepackage{xeCJK}
\setCJKmainfont{SimSun}[BoldFont=SimHei,ItalicFont=KaiTi]


\setcounter{secnumdepth}{-\maxdimen} % remove section numbering

\usepackage{bookmark}
\IfFileExists{xurl.sty}{\usepackage{xurl}}{} % add URL line breaks if available
\urlstyle{same}

\usepackage{iftex}
\defaultfontfeatures{Scale=MatchLowercase}
\defaultfontfeatures[\rmfamily]{Ligatures=TeX,Scale=1}
\usepackage{lmodern}

\makeatletter
\@ifundefined{KOMAClassName}{% if non-KOMA class
  \IfFileExists{parskip.sty}{%
    \usepackage{parskip}
  }{% else
    \setlength{\parindent}{0pt}
    \setlength{\parskip}{6pt plus 2pt minus 1pt}}
}{% if KOMA class
  \KOMAoptions{parskip=half}}
\makeatother

\setlength{\emergencystretch}{3em} % prevent overfull lines
\providecommand{\tightlist}{\setlength{\itemsep}{0pt}\setlength{\parskip}{0pt}}

\usepackage{bookmark}
\IfFileExists{xurl.sty}{\usepackage{xurl}}{} % add URL line breaks if available
\urlstyle{same}
\hypersetup{hidelinks,pdfcreator={cjx}}


\usepackage{titling}
\title{习题14}
\author{}
\date{2025年11月10日}

\usepackage{geometry}
\geometry{a4paper,left=1.27cm,right=1.27cm,top=1.7cm,bottom=1.7cm}

\usepackage{enumitem}
\usepackage{etoolbox}
\AtBeginEnvironment{enumerate}{
  \apptocmd{\tightlist}{\def\labelenumii{(\arabic{enumii})}}{}{}
}


\usepackage{fancyhdr}

\begin{document}
\maketitle

\pagestyle{fancy}
\fancyhf{}

\lhead{2025年11月10日}
\chead{习题14}

\renewcommand{\headrulewidth}{1pt}
\rfoot{\thepage}

\begin{enumerate}
\def\labelenumi{\arabic{enumi}.}
\tightlist
\item
  设函数\(f \in \mathcal{C}(\mathbb{R},\mathbb{R})\).定义函数\(F\):\[F(x)=\int_a^b f(x+t) \cos t \mathrm{d}t\]

  \begin{enumerate}
  \def\labelenumii{\arabic{enumii}.}
  \tightlist
  \item
    证明:\(F\)在\([a,b]\)上可导.
  \item
    计算\(F’\).
  \end{enumerate}
\item
  设\(f\)连续且以\(T\)为周期,\(G(x)=\int_x^{x+T}f(t)\mathrm{d}t\).证明:\(G\)为常值函数.
\item
  证明:\[\int_0^\frac{\pi}{2}\sin^nx\mathrm{d}x=\int_0^\frac{\pi}{2}\cos^nx\mathrm{d}x,n\in\mathbb{N}\].
\item
  设\(f\in\mathcal{C}([0,\pi],\mathbb{R})\).证明:

  \begin{enumerate}
  \def\labelenumii{\arabic{enumii}.}
  \tightlist
  \item
    \[\int_0^{\pi}xf(\sin x)\mathrm{d}x=\frac{\pi}{2}\int_0^{\pi}f(\sin x)\mathrm{d}x\]
  \item
    \[\int_0^\pi\frac{x\sin x}{1+\cos^2 x}\mathrm{d}x=\frac{\pi^2}{4}\]
  \end{enumerate}
\item
  证明:序列\(a_n=\int_0^n\frac{\sin x}{x}\mathrm{d}x\)收敛.
\item
  设\(f,g\in\mathcal{C}(\mathbb{R}_+,\mathbb{R}_+)\),满足\(f(x) \leqslant \Lambda + \int_0^x f(t)g(t) \mathrm{d}t\).证明:\[f(x) \leqslant \Lambda e^{\int_0^x g(t) \mathrm{d}t}\]
\item
  求下列不定积分.

  \begin{enumerate}
  \def\labelenumii{\arabic{enumii}.}
  \tightlist
  \item
    \[\int \arctan x \mathrm{d}x\]
  \item
    \[\int x^n \ln x \mathrm{d}x\]
  \item
    \[\int \sin(\ln x) \mathrm{d}x\]
  \item
    \[\int x^2 \arccos x \mathrm{d}x\]
  \item
    \[\int \sec^8 x \mathrm{d}x\]
  \item
    \[\int \frac{x \ln(x+\sqrt{1+x^2})}{1+x^2} \mathrm{d}x\]
  \item
    \[\int e^{ax} \sin bx \mathrm{d}x\]
  \item
    \[\int e^{ax} \cos bx \mathrm{d}x\]
  \item
    \[\int \frac{1}{x(x+1)(x^2+x+1)^2} \mathrm{d}x\]
  \end{enumerate}
\item
  计算下列定积分:

  \begin{enumerate}
  \def\labelenumii{\arabic{enumii}.}
  \tightlist
  \item
    \[\int_1^a \frac{1}{\sqrt{e^x-1}} \mathrm{d}x\]
  \item
    \[\int_a^b \frac{1}{\sqrt{x(1-x)}} \mathrm{d}x\]
  \item
    \[\int_0^{\frac{\pi}{2}} \frac{1}{\sin x+2\cos x+3} \mathrm{d}x\]
  \item
    \[\int_a^b \frac{\arcsin x}{x^2} \mathrm{d}x,0<a<b<1\]
  \end{enumerate}
\item
  依照下列步骤推导\textbf{Wallis公式}.

  \begin{enumerate}
  \def\labelenumii{\arabic{enumii}.}
  \tightlist
  \item
    令\(I_n=\int_0^\frac{\pi}{2} \cos^nx \mathrm{d}x\),利用分部积分证明\({} I_n=\dfrac{n-1}{n}I_{n-2} {}\).
  \item
    计算\(I_n\).
  \item
    证明:\(\lim\limits_{n\to\infty}\dfrac{I_{2n}}{I_{2n+1}}=1\).
  \item
    得到\({} \dfrac{\pi}{2}=\lim\limits_{n\to\infty}\left(\dfrac{2\cdot4\cdot6\cdots2n}{1\cdot3\cdot5\cdots(2n-1)}\right)^2\dfrac{1}{n} {}\).
  \end{enumerate}
\end{enumerate}

\end{document}
