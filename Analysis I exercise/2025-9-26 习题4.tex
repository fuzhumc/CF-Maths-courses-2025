% Options for packages loaded elsewhere
\PassOptionsToPackage{unicode}{hyperref}
\PassOptionsToPackage{hyphens}{url}
\documentclass[
  a4paper,
]{ctexart}

\usepackage{xcolor}
\usepackage{amsmath,amssymb}

\usepackage{ctex}
\usepackage{xeCJK}
\setCJKmainfont{SimSun}[BoldFont=SimHei,ItalicFont=KaiTi]


\setcounter{secnumdepth}{-\maxdimen} % remove section numbering

\usepackage{bookmark}
\IfFileExists{xurl.sty}{\usepackage{xurl}}{} % add URL line breaks if available
\urlstyle{same}

\usepackage{iftex}
\ifPDFTeX
  \usepackage[T1]{fontenc}
  \usepackage[utf8]{inputenc}
  \usepackage{textcomp} % provide euro and other symbols
\else % if luatex or xetex
  \defaultfontfeatures{Scale=MatchLowercase}
  \defaultfontfeatures[\rmfamily]{Ligatures=TeX,Scale=1}
\fi
\usepackage{lmodern}

\makeatletter
\@ifundefined{KOMAClassName}{% if non-KOMA class
  \IfFileExists{parskip.sty}{%
    \usepackage{parskip}
  }{% else
    \setlength{\parindent}{0pt}
    \setlength{\parskip}{6pt plus 2pt minus 1pt}}
}{% if KOMA class
  \KOMAoptions{parskip=half}}
\makeatother

\setlength{\emergencystretch}{3em} % prevent overfull lines
\providecommand{\tightlist}{\setlength{\itemsep}{0pt}\setlength{\parskip}{0pt}}

\usepackage{bookmark}
\IfFileExists{xurl.sty}{\usepackage{xurl}}{} % add URL line breaks if available
\urlstyle{same}
\hypersetup{hidelinks,pdfcreator={cjx}}


\usepackage{titling}
% 核心修改:若未手动设置 title,则自动用文件名()作为标题
\title{习题4}
\author{}
\date{2025年9月24日}

% 先加载 enumitem 宏包
\usepackage{enumitem}

% 重新用 enumitem 定义列表样式(此时无命令覆盖)
\usepackage{etoolbox}
\AtBeginEnvironment{enumerate}{
  \setlist[enumerate,2]{
    before={\def\labelenumii{(\arabic{enumii})}},  % 环境开始时执行一次
    itemjoin={\def\labelenumii{(\arabic{enumii})}},  % 每个 \item 前再执行一次(确保覆盖)
    label=(\arabic{enumii})  % 直接指定标签格式(双重保险)
  }
  \apptocmd{\tightlist}{\def\labelenumii{(\arabic{enumii})}}{}{}
}


\usepackage{fancyhdr}

\begin{document}
%%\maketitle
%
\pagestyle{fancy}
\fancyhf{}

\lhead{2025年9月24日}
\chead{习题4}

\renewcommand{\headrulewidth}{1pt}
\rfoot{\thepage}

\begin{enumerate}
\def\labelenumi{\arabic{enumi}.}
\tightlist
\item
  写下数列\((a_n)\)收敛的定义和发散的定义.发散一定意味着\(a_n\)跑到无穷吗?
\item
  利用定义证明\(\lim\limits_{n \to \infty}\frac{1}{n^2+n}=\lim\limits_{n \to \infty}\frac{1}{n^2-n}=0\).
\item
  分情况讨论\((a^n)\)的敛散性,其中\(a>0\).
\item
  证明:若\((a_n)\)收敛,则\((\vert a_n \rvert)\)收敛.反过来对吗?
\item
  计算极限:

  \begin{enumerate}
  \def\labelenumii{\arabic{enumii}.}
  \tightlist
  \item
    \(\lim\limits_{n \to \infty} \frac{n^2+2n+2}{n^\frac{5}{2}-n+1}\).
  \item
    \(\lim\limits_{n \to \infty} \frac{5n^{200}+3n^{100}+2}{4n^{200}-6n^{21}-7}\).
  \item
    \(\lim\limits_{n \to \infty} \frac{1}{\sqrt{n^2+n}-n}\).
  \item
    \(\lim\limits_{n \to \infty} \sin \frac{1}{n^2}\).
  \item
    \(\lim\limits_{n \to \infty} \frac{n^\varepsilon}{\ln n},\varepsilon > 0\).
  \item
    \(\lim\limits_{n \to \infty} \frac{e^x}{n!}\).
  \end{enumerate}

  \begin{quote}
  在计算或者讨论上述极限的时候,反思一下你是否正在使用一些看似自然,但未经证明的结论.如果有,不妨把他们写下来,然后思考一下这些事实对不对.
  \end{quote}
\item
  若\(a_1=\sqrt{3}\),然后归纳地定义\(a_{n+1}=\sqrt{3+a_n}\).证明\((a_n)\)收敛并计算极限.
\item
  定义\textbf{上极限}\(\varlimsup\limits_{n \to \infty} a_n =\lim\limits_{n \to \infty} \sup\limits_{j \geq n} \{a_j\}\),\textbf{下极限}\(\varliminf\limits_{n \to \infty} a_n= \lim\limits_{n \to \infty} \inf\limits_{j \geq n} \{a_n\}\).计算一下\(\sin \frac{n\pi}{2}\)的上下极限.想想看我们为什么要定义上下极限?
\item
  设\(a_1=0,a_{2n}=\frac{a_{2n-1}}{2},a_{2n+1}=\frac{1}{2}+a_{2n}\).计算\(a_n\)的上下极限.
\item
  \textbf{自然对数}我们考虑数列\(a_n=\left( 1+\frac{1}{n} \right)^n\).

  \begin{enumerate}
  \def\labelenumii{\arabic{enumii}.}
  \tightlist
  \item
    通过二项式展开,证明数列单增.
  \item
    证明数列有上界.
  \item
    由单调有界定理,得到极限,设为\(e\).
  \item
    由上面讨论,你是否能从中发现一个快速计算\(e\)的方法?
  \end{enumerate}
\end{enumerate}

\end{document}
