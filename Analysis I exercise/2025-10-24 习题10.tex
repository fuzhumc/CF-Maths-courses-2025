% Options for packages loaded elsewhere
\PassOptionsToPackage{unicode}{hyperref}
\PassOptionsToPackage{hyphens}{url}
\documentclass[
  a4paper,
]{article}

\usepackage{xcolor}
\usepackage{amsmath,amssymb}

\usepackage{ctex}
\usepackage{xeCJK}
\setCJKmainfont{SimSun}[BoldFont=SimHei,ItalicFont=KaiTi]


\setcounter{secnumdepth}{-\maxdimen} % remove section numbering

\usepackage{bookmark}
\IfFileExists{xurl.sty}{\usepackage{xurl}}{} % add URL line breaks if available
\urlstyle{same}

\usepackage{iftex}
\defaultfontfeatures{Scale=MatchLowercase}
\defaultfontfeatures[\rmfamily]{Ligatures=TeX,Scale=1}
\usepackage{lmodern}

\makeatletter
\@ifundefined{KOMAClassName}{% if non-KOMA class
  \IfFileExists{parskip.sty}{%
    \usepackage{parskip}
  }{% else
    \setlength{\parindent}{0pt}
    \setlength{\parskip}{6pt plus 2pt minus 1pt}}
}{% if KOMA class
  \KOMAoptions{parskip=half}}
\makeatother

\setlength{\emergencystretch}{3em} % prevent overfull lines
\providecommand{\tightlist}{\setlength{\itemsep}{0pt}\setlength{\parskip}{0pt}}

\usepackage{bookmark}
\IfFileExists{xurl.sty}{\usepackage{xurl}}{} % add URL line breaks if available
\urlstyle{same}
\hypersetup{hidelinks,pdfcreator={cjx}}


\usepackage{titling}
\title{习题10}
\author{}
\date{2025年10月24日}

\usepackage{geometry}
\geometry{a4paper,left=1.27cm,right=1.27cm,top=1.7cm,bottom=1.7cm}

\usepackage{enumitem}
\usepackage{etoolbox}
\AtBeginEnvironment{enumerate}{
  \apptocmd{\tightlist}{\def\labelenumii{(\arabic{enumii})}}{}{}
}


\usepackage{fancyhdr}

\begin{document}
\maketitle

\pagestyle{fancy}
\fancyhf{}

\lhead{2025年10月24日}
\chead{习题10}

\renewcommand{\headrulewidth}{1pt}
\rfoot{\thepage}

\section{不等式主题日}\label{ux4e0dux7b49ux5f0fux4e3bux9898ux65e5}

分析学的底层是证明各式各样的不等式,而不等式的核心是把握凸性.

\begin{enumerate}
\def\labelenumi{\arabic{enumi}.}
\tightlist
\item
  证明下列不等式:

  \begin{enumerate}
  \def\labelenumii{\arabic{enumii}.}
  \tightlist
  \item
    \(ln(1+x)>\frac{\arctan x}{1+x}\).
  \item
    \(e^x<\frac{1}{1-x}\).
  \item
    \((x^a+y^a)^\frac{1}{a}>(x^b+y^b)^\frac{1}{b},x>0,y>0,0<a<b\).
  \end{enumerate}
\item
  证明函数 \[
  f(x)=
  \begin{cases}
  (x-2)^2, & x \in [0,+\infty) \\
  (x+2)^2, & x \in (-\infty,0]
  \end{cases}
  \]
  在\((-\infty,0)\)与\((0,+\infty)\)上是凸函数,但在\(\mathbb{R}\)上不是凸的.
\item
  设\(f\)为凸函数,证明:

  \begin{enumerate}
  \def\labelenumii{\arabic{enumii}.}
  \tightlist
  \item
    对任意\(x \in \mathbb{R}\),\(f’(x^+)\)与\(f’(x^-)\)存在且\(f’(x^+) \leqslant f’(x^-)\).
  \item
    证明:\(f’(x^+)\)与\(f’(x^-)\)单调上升且对任意\(a<b\),有 \[
     f’(x^+) \leqslant \frac{f(b)-f(a)}{b-a} \leqslant f’(b^-)
     \]
  \end{enumerate}
\item
  设\(x,y \geqslant 0\)且\(q \geqslant 1\),证明: \[
  (x+y)^q \leqslant x^q+q2^{q-1}(x^{q-1}y+y^q)
  \]
\item
  设\(f: \mathbb{R} \to \mathbb{R}\)是任一函数,

  \begin{enumerate}
  \def\labelenumii{\arabic{enumii}.}
  \tightlist
  \item
    设\(f\)是凸的,证明:\(f\)的图像位于直线\(y=f(x)+m(x-a)\)的上方当且仅当\(f’(a^-) \leqslant m \leqslant f’(a^+)\).
  \item
    设\(f\)连续起右可导并满足不等式\(f(x) \leqslant f(a)+f’(a^+)(x-a)\).证明:\(f\)是凸函数.
  \end{enumerate}
\item
  设\(f:(a,b) \to \mathbb{R}\)是\(C^1\)的,且是严格凸的.证明:

  \begin{enumerate}
  \def\labelenumii{\arabic{enumii}.}
  \tightlist
  \item
    \(f\)每一点切线的斜率所组成的实数集合是一个开区间.
  \item
    \(f’\)可逆.
  \item
    定义函数\(h_{\lambda}(x)= \lambda x-f(x)\),则\(h_{\lambda}\)在\((a,b)\)上有唯一的极大值点,记其为\(x_{\lambda}\).
  \item
    定义\(g(\lambda)= \lambda x-f(x)\),证明\(g\)连续且凸.
  \end{enumerate}
\item
  令\(p>1,b_1,b_2,\ldots,b_n,\ldots>0\).我们接下来证明下列不等式: \[
   \sum_{x=1}^\infty \left( \frac{b_1+b_2+\cdots+b_n}{n} \right)^p \leqslant \left( \frac{p}{p-1} \right)^p \sum_{n=1}^\infty b_n^p
   \]

  \begin{enumerate}
  \def\labelenumii{\arabic{enumii}.}
  \tightlist
  \item
    考虑\(T_n=\frac{b_1+b_2+\cdots+b_n}{n},\Delta_n=T_n^p-\frac{p}{p-1}b_nT_n^{p-1}\).证明:\(\Delta_n \leqslant \frac{n-1}{p-1}T_{n-1}^p-\frac{n}{p-1}T_n^p\).
  \item
    证明:\(\sum\limits_{n=1}^N T_n^p \leqslant \frac{p}{p-1} \sum\limits_{n=1}^N b_n^p\).
  \item
    利用Hölder不等式证明:对任意\(N \in \mathbb{N}\),有 \[
     \sum_{n=1}^N \left( \frac{b_1+b_2+\cdots+b_n}{n} \right)^p \leqslant \left( \frac{p}{p-1} \right)^p \sum_{n=1}^N b_n^p
     \]
  \item
    令\(N \to \infty\)得到结论.这个不等式何时取等号?
  \item
    如果你学过一点积分,那么试着证明下面的积分版本: \[
     \int_0^\infty \left( \frac{1}{x} \int_0^x f(t) \mathrm{d} t \right)^p \mathrm{d} x \leqslant \left( \frac{p}{p-1} \right)^p \int_0^\infty f(x)^p \mathrm{d} x
     \]
  \item
    这个不等式在说一件什么事情?
  \end{enumerate}
\end{enumerate}

\end{document}
