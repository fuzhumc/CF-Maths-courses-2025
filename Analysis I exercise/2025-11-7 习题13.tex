% Options for packages loaded elsewhere
\PassOptionsToPackage{unicode}{hyperref}
\PassOptionsToPackage{hyphens}{url}
\documentclass[
  a4paper,
]{article}

\usepackage{xcolor}
\usepackage{amsmath,amssymb}

\usepackage{ctex}
\usepackage{xeCJK}
\setCJKmainfont{SimSun}[BoldFont=SimHei,ItalicFont=KaiTi]


\setcounter{secnumdepth}{-\maxdimen} % remove section numbering

\usepackage{bookmark}
\IfFileExists{xurl.sty}{\usepackage{xurl}}{} % add URL line breaks if available
\urlstyle{same}

\usepackage{iftex}
\defaultfontfeatures{Scale=MatchLowercase}
\defaultfontfeatures[\rmfamily]{Ligatures=TeX,Scale=1}
\usepackage{lmodern}

\makeatletter
\@ifundefined{KOMAClassName}{% if non-KOMA class
  \IfFileExists{parskip.sty}{%
    \usepackage{parskip}
  }{% else
    \setlength{\parindent}{0pt}
    \setlength{\parskip}{6pt plus 2pt minus 1pt}}
}{% if KOMA class
  \KOMAoptions{parskip=half}}
\makeatother

\setlength{\emergencystretch}{3em} % prevent overfull lines
\providecommand{\tightlist}{\setlength{\itemsep}{0pt}\setlength{\parskip}{0pt}}

\usepackage{bookmark}
\IfFileExists{xurl.sty}{\usepackage{xurl}}{} % add URL line breaks if available
\urlstyle{same}
\hypersetup{hidelinks,pdfcreator={cjx}}


\usepackage{titling}
\title{习题13}
\author{}
\date{2025年11月7日}

\usepackage{geometry}
\geometry{a4paper,left=1.27cm,right=1.27cm,top=1.7cm,bottom=1.7cm}

\usepackage{enumitem}
\usepackage{etoolbox}
\AtBeginEnvironment{enumerate}{
  \apptocmd{\tightlist}{\def\labelenumii{(\arabic{enumii})}}{}{}
}


\usepackage{fancyhdr}

\begin{document}
\maketitle

\pagestyle{fancy}
\fancyhf{}

\lhead{2025年11月7日}
\chead{习题13}

\renewcommand{\headrulewidth}{1pt}
\rfoot{\thepage}

\begin{enumerate}
\def\labelenumi{\arabic{enumi}.}
\tightlist
\item
  证明下面两个命题等价:

  \begin{enumerate}
  \def\labelenumii{\arabic{enumii}.}
  \tightlist
  \item
    \(f\)在\([a,b]\)上Riemann可积.
  \item
    对任意\(\varepsilon>0\),存在Riemann可积函数\(\varphi,\psi\)使得\(\varphi<f<\psi\)且\(\int_a^b\psi(x)-\varphi(x)\mathrm{d}x<\varepsilon\).
  \end{enumerate}
\item
  设\(f,g\)均为Riemann可积函数,求证:\(m(x)=\min\{f(x),g(x)\},M(x)=\{f(x),g(x)\}\)也Riemann可积.
\item
  设\(f:[a,b]\to\mathbb{R}\)Riemann可积,且存在\(m>0\)使得\(\left\lvert f(x) \right\rvert \geqslant m\).证明:\(\frac{1}{f}\)Riemann可积.
\item
  设\(f:[a,b]\to\mathbb{R}\)有界,称\(f\)是\textbf{正规的},若对任一点它的左右极限都存在(不一定相等;另外在端点处只要求单侧极限存在).依照上述定义,证明:

  \begin{enumerate}
  \def\labelenumii{\arabic{enumii}.}
  \tightlist
  \item
    若\(f\)单调,则\(f\)正规.
  \item
    Riemann函数在\([0,1]\)上正规.
  \item
    设\(f:[a,b]\to\mathbb{R}\)正规,证明:对于任意\(\varepsilon>0\),存在\([a,b]\)的一个分割\(a=a_0<a_1<a_2<\cdots<a_n=b\),使得对任意\(x,y\in(a_i,a_{i+1})\),有\(\left\lvert f(x)-f(y) \right\rvert < \varepsilon\).
  \item
    在(3)的设定中,存在阶梯函数\(h\)使得\(\left\lvert f(x)-h(x) \right\rvert < \varepsilon\).
  \item
    由此证明\(f\)Riemann可积.
  \end{enumerate}
\item
  证明:单调函数\(f:[a,b]\to\mathbb{R}\)Riemann可积.
\item
  证明:若\(f:[a,b]\to\mathbb{R}\)只有有限多个第一类间断点,则\(f\)Riemann可积.
\item
  称\(f:[a,b]\to\mathbb{R}\)\textbf{局部Riemann可积},若对于任意\([\alpha,\beta] \subseteq I\),\(f\)在\([\alpha,\beta]\)上都Riemann可积.

  \begin{enumerate}
  \def\labelenumii{\arabic{enumii}.}
  \tightlist
  \item
    证明:若函数\(f:[a,b]\to\mathbb{R}\)有界且在\((a,b)\)上局部Riemann可积,则\(f\)在\([a,b]\)上Riemann可积.
  \item
    考虑函数\[f(x)=\begin{cases} \sin\frac{1}{x}, & x\in(0,1] \\ 0, & x=0 \end{cases}\]证明:\(f\)在\([0,1]\)上不正规,但在\([0,1]\)上Riemann可积.
  \end{enumerate}
\item
  设\(E \subseteq \mathbb{R}\),称\(E\)是\textbf{零测集},若对任意\(\varepsilon>0\),存在有限个闭区间\([a_i,b_i],i=1,2,\cdots,m\)使得\(E \subseteq \bigcup\limits_{i=1}^m [a_i,b_i]\)且\(\sum\limits_{i=1}^m (b_i-a_i) <\varepsilon\).证明:

  \begin{enumerate}
  \def\labelenumii{\arabic{enumii}.}
  \tightlist
  \item
    有限点集是零测集.
  \item
    若序列\((x_n)\)收敛,则\((x_n)\)是零测集.
  \item
    若\(E\)是零测集,则它的任意子集以及它的闭包都是零测集.
  \end{enumerate}
\item
  证明:若有界函数\(f:[a,b]\to\mathbb{R}\)的间断点集是零测集,则\(f\)Riemann可积.
\item
  考虑\([0,1]\)上的Riemann函数\[R(x)=\begin{cases} 0, & x \in [0,1] \setminus \mathbb{Q} \\ 1, & x=0 \\ \frac{1}{q}, & x \in (0,1] \cup \mathbb{Q},x=\frac{p}{q},(p,q)=1 \end{cases}\]

  \begin{enumerate}
  \def\labelenumii{\arabic{enumii}.}
  \tightlist
  \item
    它的间断点集是不是零测集?
  \item
    证明:Riemann函数Riemann可积.
  \end{enumerate}
\item
  第9题和第10题说明了什么?我们对于零测集的定义出了什么问题?
\end{enumerate}

\end{document}
