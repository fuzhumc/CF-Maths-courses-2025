% Options for packages loaded elsewhere
\PassOptionsToPackage{unicode}{hyperref}
\PassOptionsToPackage{hyphens}{url}
\documentclass[
  a4paper,
]{article}

\usepackage{xcolor}
\usepackage{amsmath,amssymb}

\usepackage{ctex}
\usepackage{xeCJK}
\setCJKmainfont{SimSun}[BoldFont=SimHei,ItalicFont=KaiTi]


\setcounter{secnumdepth}{-\maxdimen} % remove section numbering

\usepackage{bookmark}
\IfFileExists{xurl.sty}{\usepackage{xurl}}{} % add URL line breaks if available
\urlstyle{same}

\usepackage{iftex}
\defaultfontfeatures{Scale=MatchLowercase}
\defaultfontfeatures[\rmfamily]{Ligatures=TeX,Scale=1}
\usepackage{lmodern}

\makeatletter
\@ifundefined{KOMAClassName}{% if non-KOMA class
  \IfFileExists{parskip.sty}{%
    \usepackage{parskip}
  }{% else
    \setlength{\parindent}{0pt}
    \setlength{\parskip}{6pt plus 2pt minus 1pt}}
}{% if KOMA class
  \KOMAoptions{parskip=half}}
\makeatother

\setlength{\emergencystretch}{3em} % prevent overfull lines
\providecommand{\tightlist}{\setlength{\itemsep}{0pt}\setlength{\parskip}{0pt}}

\usepackage{bookmark}
\IfFileExists{xurl.sty}{\usepackage{xurl}}{} % add URL line breaks if available
\urlstyle{same}
\hypersetup{hidelinks,pdfcreator={cjx}}


\usepackage{titling}
\title{习题6}
\author{}
\date{2025年10月9日}

\usepackage{geometry}
\geometry{a4paper,left=1.27cm,right=1.27cm,top=1.7cm,bottom=1.7cm}

\usepackage{enumitem}
\usepackage{etoolbox}
\AtBeginEnvironment{enumerate}{
  \apptocmd{\tightlist}{\def\labelenumii{(\arabic{enumii})}}{}{}
}


\usepackage{fancyhdr}

\begin{document}
\maketitle

\pagestyle{fancy}
\fancyhf{}

\lhead{2025年10月9日}
\chead{习题6}

\renewcommand{\headrulewidth}{1pt}
\rfoot{\thepage}

\begin{enumerate}
\def\labelenumi{\arabic{enumi}.}
\tightlist
\item
  利用极限定义证明:

  \begin{enumerate}
  \def\labelenumii{\arabic{enumii}.}
  \tightlist
  \item
    \(\lim\limits_{x \to 0} x \sin \frac{1}{x} = 0\).
  \item
    \(\lim\limits_{x \to +\infty} \frac{[x]}{x} = 1\).
  \item
    \(\lim\limits_{x \to +\infty} \left( \sqrt{x^2+x}-x \right) = \frac{1}{2}\).
  \item
    \(\lim\limits_{x \to n} x-[x]\)不存在,其中\(n \in \mathbb{Z}\).
  \end{enumerate}
\item
  设\(f:\mathbb{R} \to \mathbb{R}\)是以\(T\)为周期的周期函数,证明:若\(\lim\limits_{x \to \infty} f(x) = A\),则\(f\)恒等于\(A\).
\item
  对函数 \[
  f(x)=
  \begin{cases}
  x, & x \in \mathbb{Q} \\
  -x, & x \notin \mathbb{Q}
  \end{cases}
  \]
  证明:\(\lim\limits_{x \to 0} f(x) = 0\);对任意\(a \neq 0\),\(\lim\limits_{x \to a} f(x)\)不存在.
\item
  计算下列极限(\textbf{禁用洛必达}):

  \begin{enumerate}
  \def\labelenumii{\arabic{enumii}.}
  \tightlist
  \item
    \(\lim\limits_{x \to 0} \frac{\sin x}{x}\).
  \item
    \(\lim\limits_{x \to 0} \frac{\sin ax}{\sin bx}\).
  \item
    \(\lim\limits_{x \to 0} \frac{1-\cos x}{x^2}\).
  \item
    \(\lim\limits_{x \to 0} \frac{\sin(\sin(\sin x))}{x}\).
  \item
    \(\lim\limits_{x \to 0} \frac{(1+mx)^n-(1+nx)^m}{x^2},m,n \in \mathbb{N}\).
  \item
    \(\lim\limits_{x \to 0} \frac{\sqrt[n]{1+x}-1}{x}\).
  \end{enumerate}
\item
  默写函数在一点处左连续、右连续以及连续的定义.
\item
  我们总结一下间断点(不连续点)的定义:

  \begin{itemize}
  \tightlist
  \item
    若\(\lim\limits_{x \to a^-}f(x)\)存在,但不等于\(f(a)\),则称\(a\)是\(f\)的\textbf{第一类左侧间断点}.(类似可定义\textbf{第一类右侧间断点})
  \item
    若\(\lim\limits_{x \to a^-}f(x)\)不存在,则称\(a\)是\(f\)的\textbf{第二类左侧间断点}.(类似可定义\textbf{第二类右侧间断点})
  \item
    若\(a\)同时是\(f\)的第一类左、右侧间断点,则称\(a\)是\(f\)的\textbf{第一类间断点}.
  \item
    若进一步有\(\lim\limits_{x \to a^-}f(x)=\lim\limits_{x \to a^+}f(x)\),则称\(a\)是\(f\)的\textbf{可去间断点}.
  \item
    若\(a\)是\(f\)的间断点,但不是第一类间断点,则称\(a\)是\(f\)的\textbf{第二类间断点}.
  \end{itemize}
\item
  判断以下间断点类型

  \begin{enumerate}
  \def\labelenumii{\arabic{enumii}.}
  \tightlist
  \item
    \(x=0\), \[
     g(x)=
     \begin{cases}
     \frac{\sin x}{x}, &x \neq 0 \\
     2, &x=0
     \end{cases}
     \]
  \item
    \(x=0,f(x)=\mathrm{sgn}x\)
  \item
    \(x=0\), \[
     f(x)=
     \begin{cases}
     \sin \frac{1}{x}, & x \neq 0 \\
     0, & x=0
     \end{cases}
     \]
  \item
    对于函数 \[
     f(x)=
     \begin{cases}
     0, & x \in \mathbb{Q}^C \setminus \{0\} \\
     1, & x=0 \\
     \frac{1}{p}, & x=\frac{q}{p},(p,q)=1
     \end{cases}
     \] 考虑所有点的间断性质.
  \item
    对于函数 \[
     f(x)=
     \begin{cases}
     1, & x \in \mathbb{Q} \\
     0, & x \notin \mathbb{Q}
     \end{cases}
     \] 考虑所有点的间断性质.
  \end{enumerate}
\item
  设函数\(f\)满足:

  \begin{enumerate}
  \def\labelenumii{\arabic{enumii}.}
  \tightlist
  \item
    对任意\(x,y \in \mathbb{R},f(x)+f(y)=f(x+y)\).
  \item
    \(f\)在\(x=0\)处连续.
  \end{enumerate}

  证明:\(f\)在任意点处都连续.
\item
  设\(A \subseteq \mathbb{R}\)非空,定义\({} D(A)=\sup\limits_{x,y \in A} \left\lvert x-y \right\rvert {}\).设\(f\)为有界函数,定义\(\omega(f,x)=\inf\limits_{U\text{为}x\text{的开邻域}}D(f(U))\).证明:

  \begin{enumerate}
  \def\labelenumii{\arabic{enumii}.}
  \tightlist
  \item
    \(f\)在\(x\)处连续,当且仅当\(\omega(f,x)=0\).
  \item
    对任意\(\epsilon > 0,E=\{ x \in \mathbb{R} \mid \omega(f,x) \geqslant \varepsilon \}\)为闭集.
  \end{enumerate}
\item
  设\(f:[a,b] \to \mathbb{R}\)的每一个间断点都是可去间断点.证明:

  \begin{enumerate}
  \def\labelenumii{\arabic{enumii}.}
  \tightlist
  \item
    对任意\(\varepsilon > 0,E_\varepsilon=\{ x \in [a,b] \mid \left\lvert \lim\limits_{y \to x} f(y)-f(x) \right\rvert > \varepsilon \}\)为有限集.
  \item
    \(f\)的间断点集是可数集.
  \end{enumerate}
\end{enumerate}

\end{document}
