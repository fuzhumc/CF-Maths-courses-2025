% Options for packages loaded elsewhere
\PassOptionsToPackage{unicode}{hyperref}
\PassOptionsToPackage{hyphens}{url}
\documentclass[
  a4paper,
]{ctexart}

\usepackage{xcolor}
\usepackage{amsmath,amssymb}

\usepackage{ctex}
\usepackage{xeCJK}
\setCJKmainfont{SimSun}[BoldFont=SimHei,ItalicFont=KaiTi]


\setcounter{secnumdepth}{-\maxdimen} % remove section numbering

\usepackage{bookmark}
\IfFileExists{xurl.sty}{\usepackage{xurl}}{} % add URL line breaks if available
\urlstyle{same}

\usepackage{iftex}
\ifPDFTeX
  \usepackage[T1]{fontenc}
  \usepackage[utf8]{inputenc}
  \usepackage{textcomp} % provide euro and other symbols
\else % if luatex or xetex
  \defaultfontfeatures{Scale=MatchLowercase}
  \defaultfontfeatures[\rmfamily]{Ligatures=TeX,Scale=1}
\fi
\usepackage{lmodern}

\makeatletter
\@ifundefined{KOMAClassName}{% if non-KOMA class
  \IfFileExists{parskip.sty}{%
    \usepackage{parskip}
  }{% else
    \setlength{\parindent}{0pt}
    \setlength{\parskip}{6pt plus 2pt minus 1pt}}
}{% if KOMA class
  \KOMAoptions{parskip=half}}
\makeatother

\setlength{\emergencystretch}{3em} % prevent overfull lines
\providecommand{\tightlist}{\setlength{\itemsep}{0pt}\setlength{\parskip}{0pt}}

\usepackage{bookmark}
\IfFileExists{xurl.sty}{\usepackage{xurl}}{} % add URL line breaks if available
\urlstyle{same}
\hypersetup{hidelinks,pdfcreator={cjx}}


\usepackage{titling}
% 核心修改:若未手动设置 title,则自动用文件名()作为标题
\title{习题7}
\author{}
\date{2025年10月13日}

% 先加载 enumitem 宏包
\usepackage{enumitem}

% 重新用 enumitem 定义列表样式(此时无命令覆盖)
\usepackage{etoolbox}
\AtBeginEnvironment{enumerate}{
  \setlist[enumerate,2]{
    before={\def\labelenumii{(\arabic{enumii})}},  % 环境开始时执行一次
    itemjoin={\def\labelenumii{(\arabic{enumii})}},  % 每个 \item 前再执行一次(确保覆盖)
    label=(\arabic{enumii})  % 直接指定标签格式(双重保险)
  }
  \apptocmd{\tightlist}{\def\labelenumii{(\arabic{enumii})}}{}{}
}


\usepackage{fancyhdr}

\begin{document}
%%\maketitle
%
\pagestyle{fancy}
\fancyhf{}

\lhead{2025年10月13日}
\chead{习题7}

\renewcommand{\headrulewidth}{1pt}
\rfoot{\thepage}

\begin{enumerate}
\def\labelenumi{\arabic{enumi}.}
\tightlist
\item
  设\(f:\mathbb{R} \to \mathbb{R}\)满足\textbf{介值性质}:即若\(f(a)<c<f(b)\),则存在\(x\)在\(a,b\)之间,使得\(f(x)=c\).进一步假设对于任意有理数\(r\),集合\(\{ x \mid f(x)=r \}\)是闭集,证明:\(f\)连续.
\item
  称\((a,b)\)上的实值函数\(f\)为\textbf{凸函数},若对任意\(x,y \in (a,b),\lambda \in (0,1)\),有
  \[
   f(\lambda x+(1-\lambda)y) \leq \lambda f(x)+(1-\lambda)f(y)
   \] 证明:

  \begin{enumerate}
  \def\labelenumii{\arabic{enumii}.}
  \tightlist
  \item
    凸函数总是连续.
  \item
    单增凸函数和凸函数的复合也是凸函数.
  \item
    对于\(a<s<t<u<b\),证明: \[
     \frac{f(t)-f(s)}{t-s} \leq \frac{f(u)-f(s)}{u-s} \leq \frac{f(u)-f(t)}{u-t}
     \]
  \end{enumerate}
\item
  设连续单增函数\(f:(a,b) \to \mathbb{R}\)满足 \[
  f\left( \frac{x+y}{2} \right) \leq \frac{f(x)+f(y)}{2},\forall x,y \in (a,b)
  \] 证明:\(f\)为凸函数.
\item
  我们首先回顾聚点定义:称\(a\)是集合\(X\)的\textbf{聚点},若对任意\(\delta > 0\),存在\(x \in X\),使得\(x \in (a-\delta,a+\delta)\).证明:若对任意\(x_n \in X \setminus \{ a \},x_n \to a\),且满足\(\vert x_{n+1}-a \rvert<\vert x_n-a \rvert\),都有\(\lim\limits_{n \to \infty}f(x_n)=A\),则\(\lim\limits_{x \to a}f(x)=A\).
\item
  设\(f:(a,+\infty) \to \mathbb{R}\)满足:

  \begin{enumerate}
  \def\labelenumii{\arabic{enumii}.}
  \tightlist
  \item
    对任意\(b>a\),存在\(M>0\)使得\(\left\lvert f(x) \right\rvert \leq M\)对任意\(x \in (a,b)\)成立.
  \item
    \(\lim\limits_{x \to +\infty}\left( f(x+1)-f(x) \right)=+\infty\).
  \end{enumerate}

  证明:\(\lim\limits_{x \to +\infty} \frac{f(x)}{x}=+\infty\).
\item
  定义:对于任意函数\(f:X \to \mathbb{R}\),称\(f\)在\(a \in X\)处\textbf{局部有界},若存在\(\delta > 0\)使得\(f\)在\(X \cap (a-\delta,a+\delta)\)上有界.称\(f\)\textbf{在\(X\)上局部有界},若\(f\)在\(X\)上每一点都局部有界.举例:

  \begin{enumerate}
  \def\labelenumii{\arabic{enumii}.}
  \tightlist
  \item
    局部有界不一定有界.
  \item
    存在处处局部不有界的函数.
  \end{enumerate}
\item
  设\(I \subseteq \mathbb{R}\)是任一区间,\(f:I \to \mathbb{R}\)是任意函数,我们称\(f\)在\(x_0 \in I\)处是\textbf{增大的},若存在\(\delta>0\)使得
  \[
   f(x)-f(x_0)
   \begin{cases}
   >0, & \forall x \in I \cap (x_0,x_0+\delta) \\
   <0, & \forall x \in I \cap (x_0-\delta,x_0)
   \end{cases}
   \] 证明:

  \begin{enumerate}
  \def\labelenumii{\arabic{enumii}.}
  \tightlist
  \item
    函数 \[
     f(x)=
     \begin{cases}
     0, & x=0 \\
     x \left( 1+x \sin \frac{1}{x} \right), & x \neq 0
     \end{cases}
     \]
    在\(x=0\)处增大,但对任意\(\varepsilon > 0\),\(f\)不是在\((-\varepsilon,\varepsilon)\)的每一点处都增大.
  \item
    证明:\(f\)在每一点处都增大当且仅当\(f\)严格单增.
  \end{enumerate}
\item
  对于\(f:\mathbb{R} \to \mathbb{R}\)连续,证明:

  \begin{enumerate}
  \def\labelenumii{\arabic{enumii}.}
  \tightlist
  \item
    开集(闭集)的原象是开集(闭集).
  \item
    紧集的象是紧集.(回忆一下\textbf{紧集}的定义:任何开覆盖都有有限子覆盖)
  \end{enumerate}
\item
  设连续函数\(f,g:[a,b] \to \mathbb{R}\)满足\(0<g(x)<f(x),\forall x \in [a,b]\).证明:存在\(k>1\)使得\(kg(x) \leq f(x)\).
\item
  设\(f,g:[a,b] \to [a,b]\)都为连续函数.证明:

  \begin{enumerate}
  \def\labelenumii{\arabic{enumii}.}
  \tightlist
  \item
    存在\(x_0 \in [a,b]\)使得\(f(x_0)=x_0\).
  \item
    若\(f \circ g = g \circ f\),则存在\(x_0 \in [a,b]\)使得\(f(x_0)=g(x_0)\).
  \end{enumerate}
\item
  设\(I \subseteq \mathbb{R}\)为一区间,\(f:I \to \mathbb{R}\)连续.证明:

  \begin{enumerate}
  \def\labelenumii{\arabic{enumii}.}
  \tightlist
  \item
    若\(f(I) \subseteq \mathbb{Q}\),则\(f\)为常值函数.
  \item
    若\(f(I) \subseteq \mathbb{Q}^C\),则\(f\)为常值函数.
  \end{enumerate}
\item
  回忆一致连续定义.

  \begin{enumerate}
  \def\labelenumii{\arabic{enumii}.}
  \tightlist
  \item
    证明:\(f(x)=\sqrt x\)在\([0,1]\)上一致连续.
  \item
    证明:\(f(x)=\sin x^2\)在\(\mathbb{R}\)上不一致连续.
  \item
    设\(X \subseteq \mathbb{R}\)为闭集,\(f:X \to \mathbb{R}\)为单调的有界连续函数,证明:\(f\)在\(X\)上一致连续.
  \end{enumerate}
\end{enumerate}

\end{document}
