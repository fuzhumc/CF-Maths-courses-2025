% Options for packages loaded elsewhere
\PassOptionsToPackage{unicode}{hyperref}
\PassOptionsToPackage{hyphens}{url}
\documentclass[
  a4paper,
]{ctexart}

\usepackage{xcolor}
\usepackage{amsmath,amssymb}

\usepackage{ctex}
\usepackage{xeCJK}
\setCJKmainfont{SimSun}[BoldFont=SimHei,ItalicFont=KaiTi]


\setcounter{secnumdepth}{-\maxdimen} % remove section numbering

\usepackage{bookmark}
\IfFileExists{xurl.sty}{\usepackage{xurl}}{} % add URL line breaks if available
\urlstyle{same}

\usepackage{iftex}
\ifPDFTeX
  \usepackage[T1]{fontenc}
  \usepackage[utf8]{inputenc}
  \usepackage{textcomp} % provide euro and other symbols
\else % if luatex or xetex
  \defaultfontfeatures{Scale=MatchLowercase}
  \defaultfontfeatures[\rmfamily]{Ligatures=TeX,Scale=1}
\fi
\usepackage{lmodern}

\makeatletter
\@ifundefined{KOMAClassName}{% if non-KOMA class
  \IfFileExists{parskip.sty}{%
    \usepackage{parskip}
  }{% else
    \setlength{\parindent}{0pt}
    \setlength{\parskip}{6pt plus 2pt minus 1pt}}
}{% if KOMA class
  \KOMAoptions{parskip=half}}
\makeatother

\setlength{\emergencystretch}{3em} % prevent overfull lines
\providecommand{\tightlist}{\setlength{\itemsep}{0pt}\setlength{\parskip}{0pt}}

\usepackage{bookmark}
\IfFileExists{xurl.sty}{\usepackage{xurl}}{} % add URL line breaks if available
\urlstyle{same}
\hypersetup{hidelinks,pdfcreator={cjx}}


\usepackage{titling}
% 核心修改:若未手动设置 title,则自动用文件名()作为标题
\title{习题8}
\author{}
\date{2025年10月17日}

% 先加载 enumitem 宏包
\usepackage{enumitem}

% 重新用 enumitem 定义列表样式(此时无命令覆盖)
\usepackage{etoolbox}
\AtBeginEnvironment{enumerate}{
  \setlist[enumerate,2]{
    before={\def\labelenumii{(\arabic{enumii})}},  % 环境开始时执行一次
    itemjoin={\def\labelenumii{(\arabic{enumii})}},  % 每个 \item 前再执行一次(确保覆盖)
    label=(\arabic{enumii})  % 直接指定标签格式(双重保险)
  }
  \apptocmd{\tightlist}{\def\labelenumii{(\arabic{enumii})}}{}{}
}


\usepackage{fancyhdr}

\begin{document}
%%\maketitle
%
\pagestyle{fancy}
\fancyhf{}

\lhead{2025年10月17日}
\chead{习题8}

\renewcommand{\headrulewidth}{1pt}
\rfoot{\thepage}

\begin{enumerate}
\def\labelenumi{\arabic{enumi}.}
\item
  考虑Dirichlet函数 \[
   D(x)=
   \begin{cases}
   1, & x \in \mathbb{Q} \\
   0, & x \notin \mathbb{Q}
   \end{cases}
   \]

  \begin{enumerate}
  \def\labelenumii{\arabic{enumii}.}
  \tightlist
  \item
    证明:\(D\)处处不可导.
  \item
    证明:\(g(x)=x^2D(x)\)只在\(x=0\)处可导且\(g’(0)=0\).
  \end{enumerate}
\item
  考虑函数 \[
  f(x)=
  \begin{cases}
  x^2 \vert \sin \frac{\pi}{x} \rvert, & x \neq 0 \\
  0, & x=0
  \end{cases}
  \]
  证明:\(f\)在\(x=0\)的任何邻域内都有不可导的点,但在\(x=0\)处可导,且\(f’(0)=0\).
\item
  设函数\(f:\mathbb{R} \to \mathbb{R}\)在\(x=0\)连续,且\(f(0)=0\).证明:若\(\lim\limits_{x \to 0} \frac{f(2x)-f(x)}{x} = A\),则\(f\)在\(x=0\)处可导且\(f’(0)=A\).
\item
  设\(f,g\)在\(a\)处均可导,证明:

  \begin{enumerate}
  \def\labelenumii{\arabic{enumii}.}
  \tightlist
  \item
    若\(f(a)\neq 0\),则\(\vert f \rvert\)在\(a\)处也可导.
  \item
    若\(f(a) \neq g(a)\),则函数\(m(x)=\min \{f(x),g(x)\},M(x)=\max \{f(x),g(x)\}\)在\(a\)处也可导.
  \end{enumerate}
\item
  计算导数:

  \begin{enumerate}
  \def\labelenumii{\arabic{enumii}.}
  \tightlist
  \item
    \(f(x)=\frac{1}{\left( 1+ \vert x \rvert ^2 \right)^\alpha}\).
  \item
    \(f(x)=\arcsin \left( \cos x^2 \right)\).
  \item
    \(f(x)=\ln \left( \cosh x \right)+\frac{1}{2 \sinh^2 x}\).
  \item
    \(f(x)={\sin x}^{\cos x}+{\cos x}^{\sin x}\).
  \item
    \(f(x)=\left( \ln x \right)^{x^\alpha}+x^{\alpha \ln x}\).
  \end{enumerate}
\item
  设函数\(f\)满足\(f(x)^x=x^{f(x)}\),试算\(f’(x)\).
\item
  利用函数\((x-a)^n(x-b)^m\)证明 \[
  \binom{n}{0}^2+\binom{n}{1}^2+\cdots+\binom{n}{n}^2=\frac{2n(2n-1) \cdots (n+1)}{n!}
  \]
\item
  证明函数 \[
  f(x)=
  \begin{cases}
  e^{\frac{1}{\vert x \rvert^2-1}}, & \vert x \rvert \leq 1 \\
  0, & \vert x \rvert > 1
  \end{cases}
  \] 是\(C^\infty\)的(任意阶可导).
\item
  证明\textbf{Legendre多项式} \[
  P_n(x)=\frac{1}{2^n n!}\left[ \left( x^2-1 \right)^n \right]^{(n)}
  \] 满足方程 \[
  \left( 1-x^2 \right)P’’_n(x)-2xP’_n(x)+n(n+1)P_n(x)=0
  \]
\item
  设函数\(f,g:[a,b] \to \mathbb{R}\)满足:

  \begin{enumerate}
  \def\labelenumii{\arabic{enumii}.}
  \tightlist
  \item
    \(f,g\)在\([a,b]\)上连续.
  \item
    \(f,g\)在\((a,b)\)上可导,且\(\left( f’ \right)^2+\left( g’ \right)^2 \neq 0\).
  \item
    \(g(a) \neq g(b)\).
  \end{enumerate}

  证明:存在\(\xi \in (a,b)\)使得 \[
  \frac{f(b)-f(a)}{g(b)-g(a)}=\frac{f’(\xi)}{g’(\xi)}
  \]
\item
  证明:在\(e^x \sin x = 1\)的两个实根中至少有\(e^x \cos x = -1\)的一个实根.
\item
  设\(f:[a,b] \to \mathbb{R}\)是任意函数,对\(\alpha > 0\),我们称\(f\)为\textbf{\(\alpha\)次Lipschitz函数},如果存在\(k>0\)(称为\textbf{Lipschitz系数})使得
  \[
  \vert f(x)-f(y) \rvert \leq k \vert x-y \rvert ^\alpha,\forall x,y \in [a,b]
  \]

  \begin{enumerate}
  \def\labelenumii{\arabic{enumii}.}
  \tightlist
  \item
    证明:若\(f\)在\([a,b]\)上是连续可导的,则\(f\)是1次Lipschitz函数(简称为\textbf{Lipschitz函数}).
  \item
    证明:若\(f\)是可导的Lipschitz函数,则\(f’\)在\([a,b]\)上有界.
  \item
    证明:若\(f\)是\(\alpha > 1\)次Lipschitz函数,则\(f\)在\([a,b]\)上为常值函数.
  \end{enumerate}
\item
  设\(f(x):\mathbb{R} \to \mathbb{R}\)为Lipschitz函数,我们称\(E=\{(x,f(x)) \in \mathbb{R}^2 \mid x \in \mathbb{R}\}\)为\textbf{Lipchitz图像}.设\(f(x)\)在\(x=x_1\)处可导,向量\((1,f’(x_1))\)称为图像在\(x=x_1\)处的\textbf{切向量},\(\nu(x_1)=\frac{1}{\sqrt{1+ \vert f’(x_1) \rvert ^2}}\left( f’(x_1),-1 \right)\)称为此处的\textbf{单位外法向量}.现考虑两点\(X_1=\left( x_1,f\left( x_1 \right) \right),X_2=\left( x_2,y_2 \right)\),其中\(y_2 \geq f\left( x_2 \right)\).定义\(dist \left( X,E \right)=\inf_{Y \in E} \vert X-Y \rvert\).若存在\(C>0\)使得
  \[
  \frac{1}{C}dist \left( X_2,E \right) \leq |X_1-X_2| \leq Cdist \left( X_2,E \right)
  \]
  证明:存在\(\varepsilon > 0\)使得当Lipschitz常数\(k < \varepsilon\)时,\(\left( X_1-X_2 \right) \cdot \nu \left( X_2 \right) \geq 0\).(思考一下这个问题在说一件什么事情?)
\end{enumerate}

\end{document}
