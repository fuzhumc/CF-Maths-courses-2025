% Options for packages loaded elsewhere
\PassOptionsToPackage{unicode}{hyperref}
\PassOptionsToPackage{hyphens}{url}
\documentclass[
  a4paper,
]{article}

\usepackage{xcolor}
\usepackage{amsmath,amssymb}

\usepackage{ctex}
\usepackage{xeCJK}
\setCJKmainfont{SimSun}[BoldFont=SimHei,ItalicFont=KaiTi]


\setcounter{secnumdepth}{-\maxdimen} % remove section numbering

\usepackage{bookmark}
\IfFileExists{xurl.sty}{\usepackage{xurl}}{} % add URL line breaks if available
\urlstyle{same}

\usepackage{iftex}
\defaultfontfeatures{Scale=MatchLowercase}
\defaultfontfeatures[\rmfamily]{Ligatures=TeX,Scale=1}
\usepackage{lmodern}

\makeatletter
\@ifundefined{KOMAClassName}{% if non-KOMA class
  \IfFileExists{parskip.sty}{%
    \usepackage{parskip}
  }{% else
    \setlength{\parindent}{0pt}
    \setlength{\parskip}{6pt plus 2pt minus 1pt}}
}{% if KOMA class
  \KOMAoptions{parskip=half}}
\makeatother

\setlength{\emergencystretch}{3em} % prevent overfull lines
\providecommand{\tightlist}{\setlength{\itemsep}{0pt}\setlength{\parskip}{0pt}}

\usepackage{bookmark}
\IfFileExists{xurl.sty}{\usepackage{xurl}}{} % add URL line breaks if available
\urlstyle{same}
\hypersetup{hidelinks,pdfcreator={cjx}}


\usepackage{titling}
\title{Ch13习题}
\author{}
\date{\today}

\usepackage{geometry}
\geometry{a4paper,left=1.27cm,right=1.27cm,top=1.7cm,bottom=1.7cm}

\usepackage{enumitem}
\usepackage{etoolbox}
\AtBeginEnvironment{enumerate}{
  \apptocmd{\tightlist}{\def\labelenumii{(\arabic{enumii})}}{}{}
}


\usepackage{fancyhdr}

\begin{document}
\maketitle

\pagestyle{fancy}
\fancyhf{}

\lhead{\today}
\chead{Ch13习题}

\renewcommand{\headrulewidth}{1pt}
\rfoot{\thepage}

\begin{enumerate}
\def\labelenumi{\arabic{enumi}.}
\tightlist
\item
  \begin{enumerate}
  \def\labelenumii{\arabic{enumii}.}
  \tightlist
  \item
    证明:函数\(f(x)=\ln(1+2x)\)在0附近相对于函数\(g(x)=x\ln x\)可忽略。
  \item
    证明:函数\(f(x)=\sqrt{x^2+3x}\ln(x^2)\sin x\)在0附近被函数\(g(x)=x\ln x\)控制.
  \end{enumerate}
\item
  在\(+\infty\)附近比较\(e^{x^2}\)与\(\int_0^xe^{t^2}\mathrm{d}t\).
\item
  通过等价式求下列极限.

  \begin{enumerate}
  \def\labelenumii{\arabic{enumii}.}
  \tightlist
  \item
    \(\lim\limits_{x\to0}x(3+x)\dfrac{\sqrt{x+3}}{\sqrt{x}\sin\sqrt{x}}\).
  \item
    \(\lim\limits_{x\to0}\dfrac{(1-e^x)(1-\cos x)}{3x^3+2x^4}\).
  \item
    \(\lim\limits_{x\to0}(1+\sin x)^{\frac{1}{x}}\).
  \end{enumerate}
\item
  求\(n\to\infty\)时的简单等价式

  \begin{enumerate}
  \def\labelenumii{\arabic{enumii}.}
  \tightlist
  \item
    \(\left(\ln\left(1+e^{-n^2}\right)\right)^{\frac{1}{n}}\).
  \item
    \(\left(\dfrac{e^n}{1+e^{-n}}\right)^n\).
  \end{enumerate}
\item
  \begin{enumerate}
  \def\labelenumii{\arabic{enumii}.}
  \tightlist
  \item
    求\(x\to0\)时\(\left(\dfrac{x}{\sin x}\right)^{\frac{\sin x}{x-\sin x}}\)的极限.
  \item
    求\(x\to0\)时\(\left(1+3\tan^2x\right)^\frac{1}{x\sin x}\)的极限.
  \end{enumerate}
\item
  设\(n\in\mathbb{N}^*\),考虑定义在\([0,\dfrac{\pi}{2}]\)上的函数\[f_n(x)=n\cos^n x\sin x\]

  \begin{enumerate}
  \def\labelenumii{\arabic{enumii}.}
  \tightlist
  \item
    证明:对任意\(n\in\mathbb{N}^*\),函数\(f_n\)有最大值,且最大值在唯一一点处取到.
    记\(x_n\)为\(f_n\)取极大值的点,\(y_n=f_n(x_n)\)
  \item
    求\(x_n,y_n\)的等价式.
  \end{enumerate}
\item
  求\(u_n=\dfrac{(n+1)^{\frac{n+1}{n}}-(n-1)^{\frac{n-1}{n}}}{n}\)的等价式.
\item
  \begin{enumerate}
  \def\labelenumii{\arabic{enumii}.}
  \tightlist
  \item
    求\(f(x)=\sqrt{x}\)在2处的3阶限定展开.
  \item
    \(f\)在0处有\(n\geqslant0\)阶限定展开吗?
  \end{enumerate}
\item
  求下列限定展开.

  \begin{enumerate}
  \def\labelenumii{\arabic{enumii}.}
  \tightlist
  \item
    \(f(x)=\sqrt{1-x}+\sqrt{1+x}\),4阶,0处.
  \item
    \(f(x)=\left(\ln(1+x)\right)^2\),4阶,0处.
  \item
    \(f(x)=\dfrac{x^2+1}{x^2+2x+2}\),3阶,0处.
  \item
    \(f(x)=\ln\left(\dfrac{1}{\cos x}\right)\),4阶,0处.
  \item
    \(f(x)=\dfrac{1}{(x+1)(x-2)}\),3阶,0处.
  \end{enumerate}
\item
  求函数\(\arctan\)在1处的4阶限定展开.
\item
  设\(f(x)=\sqrt[3]{(x^2-2)(x+3)}\).

  \begin{enumerate}
  \def\labelenumii{\arabic{enumii}.}
  \tightlist
  \item
    证明:存在\(a,b,c\in\mathbb{R}\)使得\(f(x)=ax+b+\dfrac{c}{x}+o(\dfrac{1}{x})\).
  \item
    用图像解释上述结果.
  \end{enumerate}
\item
  求下列极限.

  \begin{enumerate}
  \def\labelenumii{\arabic{enumii}.}
  \tightlist
  \item
    \(\lim\limits_{x\to0}\dfrac{a^x-b^x}{x},a,b\in\mathbb{R}_+^*\).
  \item
    \(\lim\limits_{x\to0}\left(\dfrac{a^x+b^x}{2}\right)^{\frac{1}{x}},a,b\in\mathbb{R}_+^*\).
  \item
    \(\lim\limits_{x\to0}\dfrac{(1+x)^{\frac{1}{x}}-e}{x}\).
  \end{enumerate}
\item
  设函数\(f\)定义如下:\[f(x)=\int_x^{x^2}\frac{\mathrm{d}t}{1+t^2}\]

  \begin{enumerate}
  \def\labelenumii{\arabic{enumii}.}
  \tightlist
  \item
    求\(f\)在0处的4阶限定展开.
  \item
    求\(f\)在0附近的图像.
  \end{enumerate}
\end{enumerate}

\end{document}
