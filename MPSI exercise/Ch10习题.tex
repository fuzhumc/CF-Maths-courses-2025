% Options for packages loaded elsewhere
\PassOptionsToPackage{unicode}{hyperref}
\PassOptionsToPackage{hyphens}{url}
\documentclass[
  a4paper,
]{article}

\usepackage{xcolor}
\usepackage{amsmath,amssymb}

\usepackage{ctex}
\usepackage{xeCJK}
\setCJKmainfont{SimSun}[BoldFont=SimHei,ItalicFont=KaiTi]


\setcounter{secnumdepth}{-\maxdimen} % remove section numbering

\usepackage{bookmark}
\IfFileExists{xurl.sty}{\usepackage{xurl}}{} % add URL line breaks if available
\urlstyle{same}

\usepackage{iftex}
\defaultfontfeatures{Scale=MatchLowercase}
\defaultfontfeatures[\rmfamily]{Ligatures=TeX,Scale=1}
\usepackage{lmodern}

\makeatletter
\@ifundefined{KOMAClassName}{% if non-KOMA class
  \IfFileExists{parskip.sty}{%
    \usepackage{parskip}
  }{% else
    \setlength{\parindent}{0pt}
    \setlength{\parskip}{6pt plus 2pt minus 1pt}}
}{% if KOMA class
  \KOMAoptions{parskip=half}}
\makeatother

\setlength{\emergencystretch}{3em} % prevent overfull lines
\providecommand{\tightlist}{\setlength{\itemsep}{0pt}\setlength{\parskip}{0pt}}

\usepackage{bookmark}
\IfFileExists{xurl.sty}{\usepackage{xurl}}{} % add URL line breaks if available
\urlstyle{same}
\hypersetup{hidelinks,pdfcreator={cjx}}


\usepackage{titling}
\title{Ch10习题}
\author{}
\date{\today}

\usepackage{enumitem}
\usepackage{etoolbox}
\AtBeginEnvironment{enumerate}{
  \apptocmd{\tightlist}{\def\labelenumii{(\arabic{enumii})}}{}{}
}


\usepackage{fancyhdr}

\begin{document}
\maketitle

\pagestyle{fancy}
\fancyhf{}

\lhead{\today}
\chead{Ch10习题}

\renewcommand{\headrulewidth}{1pt}
\rfoot{\thepage}

\begin{enumerate}
\def\labelenumi{\arabic{enumi}.}
\tightlist
\item
  设\(f\)为\(\mathbb{R}\)上函数.

  \begin{enumerate}
  \def\labelenumii{\arabic{enumii}.}
  \tightlist
  \item
    证明:若\(f\)在\(0\)处可导,则\({} \lim\limits_{x\to0}\dfrac{f(x)-f(-x)}{2x}=f’(0) {}\).
  \item
    反过来呢?
  \end{enumerate}
\item
  设\(f:\mathbb{R}\to\mathbb{R}\)在\(0\)处连续,满足\({} \lim\limits_{x\to0}\dfrac{f(2x)-f(x)}{x} {}\)存在且有限,求证\(f\)在\(0\)处可导.
\item
  设\(a \in I\),函数\(f:I\to\mathbb{R}\)在\(a\)处可导.设\(\varphi(x)=f(a)+f’(a)(x-a),g(x)=\mu+\lambda(x-a)\),其中\(\lambda,\mu\in\mathbb{R}\)证明:若\((\lambda,\mu)\neq(f’(a),f(a))\),则存在\(\eta>0\)使得当\(x \in I\)满足\(0<\left\lvert x-a \right\rvert \leqslant \eta {}\)时,有\(\left\lvert f(x)-g(x) \right\rvert < \left\lvert f(x)-\varphi(x) \right\rvert\).\emph{分为情形}\(\mu \neq f(a)\)\emph{与}\(\mu = f(a),\lambda \neq f’(a)\).
\item
  设两函数\(f,g\)在\([a,b]\)上连续,在\((a,b)\)上可导.证明:\[\exists c \in (a,b),(f(a)-f(b))g’(c)=(g(a)-g(b))f’(c)\]
\item
  设函数\(f\)在\([a,b]\)上可导,满足\(f(a)=f(b)=0,f’(a)=0\).证明:\[\exists c \in (a,b),f’(c)=\dfrac{f(c)-f(a)}{c-a}\]
\item
  设函数\(f:\mathbb{R}_+\to\mathbb{R}\)在\(\mathbb{R}_+^*\)上可导,满足\(f(0)=\lim\limits_{x\to+\infty}f(x)=0\).求证:\(\exists c \in (0,+\infty),f’(c)=0\).
\item
  证明不等式:\[\forall x \in [0,\frac{\pi}{2}],\frac{2}{\pi}x \leqslant \sin x \leqslant x\]
\item
  设\(f\)在\(I\)上可导.

  \begin{enumerate}
  \def\labelenumii{\arabic{enumii}.}
  \tightlist
  \item
    设\(a,b \in I\)满足\(f’(a)<0,f’(b)>0\).证明:\(\exists c \in I,f’(c)=0\).
  \item
    \textbf{Thm.(Darboux)} 证明\(f’(I)\)为区间.
  \end{enumerate}
\item
  设两函数\(f,g\)在\(I\)上连续,在\(I\)的内部可导,设\(\left\lvert f’(x) \right\rvert \leqslant g’(x)\).证明:\(\left\lvert f(a)-f(b) \right\rvert \leqslant g(a)-g(b)\).
\item
  设\(a\)为\(I\)内点,函数\(f:i\to\mathbb{R}\)在\(I\)上可导,且在\(a\)处二阶可导(即\(f’\)在\(a\)处可导).称\(f\)在\(a\)处取到\textbf{严格局部最小值},若存在\(\eta>0\)使得当\(x \in I\)满足\(0<\left\lvert x-a \right\rvert \leqslant \eta\)时,有\(f(x)>f(a)\).证明:若\(f’(a)=0,f’’(a)>0\),则\(a\)为一个严格局部最小值.
\item
  设\({} f(x)=\dfrac{1}{\sqrt{1-x^2}},x\in[0,1) {}\).证明:对于任意自然数\(n\)与\(x\in[0,1)\),有\(f^{(n)}(x)\geq0\).\emph{思考}\(f^{(n+1)}\)\emph{与}\(f^{(n)}\)\emph{之间的关系}.
\item
  设\(f(x)=\dfrac{1}{1+x^2}\).

  \begin{enumerate}
  \def\labelenumii{\arabic{enumii}.}
  \tightlist
  \item
    证明:对于任意自然数\(n\),存在多项式\(P_n\)使得\(f^{(n)}(x)=\dfrac{P_n(x)}{(x^2+1)^{n+1}}\).\emph{可以写}\(f(x)=\dfrac{1}{2i}(\dfrac{1}{x-i}-\dfrac{1}{x+i})\).
  \item
    给出\(P_n\)的根.
  \end{enumerate}
\item
  设\(f \in \mathcal{C}^n(I,\mathbb{R})\),且在\(x_1<x_2<\cdots<x_n\)处取\(0\)值.证明:\[\forall x \in I,\exists \xi \in I,f(x)=\frac{(x-x_1)(x-x_2)\cdots(x-x_n)}{n!}f^{(n)}(\xi)\]
\item
  设\(f(x)=x^2-1\).对于任意自然数\(n\),设\(\mathbb{R}\)上的多项式函数\(P_n(x)=\frac{1}{2^nn!}D^n(f^n)\).

  \begin{enumerate}
  \def\labelenumii{\arabic{enumii}.}
  \tightlist
  \item
    证明:证明\({} P_n\)与\(n\)有相同的奇偶性.
  \item
    求\(P_n(1)\),进而给出\(P_n(-1)\).
  \item
    证明\(P_n\)在\((-1,1)\)上有\(n\)个零点.
  \end{enumerate}
\item
  设\(q \in \mathcal{C}^1(\mathbb{R},\mathbb{R})\)满足:\[\forall A \in \mathbb{R}_+,\exists x \geq A,q(x)>0 \text{且} q’(x)>0\]证明:满足\(y’’(x)+q(x)y(x)=0\)的二阶可导函数\(y\)在\(+\infty\)附近有界.\emph{可以考虑定义在}\([A,+\infty)\)\emph{上的函数}\(z(x)=y^2(x)+\frac{y’^2(x)}{q(x)}\).
\item
  寻找面积最大的三角形\(ABC\),它内接于以原点为圆心,以\(1\)为半径的圆.我们可以总是假设\(A,B\)在一条平行于轴\((O,\vec{i})\)的直线上.

  \begin{enumerate}
  \def\labelenumii{\arabic{enumii}.}
  \tightlist
  \item
    确定最大的三角形面积,对于\(A,B\)在直线\(Y=y\)上的三角形,其中\(y \in [-1,1]\).记这个面积为\(f(y)\).
  \item
    得出结论.
  \end{enumerate}
\item
  设\(f:I \to I\)可导,序列\((u_n)\)满足\(u_{n+1}=f(u_n),u_0 \in I\).再设\(c \in I\)为\(f\)的不动点,\(\left\lvert f'(c) \right\rvert>1\),\((u_n)\)收敛于\(c\).

  \begin{enumerate}
  \def\labelenumii{\arabic{enumii}.}
  \tightlist
  \item
    设\(u_n \neq c\),给出\(\lim\limits_{n\to\infty}\frac{u_{n+1}-c}{u_n-c}\).导出矛盾.
  \item
    证明该序列收敛于\(c\)当且仅当它从某项起为\(c\).
  \item
    证明:由\(u_0=\frac{1}{3},u_{n+1}=4u_n(1-u_n)\)定义的序列发散.
  \end{enumerate}
\item
  研究满足\(u_0 \in [0,2],u_{n+1}=\sqrt{2-u_n}\)的序列\((u_n)\).
\item
  考虑序列\((u_n)\)满足\(u_0 \in [0,\frac{\pi}{2}]\),\(u_{n+1}=\sin(2u_n)\)

  \begin{enumerate}
  \def\labelenumii{\arabic{enumii}.}
  \tightlist
  \item
    证明区间\([\frac{\pi}{4},1]\)在\(f:x \mapsto \sin(2x)\)下稳定.
  \item
    研究序列\((u_n)\).
  \end{enumerate}
\item
  \textbf{有限增量不等式}
  设\(f:[a,b]\to\mathbb{C}\)在\((a,b)\)上可导,存在实数\(M\)使得\(\left\lvert f'(x) \right\rvert \leqslant M\).

  \begin{enumerate}
  \def\labelenumii{\arabic{enumii}.}
  \tightlist
  \item
    在这一问中,设\(f(a),f(b)\)都是实数.证明\(\left\lvert f(a)-f(b) \right\rvert \leqslant M \left\lvert a-b \right\rvert\).
  \item
    不再假设\(f(a),f(b)\)是实数.考虑\(g(x)=u(f(x)-f(a))\),其中选取适当\(u \in \mathbb{C}\).证明\(\left\lvert f(a)-f(b) \right\rvert \leqslant M \left\lvert a-b \right\rvert\).
  \end{enumerate}
\end{enumerate}

\end{document}
